\section{Επεξεργασία Φυσικής Γλώσσας}
\label{sec:theory_nlp}

\subsection{Εισαγωγή στην Επεξεργασία Φυσικής Γλώσσας}
Η επεξεργασία φυσικής γλώσσας αποτελεί έναν από τους πλέον εξελισσόμενους τομείς της τεχνητής νοημοσύνης. Στόχος της είναι η κατανόηση, ανάλυση και εξαγωγή χαρακτηριστικών από κείμενα, δίνοντας τη δυνατότητα στα συστήματα να ερμηνεύουν και να επεξεργάζονται τη γλώσσα. Ο εξαιρετικά μεγάλος όγκος δεδομένων που ανταλλάσσονται διαρκώς υπό τη μορφή κειμένου καθιστά επιτακτική την ανάγκη για αυξημένη ασφάλεια. Στο πλαίσιο αυτής της εργασίας, το NLP αξιοποιείται για την εξαγωγή χαρακτηριστικών που αποτυπώνουν μοναδικές γλωσσικές και συμπεριφορικές πτυχές κάθε χρήστη.

Η κατανόηση των κειμένων περιλαμβάνει διαδικασίες όπως η αναγνώριση της δομής, η σημασιολογική ανάλυση και η δημιουργία μοναδικών γλωσσικών προφίλ. Αυτές οι διαδικασίες παίζουν κρίσιμο ρόλο στη δημιουργία συστημάτων συνεχούς αυθεντικοποίησης χρηστών, όπου ο στόχος είναι η ανίχνευση μοτίβων γραφής που επιτρέπουν την ασφαλή ταυτοποίηση.

Η εξέλιξη του κλάδου NLP ξεκίνησε με τις πρώτες γλωσσολογικές προσεγγίσεις, όπως οι τεχνικές Bag-of-Words και Term Frequency - Inverse Document Frequency (TF-IDF), οι οποίες βασίζονταν σε απλή μέτρηση συχνοτήτων λέξεων. Με την έλευση της μηχανικής μάθησης, η ανάλυση κειμένων πέρασε σε πιο σύνθετα επίπεδα, όπως η εξαγωγή γλωσσολογικών και σημασιολογικών χαρακτηριστικών.

Η εξέλιξη των εργαλείων NLP είναι ραγδαία. Αρχικά, το Word2Vec~\cite{mikolov2013efficient} εισήγαγε τη δυνατότητα εκμάθησης συσχετίσεων μεταξύ διανυσμάτων αναπαράστασης των λέξεων, οδηγώντας στην ανάλυση σημασιολογικών χαρακτηριστικών, πέρα από καθαρά γλωσσικών. Επίσης, το GloVe~\cite{pennington2014glove} βελτίωσε την προσέγγιση αυτή καθώς παρείχε στατιστικά σχετικά με τη συχνότητα των λέξεων μέσω της γεωμετρικής τους απεικόνισης σε πολυδιάστατους χώρους. Τέλος, μετασχηματιστές όπως ο BERT~\cite{devlin2018bert} αναπαριστούν το κείμενο ως μια σειρά από διανύσματα χρησιμοποιώντας επιβλεπόμενη μάθηση και μαθαίνουν λανθάνουσες αναπαραστάσεις των σημείων στο πλαίσιο των συμφραζομένων τους. Η τεχνική αυτή αποτέλεσε σημαντική βελτίωση έναντι των προηγούμενων μοντέλων.

Συνολικά, το NLP διαδραματίζει κεντρικό ρόλο στη δημιουργία μοναδικών προφίλ χρηστών. Μέσω της ανάλυσης γλωσσικών μοτίβων, επιτυγχάνεται η ταυτοποίηση χρηστών βασισμένη στη γραφή τους. Συστήματα αυθεντικοποίησης βασισμένα στο NLP χρησιμοποιούνται σε εφαρμογές ασφάλειας, όπου απαιτείται υψηλή ακρίβεια και αξιοπιστία.

\subsubsection{Tokenization}
Η διαίρεση του κειμένου σε λέξεις, φράσεις ή προτάσεις αποτελεί το πρώτο στάδιο επεξεργασίας. Το Tokenization επιτρέπει την απομόνωση σημαντικών τμημάτων του κειμένου για περαιτέρω ανάλυση.

\subsubsection{Stemming και Lemmatization}
Η απλοποίηση λέξεων στην αρχική τους μορφή βελτιώνει την ακρίβεια της γλωσσικής ανάλυσης. Το Stemming αφαιρεί τα προσφύματα των λέξεων, ενώ το Lemmatization διατηρεί τη γραμματική ακεραιότητα.

\subsubsection{Part-of-Speech Tagging}
Η επισήμανση της γραμματικής κατηγορίας (π.χ. ουσιαστικά, ρήματα) παρέχει σημαντικές πληροφορίες για τη σύνταξη και τη σημασιολογία.

\subsubsection{Named Entity Recognition (NER)}
Το NER αναγνωρίζει οντότητες, όπως ονόματα, ημερομηνίες ή τοποθεσίες, βοηθώντας στη δημιουργία πλούσιων γλωσσικών προφίλ.

\subsubsection{Dependency Parsing}
Αναλύει τις συντακτικές σχέσεις μεταξύ λέξεων, αποκαλύπτοντας τη δομή του κειμένου.

\subsection{Εργαλεία και Μέθοδοι Εξαγωγής Χαρακτηριστικών}
Τα χαρακτηριστικά εξάγονται με τη χρήση εργαλείων όπως:
\begin{itemize}
    \item \textbf{NLTK\footnote{\url{https://www.nltk.org/}} και spaCy\footnote{\url{https://spacy.io/}}:} Για μορφολογική και συντακτική ανάλυση.
    \item \textbf{textstat\footnote{\url{https://texstat.org/}}:} Για δείκτες αναγνωσιμότητας και πολυπλοκότητας.
\end{itemize}

