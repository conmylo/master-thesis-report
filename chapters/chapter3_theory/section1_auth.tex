\section{Αυθεντικοποίηση}
\label{sec:theory_auth}

Η αυθεντικοποίηση αποτελεί θεμελιώδη διαδικασία στον τομέα της ασφάλειας πληροφοριακών συστημάτων. Στόχος της είναι η επαλήθευση της ταυτότητας ενός χρήστη ή μιας συσκευής προτού παραχωρηθεί πρόσβαση σε δεδομένα ή υπηρεσίες. Η ανάγκη για αξιόπιστες μεθόδους αυθεντικοποίησης γίνεται ολοένα και πιο επιτακτική, εξαιτίας της αυξανόμενης πολυπλοκότητας των απειλών κυβερνοασφάλειας και των επιτιθέμενων που αναζητούν διαρκώς τρόπους να παρακάμψουν τα παραδοσιακά συστήματα ελέγχου ταυτότητας.


\subsection{Ορισμός Αυθεντικοποίησης}
Η αυθεντικοποίηση αναφέρεται στη διαδικασία επαλήθευσης της ταυτότητας ενός χρήστη ή συσκευής προτού επιτραπεί η πρόσβαση σε ένα σύστημα. Η διαδικασία περιλαμβάνει την ταυτοποίηση, δηλαδή τη δήλωση της ταυτότητας, και την επαλήθευση, που επιβεβαιώνει την ακρίβεια της δήλωσης. Για παράδειγμα, ένας χρήστης μπορεί να δηλώσει την ταυτότητά του μέσω του ονόματος χρήστη (ταυτοποίηση) και να την επαληθεύσει μέσω ενός κωδικού πρόσβασης ή μιας βιομετρικής μεθόδου (επαλήθευση).


\subsection{Κατηγορίες Τεχνικών Αυθεντικοποίησης}
Η εξέλιξη της τεχνολογίας έχει οδηγήσει στην ανάπτυξη ποικίλων συστημάτων αυθεντικοποίησης, που απαιτούν από τον χρήστη κάτι διαφορετικό κάθε φορά για την επαλήθευση της ταυτότητάς του. Η κατηγοριοποίηση αυτή επομένως μπορεί να αναλυθεί σε:
\begin{itemize}
    \item \textbf{Γνώση}: Αφορά κάτι που ο χρήστης γνωρίζει και συμπεριλαμβάνει κωδικούς πρόσβασης / PIN, μοτίβο ή απάντηση σε κάποια ερώτηση.
    \item \textbf{Κατοχή}: Αφορά κάτι που ο χρήστης κατέχει και συμπεριλαμβάνει φυσικά αντικείμενα όπως κάρτες ή tokens.
    \item \textbf{Βιομετρικά χαρακτηριστικά}: Αφορά έμφυτα φυσιολογικά ή συμπεριφορικά χαρακτηριστικά, γνωστά και ως βιομετρικά, που είναι μοναδικά για κάθε άτομο. Οι βιομετρικές τεχνικές χωρίζονται σε δύο κύριες κατηγορίες:
\begin{enumerate}
    \item \textbf{Φυσιολογικές Μέθοδοι}:
    \begin{itemize}
        \item \textit{Αναγνώριση προσώπου}: Χρήση χαρακτηριστικών του προσώπου για την ταυτοποίηση.
        \item \textit{Δακτυλικά αποτυπώματα}: Καταγραφή και αντιστοίχιση μοναδικών αποτυπωμάτων.
        \item \textit{Σάρωση ίριδας}: Εξαιρετικά ασφαλής μέθοδος, αλλά απαιτεί εξειδικευμένο εξοπλισμό.
    \end{itemize}
    \item \textbf{Συμπεριφορικές Μέθοδοι}:
    \begin{itemize}
        \item \textit{Ανάλυση γραφής}: Εξαγωγή χαρακτηριστικών από τον τρόπο που γράφει ένας χρήστης.
        \item \textit{Δυναμική πληκτρολόγηση}: Παρακολούθηση των μοτίβων πληκτρολόγησης.
        \item \textit{Συμπεριφορά πλοήγησης}: Ανάλυση τρόπων πλοήγησης σε περιβάλλοντα χρήστη.
    \end{itemize}
\end{enumerate}

Οι βιομετρικές μέθοδοι, σε αντίθεση με τις παραδοσιακές, δεν μπορούν εύκολα να παραβιαστούν, καθώς βασίζονται σε εγγενή χαρακτηριστικά του χρήστη.

\end{itemize}

Ακόμη, μπορούμε να διακρίνουμε 2 κατηγορίες τεχνικών αυθεντικοποίησης ανάλογα με τη διαφάνεια του συστήματος ως προς τον τελικό χρήστη.

\begin{itemize}
    \item \textbf{Ενεργή / Άμεση}: το σύστημα απαιτεί την εισαγωγή δεδομένων από τον χρήστη, όπως την πληκτρολόγηση ενός κωδικού ή μίας απάντησης σε μία ερώτηση ασφαλείας.
    \item \textbf{Παθητική / Έμμεση}: εκτελείται στο παρασκήνιο, χωρίς να χρειάζεται ενέργεια από τον χρήστη. Συστήματα αυτής της κατηγορίας ξεχωρίζουν για την δυνατότητά τους να εκτελούνται συνεχώς χωρίς να επεμβαίνουν στην λειτουργικότητα της συσκευής.
\end{itemize}

Σήμερα πολλές υπηρεσίες και εφαρμογές χρησιμοποιούν συνδυασμούς τεχνικών αυθεντικοποίησης (Multi Factor Authentication - MFA). Συχνότερο παράδειγμα αποτελεί η αυθεντικοποίηση 2 παραγόντων (two-factor authentication ή 2FA). Στη συγκεκριμένη κατηγορία εμπίπτουν η ανάληψη χρημάτων από το ATM με την χρήση κάρτας (κατοχή) και την πληκτρολόγηση του PIN (γνώση), αλλά και η σύνδεση σε λογαριασμούς ηλεκτρονικού ταχυδρομείου με τον κωδικό πρόσβασης (γνώση) και ένα συνθηματικό που αποστέλλεται σε κάποια συσκευή του χρήστη (κατοχή).

\subsection{Συνεχής και Έμμεση Αυθεντικοποίηση βασιζόμενη σε συμπεριφορικές μεθόδους}
Η συνεχής και έμμεση αυθεντικοποίηση αποτελεί μία καινοτόμο προσέγγιση στον τομέα της ασφάλειας πληροφοριακών συστημάτων, η οποία δίνει έμφαση στην αδιάλειπτη και μη παρεμβατική επαλήθευση της ταυτότητας του χρήστη. Σε αντίθεση με τις παραδοσιακές μεθόδους αυθεντικοποίησης, οι οποίες συχνά απαιτούν τη ρητή συμμετοχή του χρήστη, όπως η εισαγωγή κωδικών πρόσβασης ή η χρήση βιομετρικών αναγνωριστικών, η συνεχής αυθεντικοποίηση αξιοποιεί πληροφορίες που συλλέγονται από τη συμπεριφορά του χρήστη και τις αλληλεπιδράσεις του με το σύστημα.

Οι τεχνικές συνεχούς και έμμεσης αυθεντικοποίησης χρησιμοποιούν δεδομένα όπως:
\begin{itemize}
    \item Τα μοτίβα γραφής και πληκτρολόγησης, που αναλύονται μέσω τεχνικών επεξεργασίας φυσικής γλώσσας (NLP) και μηχανικής μάθησης.
    \item Τα μοτίβα πλοήγησης σε ψηφιακά περιβάλλοντα, τα οποία παρέχουν στοιχεία σχετικά με τη συμπεριφορά του χρήστη.
    \item Βιομετρικά δεδομένα χαμηλής συχνότητας, όπως η δυναμική χρήσης της συσκευής (π.χ., κλίση ή ταχύτητα κύλισης).
\end{itemize}

Η συνεχής και έμμεση αυθεντικοποίηση προσφέρει σημαντικά πλεονεκτήματα:
\begin{enumerate}
    \item \textbf{Αυξημένη ασφάλεια:} Η συνεχής παρακολούθηση καθιστά δυσκολότερη την παραβίαση του συστήματος.
    \item \textbf{Μη παρεμβατική λειτουργία:} Οι χρήστες δεν χρειάζεται να διακόπτουν τη ροή των ενεργειών τους για να επαληθεύσουν την ταυτότητά τους.
    \item \textbf{Δυναμική προσαρμογή:} Τα συστήματα μπορούν να προσαρμόζονται στις αλλαγές της συμπεριφοράς του χρήστη, βελτιώνοντας τη συνολική ακρίβεια.
    \item \textbf{Εύκολη ενσωμάτωση:} Μπορεί να ενσωματωθεί σε υπάρχον hardware, χωρίς να απαιτούνται πρόσθετα κόστη εξοπλισμού.
    \item \textbf{Ευελιξία:} Μπορούν να χρησιμοποιηθούν πολλά διαφορετικά συμπεριφορικά χαρακτηριστικά ανάλογα με τις απαιτήσεις και το τελικό προϊόν που θα εξυπηρετεί το σύστημα. 
\end{enumerate}

Ένα παράδειγμα εφαρμογής αυτής της τεχνολογίας είναι η ανάλυση γραφής για αυθεντικοποίηση σε περιβάλλοντα συνομιλιών. Η τεχνική αυτή βασίζεται στην εξαγωγή χαρακτηριστικών, όπως η επιλογή λέξεων, η σύνταξη και η δομή των προτάσεων, που αποτελούν μοναδικά χαρακτηριστικά του χρήστη. Σε συνδυασμό με τεχνολογίες όπως οι αλγόριθμοι ανίχνευσης ανωμαλιών, η συνεχής αυθεντικοποίηση μπορεί να εξασφαλίσει ένα υψηλό επίπεδο ασφάλειας, χωρίς να επηρεάζει την εμπειρία του χρήστη.

\begin{figure}[H]
    \centering
    \includegraphics[width=\textwidth]{images/chapter3/CIAtheoreticalBackground.png}
    \caption{Πτυχές της συνεχούς και έμμεσης αυθεντικοποίησης}
    \label{fig:chapter3_CIA}
\end{figure}

Η χρήση τέτοιων τεχνολογιών ανοίγει νέες προοπτικές για εφαρμογές όπως η προστασία προσωπικών δεδομένων, η ασφαλής πρόσβαση σε κρίσιμες υποδομές, και η βελτίωση της εμπειρίας των χρηστών σε ψηφιακά περιβάλλοντα.
