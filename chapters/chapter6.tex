\chapter{Συμπεράσματα}
\label{chapter:conclusions}

Στο κεφάλαιο αυτό παρουσιάζονται συνοπτικά τα συμπεράσματα που προέκυψαν από την έκβαση των πειραμάτων.
Γίνεται σύγκριση της απόδοσης των μοντέλων έχοντας ως
παραμέτρους αξιολόγησης τις μετρικές FAR, FRR, Mean Accepted Prompts by Genuine User, Mean Accepted Prompts by Impostor User.
Στην συνέχεια αναφέρονται τα προβλήματα που παρουσιάστηκαν
κατά την διάρκεια των υλοποιήσεων και των πειραμάτων.

\section{Γενικά Συμπεράσματα}
Στα πλαίσια της παρούσας διπλωματικής εργασίας υλοποιήθηκε ένα σύστημα συνεχούς και έμμεσου ελέγχου ταυτότητας σε διαλογικά περιβάλλοντα. Η υλοποίηση του συστήματος βρίσκεται στο Github\footnote{Υλοποίηση συστήματος συνεχούς και έμμεσου ελέγχου ταυτότητας: \url{https://github.com/conmylo/master-thesis/tree/main/final}}.

Με βάση τα αποτελέσματα των πειραμάτων (βλ. \autoref{chapter:experiments}) που αφορούν στις μετρικές FAR, FRR, Mean Accepted Prompts Before Locking by Genuine User \& Mean Accepted Prompts Before Locking by Impostor User παρατηρήθηκαν τα εξής:

\begin{itemize}
    \item{
        Οι καλύτερες μετρικές των ποσοστών λανθασμένης αποδοχής (FAR: 26.87\%) και λανθασμένης απόρριψης (FRR: 17.80\%) σημειώθηκαν για το παρακάτω πλέγμα υπερπαραμέτρων: 
        \[
        \nu \in \{0.001, 0.005, 0.01\}, \quad \gamma \in \{0.05, 0.07, 0.1, 0.15, 0.2, 0.5\}
        \]
    }
    \item{
        Η ενσωμάτωση πολλαπλών μοντέλων OC-SVM με τη μέθοδο σταθμισμένης πλειοψηφίας έναντι της βασικής πλειοψηφικής συνάρτησης βελτίωσε τη συνολική ακρίβεια κατά 11\%, καθώς επέτρεψε την καλύτερη ισορροπία μεταξύ των μετρικών FAR και FRR.
    }
    \item{
        Η χρήση της συνάρτησης εμπιστοσύνης (confidence level) συνέβαλε στη σταθερότερη απόδοση του συστήματος, ιδιαίτερα όταν προσαρμόστηκε δυναμικά στις εναλλαγές μεταξύ γνήσιων και μη γνήσιων προτροπών, στη παρακολούθηση διαδοχικών ταξινομήσεων και τη βεβαιότητα της κάθε απόφασης.
    }
    \item{
        Τα αποτελέσματα έδειξαν ότι η εισαγωγή δεδομένων από διαφορετικούς χρήστες - 14 χρήστες έναντι 8 αρχικά - αύξησε την διακύμανση των τιμών των μετρικών κατά $17\%$, αλλά απαιτούσε $140\%$ περισσότερους υπολογιστικούς πόρους για την εκπαίδευση και τον έλεγχο.
    }
    \item{
    \begin{sloppypar}
        Η σύγκριση των παραμέτρων αξιολόγησης (Mean Accepted Prompts by Genuine User: $42.22$, Mean Accepted Prompts by Impostor User: $3.03$) ανέδειξε ότι το σύστημα διατηρεί υψηλά ποσοστά ακρίβειας, αποτρέποντας τη μη εξουσιοδοτημένη πρόσβαση χωρίς να ενοχλεί υπερβολικά τον γνήσιο χρήστη.
    \end{sloppypar}
    }
\end{itemize}

Επιπλέον, παρατηρήθηκαν και τα εξής συμπεράσματα:

\begin{itemize}
    \item[-]
    Η μείωση των ποσοστών FAR και FRR απαιτεί λεπτομερή ρύθμιση των υπερπαραμέτρων και κατάλληλη επιλογή χαρακτηριστικών, οδηγώντας σε μείωση των σφαλμάτων κατά 20\%.
    \item[-]
    Η απόδοση του συστήματος μπορεί να μεταβληθεί ανάλογα με το περιβάλλον χρήσης του. 
    \begin{itemize}
        \item{
        Για περιβάλλοντα με λίγα prompts ανά συνεδρία (social media, emails), με ανάλογη παραμετροποίηση του πλέγματος, προσαρμόζουμε τις μετρικές σε: Mean Accepted Prompts by Genuine User: 42.22, Mean Accepted Prompts by Impostor User: 3.03
        }
        \item{
        Για περιβάλλοντα με πολλά prompts ανά συνεδρία (διαλογικά περιβάλλοντα), με ανάλογη παραμετροποίηση του πλέγματος, προσαρμόζουμε τις μετρικές σε: Mean Accepted Prompts by Genuine User: 86.85, Mean Accepted Prompts by Impostor User: 6.08
        }
    \end{itemize}
\end{itemize}


\section{Προβλήματα}
Ένα από τα αρχικά προβλήματα που παρουσιάστηκαν κατά την διάρκεια των υλοποιήσεων ήταν η έλλειψη κατάλληλου συνόλου δεδομένων. Τα διαθέσιμα σετ δεδομένων δεν συνδύαζαν ταυτόχρονα επαρκή αριθμό καταχωρίσεων ανά χρήστη, σύντομα κείμενα και ικανοποιητικό αριθμό από χρήστες. Ο περιορισμένος αριθμός χρηστών στα δεδομένα που χρησιμοποιήθηκαν κατέστησε δύσκολη τη γενίκευση του μοντέλου σε διαφορετικά δημογραφικά χαρακτηριστικά και την αξιολόγηση του συστήματος να διαχειρίζεται ετερογενή προφίλ.

Ένα δεύτερο πρόβλημα αφορούσε τον ασαφή τρόπο γραφής ορισμένων χρηστών, ο οποίος δυσχέραινε την εκπαίδευση των μοντέλων. Οι χρήστες που δεν είχαν σταθερά μοτίβα γραφής και χρησιμοποιούσαν διαφορετικά γλωσσικά και συμπεριφορικά χαρακτηριστικά από φορά σε φορά εμπόδισαν την αποδοτικότητα του συστήματος συνολικά. Ωστόσο, συνέβαλαν στην συνειδητοποίηση πως το σύστημα θα πρέπει να γίνει περισσότερο ευέλικτο και ανθεκτικό σε διαφορετικές συνήθειες και τρόπους γραφής.

Επιπλέον, εμπόδιο αποτέλεσαν και οι υψηλές απαιτήσεις υπολογιστικής ισχύος και χρόνου που απαιτήθηκαν για τη διεξαγωγή των πειραμάτων και τη λήψη αποτελεσμάτων. Η εκτέλεση δοκιμών με τη χρήση πλεγμάτων υπερπαραμέτρων (άνω των 40-50 συνδυασμών) ήταν ιδιαίτερα χρονοβόρα, τονίζοντας όμως την ανάγκη για βελτιστοποίηση των διαδικασιών εκπαίδευσης και μείωσης των κύριων συνιστωσών των χαρακτηριστικών που εξαγάγαμε.

Τέλος, η δυσκολία μείωσης των μετρικών FAR και FRR σε ικανοποιητικά επίπεδα αποτέλεσε μία από τις μεγαλύτερες προκλήσεις της εργασίας. Παρά τη χρήση τεχνικών και την ενσωμάτωση δυναμικά μεταβαλλόμενων συναρτήσεων, όπως της confidence level και της weighted majority voting function, η εύρεση ισορροπίας μεταξύ χαμηλών επιπέδων FAR και FRR αποδείχθηκε ιδιαίτερα απαιτητική.
