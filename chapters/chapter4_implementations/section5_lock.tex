\section{Σύστημα Κλειδώματος/Εμπιστοσύνης}
\label{sec:implementation_lock}

\subsection{Συνάρτηση Εμπιστοσύνης}
Η ασφάλεια ενός συστήματος αυθεντικοποίησης εξαρτάται όχι μόνο από την ακρίβεια των μοντέλων μηχανικής μάθησης, αλλά και από τη δυνατότητά του να διαχειρίζεται καταστάσεις όπου οι αποφάσεις μπορεί να είναι αβέβαιες ή να υπόκεινται σε διαδοχικά λάθη. Το σύστημα κλειδώματος/εμπιστοσύνης ενσωματώνει έναν μηχανισμό παρακολούθησης του επιπέδου εμπιστοσύνης ($C$) του συστήματος προς τον χρήστη. Αυτό το επίπεδο αυξάνεται ή μειώνεται ανάλογα με την απόδοση του χρήστη, και σε περίπτωση που το $C$ πέσει κάτω από ένα όριο ($\text{confidence\_threshold}$), το σύστημα ενεργοποιεί μηχανισμούς κλειδώματος για την προστασία από κακόβουλη χρήση.

\subsubsection{Συνάρτηση Εμπιστοσύνης}
Η συνάρτηση εμπιστοσύνης περιγράφεται μαθηματικά ως:

\begin{figure}[H]
    \centering
    \includegraphics[width=\textwidth]{images/chapter4/confidenceLevelFunc.png}
    \caption{Συνάρτηση Confidence Level}
    \label{fig:confidenceLevelFunction}
\end{figure}

% \[
% C = 
% \begin{cases} 
% C + \text{base}_{\text{increase}} 
% + 
% \begin{cases} 
% \text{high\_certainty\_boost\_increase}, & \text{if } \text{certaintyScore} > \text{high\_certainty\_threshold} \\
% 0, & \text{otherwise}
% \end{cases} 
% +
% \begin{cases} 
% \text{consecutive\_genuine\_boost}, & \text{if } \text{consec}_{\text{genuine}} \mod 3 = 0 \\
% 0, & \text{otherwise}
% \end{cases}, 
% & \text{for genuine decisions} \\[10pt]

% C - \text{base}_{\text{decrease}} 
% -
% \begin{cases} 
% \text{high\_certainty\_boost\_decrease}, & \text{if } \text{certaintyScore} > \text{high\_certainty\_threshold} \\
% 0, & \text{otherwise}
% \end{cases} 
% -
% \begin{cases} 
% \text{consecutive\_impostor\_penalty}, & \text{if } \text{consec}_{\text{impostor}} \mod 2 = 0 \\
% 0, & \text{otherwise}
% \end{cases}, 
% & \text{for impostor decisions}
% \end{cases}
% \]

\subsubsection{Βασικές Σταθερές}
Ύστερα από μεγάλο αριθμό δοκιμών, οι τιμές των βασικών παραμέτρων της συνάρτησης του σχήματος~\ref{fig:confidenceLevelFunction} καθορίστηκαν στις παρακάτω. Σε επόμενο κεφάλαιο συγκρίνεται η απόδοση της συνάρτησης εμπιστοσύνης με διαφορετικές τιμές των βασικών παραμέτρων.

\begin{itemize}
    \item $\text{base}_{\text{increase}} = 0.06$
    \item $\text{base}_{\text{decrease}} = 0.12$
    \item $\text{high\_certainty\_threshold} = 0.7$
    \item $\text{high\_certainty\_boost\_factor} = 0.4$
    \item $\text{consecutive\_genuine\_boost} = 0.04$
    \item $\text{consecutive\_impostor\_penalty} = 0.05$
    \item $\text{confidence\_threshold} = 0.3$
    \item αρχικό επίπεδο εμπιστοσύνης: $C_0 = 0.6$
\end{itemize}

\subsubsection{Υπολογισμός Ενισχύσεων Υψηλής Βεβαιότητας}

Οι ενισχύσεις λόγω υψηλής βεβαιότητας υπολογίζονται ως:

\[
\text{high\_certainty\_boost\_increase} = \text{base}_{\text{increase}} \times \text{high\_certainty\_boost\_factor} = 0.06 \times 0.4 = 0.024
\]

\[
\text{high\_certainty\_boost\_decrease} = \text{base}_{\text{decrease}} \times \text{high\_certainty\_boost\_factor} = 0.12 \times 0.4 = 0.048
\]

\paragraph{Επίπεδο Βεβαιότητας ($\text{certaintyScore}$)}

Το επίπεδο βεβαιότητας υπολογίζεται με βάση την \texttt{certainty\_level\_function} ως:

\[
\text{certaintyScore} = \frac{\lvert \text{απόσταση απόφασης} \rvert}{\text{μέγιστη απόσταση απόφασης}}
\]

\subsubsection{Ενίσχυση Διαδοχικών Αποφάσεων}
Ο όρος της ενίσχυσης διαδοχικών αποφάσεων προστίθεται ως ένα μέσο σταθερότητας στη συνάρτηση, ώστε να αποτρέπεται η υπερβολική ευαισθησία του συστήματος σε μεμονωμένες ανωμαλίες ή θορύβους. Με αυτόν τον τρόπο ενισχύεται η εμπιστοσύνη στον χρήστη όσο περισσότερο ο ίδιος χρησιμοποιεί το σύστημα, ενώ μειώνεται η εμπιστοσύνη όσο συχνότερα το χρησιμποιεί κάποιος εισβολέας. Βελτιώνεται, συνεπώς, και η συνολικότερη ακρίβεια του συστήματος.


- Για τις γνήσιες αποφάσεις, η ενίσχυση λόγω διαδοχικών γνήσιων αποφάσεων εφαρμόζεται όταν ο αριθμός των διαδοχικών γνήσιων αποφάσεων ($\text{consec}_{\text{genuine}}$) είναι πολλαπλάσιο του 3:

\[
\text{consecutive\_genuine\_boost} = 
\begin{cases} 
0.04, & \text{if } \text{consec}_{\text{genuine}} \mod 3 = 0 \\
0, & \text{otherwise}
\end{cases}
\]

- Για τις αποφάσεις εισβολέα, η ποινή λόγω διαδοχικών αποφάσεων εισβολέα εφαρμόζεται όταν ο αριθμός των διαδοχικών αποφάσεων εισβολέα ($\text{consec}_{\text{impostor}}$) είναι πολλαπλάσιο του 2:

\[
\text{consecutive\_impostor\_penalty} = 
\begin{cases} 
0.05, & \text{if } \text{consec}_{\text{impostor}} \mod 2 = 0 \\
0, & \text{otherwise}
\end{cases}
\]

\subsubsection{Μηχανισμός Κλειδώματος}
Ο μηχανισμός κλειδώματος ενεργοποιείται όταν $C < \text{confidence\_threshold}$. Σε αυτή την περίπτωση:

\begin{itemize}
    \item Το σύστημα απαιτεί επανεξουσιοδότηση μέσω πρόσθετων στοιχείων ταυτοποίησης.
    \item Ο δείκτης $C$ επανέρχεται στο $C_0$ μετά από επιτυχημένη επανεξουσιοδότηση.
\end{itemize}

\subsection{Παραδείγματα Λειτουργίας}
Παρακάτω παρουσιάζονται εκτενώς παραδείγματα λειτουργίας του συστήματος κλειδώματος/εμπιστοσύνης για διαφορετικά σενάρια. Κάθε σενάριο περιλαμβάνει ακολουθία αποφάσεων, το επίπεδο βεβαιότητας ($\text{certaintyScore}$), την αλλαγή στο επίπεδο εμπιστοσύνης ($\Delta C$), και το τελικό επίπεδο εμπιστοσύνης ($C$).

\paragraph{Παράδειγμα 1: Διαδοχικές Γνήσιες Αποφάσεις}
Σε αυτό το παράδειγμα, ο χρήστης λαμβάνει διαδοχικές γνήσιες αποφάσεις με διαφορετικά επίπεδα βεβαιότητας ($\text{certaintyScore}$). Παρατηρούμε ότι κάθε τρίτη διαδοχική γνήσια απόφαση ενισχύεται με τον παράγοντα $\text{consecutive\_genuine\_boost}$:
\[
\Delta C = \text{base}_{\text{increase}} + \text{high\_certainty\_boost\_increase} + \text{consecutive\_genuine\_boost}.
\]

Η επίδραση των τιμών φαίνεται στον πίνακα \ref{tab:trust_examples_1}, όπου το επίπεδο εμπιστοσύνης αυξάνεται σημαντικά μετά από κάθε απόφαση.

\begin{table}[H]
\centering
\begin{tabular}{|c|c|c|c|c|c|}
\hline
\textbf{Απόφαση} & \textbf{$\text{Certainty}$} & \textbf{Διαδοχικές} & \textbf{True Label} & $\Delta C$ & $C$ \\
\hline
Γνήσιος & 0.8 & 1 & Γνήσιος & $+0.06 + 0.024$ & 0.684 \\
Γνήσιος & 0.9 & 2 & Γνήσιος & $+0.06 + 0.024$ & 0.768 \\
Γνήσιος & 0.85 & 3 & Γνήσιος & $+0.06 + 0.024 + 0.04$ & 0.892 \\
Εισβολεάς & 0.82 & 1 & Εισβολεάς & $-0.12 - 0.048$ & 0.724 \\
Εισβολεάς & 0.6 & 2 & Γνήσιος & $-0.12 - 0.05$ & 0.554 \\
Γνήσιος & 0.55 & 1 & Εισβολεάς & $+0.06$ & 0.614 \\
Εισβολεάς & 0.8 & 1 & Εισβολεάς & $-0.12 - 0.048$ & 0.446 \\
\hline
\end{tabular}
\caption{Παραδείγματα Ενημέρωσης Εμπιστοσύνης}
\label{tab:trust_examples_1}
\end{table}

\paragraph{Παράδειγμα 2: Εναλλαγή Γνήσιων και Εισβολέων}
Σε αυτή την περίπτωση, εναλλάσσονται γνήσιες και απατηλές αποφάσεις. Ο πίνακας \ref{tab:trust_examples_2} δείχνει τη σταδιακή μείωση του επιπέδου εμπιστοσύνης λόγω αποφάσεων εισβολέα:

\begin{table}[H]
\centering
\begin{tabular}{|c|c|c|c|c|c|}
\hline
\textbf{Απόφαση} & \textbf{$\text{Certainty}$} & \textbf{Διαδοχικές} & \textbf{True Label} & $\Delta C$ & $C$ \\
\hline
Γνήσιος & 0.75 & 1 & Γνήσιος & $+0.06 + 0.024$ & 0.684 \\
Εισβολεάς & 0.65 & 1 & Γνήσιος & $-0.12$ & 0.564 \\
Γνήσιος & 0.8 & 1 & Γνήσιος & $+0.06 + 0.024$ & 0.648 \\
Εισβολεάς & 0.55 & 1 & Γνήσιος & $-0.12$ & 0.528 \\
Γνήσιος & 0.9 & 1 & Γνήσιος & $+0.06 + 0.024$ & 0.612 \\
Εισβολεάς & 0.78 & 1 & Εισβολεάς & $-0.12 - 0.048$ & 0.444 \\
\hline
\end{tabular}
\caption{Παραδείγματα Εναλλαγής Γνήσιων και Απατηλών Αποφάσεων}
\label{tab:trust_examples_2}
\end{table}

\paragraph{Παράδειγμα 3: Απώλεια Εμπιστοσύνης και Επανεξουσιοδότηση}
Εάν το επίπεδο εμπιστοσύνης πέσει κάτω από το κατώφλι $\text{confidence\_threshold} = 0.3$, ενεργοποιείται ο μηχανισμός κλειδώματος, όπως φαίνεται στον πίνακα \ref{tab:trust_examples_3}. Επίσης, φαίνεται πώς ο μηχανισμός επαναφέρει το επίπεδο εμπιστοσύνης μετά από επιτυχημένη επανεξουσιοδότηση:

\begin{table}[H]
\centering
\begin{tabular}{|c|c|c|c|c|c|}
\hline
\textbf{Απόφαση} & \textbf{$\text{Certainty}$} & \textbf{Διαδοχικές} & \textbf{True Label} & $\Delta C$ & $C$ \\
\hline
Εισβολεάς & 0.6 & 2 & Εισβολεάς & $-0.12 - 0.048 - 0.05$ & 0.102 \\
Κλείδωμα & - & - & - & - & 0.6 \\
Γνήσιος & 0.85 & 1 & Γνήσιος & $+0.06 + 0.024$ & 0.684 \\
\hline
\end{tabular}
\caption{Παραδείγματα Ενεργοποίησης Μηχανισμού Κλειδώματος}
\label{tab:trust_examples_3}
\end{table}

\paragraph{Παράδειγμα 4: Συνεχής Ενίσχυση λόγω Υψηλής Βεβαιότητας}
Σε αυτό το σενάριο, όλες οι αποφάσεις είναι γνήσιες και συνοδεύονται από υψηλή βεβαιότητα ($\text{certaintyScore} > \text{high\_certainty\_threshold}$). Στον πίνακα \ref{tab:trust_examples_4} παρατηρούμε τη σημαντική ενίσχυση του επιπέδου εμπιστοσύνης:

\begin{table}[H]
\centering
\begin{tabular}{|c|c|c|c|c|c|}
\hline
\textbf{Απόφαση} & \textbf{$\text{Certainty}$} & \textbf{Διαδοχικές} & \textbf{True Label} & $\Delta C$ & $C$ \\
\hline
Γνήσιος & 0.9 & 1 & Γνήσιος & $+0.06 + 0.024$ & 0.684 \\
Γνήσιος & 0.95 & 2 & Γνήσιος & $+0.06 + 0.024$ & 0.768 \\
Γνήσιος & 0.92 & 3 & Γνήσιος & $+0.06 + 0.024 + 0.04$ & 0.892 \\
Γνήσιος & 0.94 & 1 & Γνήσιος & $+0.06 + 0.024$ & 0.976 \\
Γνήσιος & 0.97 & 2 & Γνήσιος & $+0.06 + 0.024$ & 1.060 \\
\hline
\end{tabular}
\caption{Παραδείγματα Συνεχούς Ενίσχυσης Λόγω Υψηλής Βεβαιότητας}
\label{tab:trust_examples_4}
\end{table}

\paragraph{Παράδειγμα 5: Επανάληψη Κλειδώματος Λόγω Απατηλών Αποφάσεων}
Σε αυτό το σενάριο, ο χρήστης λαμβάνει συστηματικά αποφάσεις εισβολέα, προκαλώντας επαναλαμβανόμενη ενεργοποίηση του μηχανισμού κλειδώματος, όπως φαίνεται στον πίνακα \ref{tab:trust_examples_5}:

\begin{table}[H]
\centering
\begin{tabular}{|c|c|c|c|c|c|}
\hline
\textbf{Απόφαση} & \textbf{$\text{Certainty}$} & \textbf{Διαδοχικές} & \textbf{True Label} & $\Delta C$ & $C$ \\
\hline
Εισβολεάς & 0.86 & 1 & Εισβολεάς & $-0.12 - 0.048$ & 0.432 \\
Εισβολεάς & 0.92 & 2 & Εισβολεάς & $-0.12 - 0.048 - 0.05$ & 0.214 \\
Κλείδωμα & - & - & - & - & 0.6 \\
Εισβολεάς & 0.91 & 1 & Εισβολεάς & $-0.12 - 0.048$ & 0.432 \\
Εισβολεάς & 0.83 & 2 & Εισβολεάς & $-0.12 - 0.048 - 0.05$ & 0.214 \\
Κλείδωμα & - & - & - & - & 0.6 \\
\hline
\end{tabular}
\caption{Παραδείγματα Επαναλαμβανόμενου Κλειδώματος}
\label{tab:trust_examples_5}
\end{table}


Το παράδειγμα δείχνει ότι η συνεχής απώλεια εμπιστοσύνης λόγω αποφάσεων εισβολέα οδηγεί σε συχνή ενεργοποίηση του μηχανισμού κλειδώματος. Αυτό εξασφαλίζει την προστασία του συστήματος από κακόβουλη χρήση.

\subsection{Παρατηρήσεις}
Τα παραπάνω παραδείγματα αναδεικνύουν τη λειτουργία του συστήματος εμπιστοσύνης σε διαφορετικά σενάρια χρήσης. Οι μαθηματικοί υπολογισμοί και οι διαδοχικές αποφάσεις παρουσιάζουν τη δυναμική φύση του μηχανισμού εμπιστοσύνης, ο οποίος μπορεί να προσαρμοστεί σε διαφορετικά μοτίβα συμπεριφοράς χρηστών. Ο συνδυασμός υψηλής βεβαιότητας, διαδοχικών αποφάσεων και μηχανισμού κλειδώματος διασφαλίζει την ασφάλεια και την αξιοπιστία του συστήματος.

Το σύστημα κλειδώματος/εμπιστοσύνης παρέχει ένα δυναμικό μέσο διαχείρισης αποφάσεων, ενισχύοντας την ασφάλεια και την αξιοπιστία. Η μαθηματική του θεμελίωση το καθιστά ικανό να προσαρμόζεται σε διαφορετικά σενάρια χρήσης.
