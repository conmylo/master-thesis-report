\section{Συλλογή και Προεπεξεργασία Δεδομένων}
\label{sec:implementations_data}

Η ποιότητα και η επεξεργασία των δεδομένων αποτελούν κρίσιμο στάδιο για την επιτυχία ενός συστήματος συνεχούς και έμμεσης αυθεντικοποίησης. Στο παρόν σύστημα, η διαδικασία συλλογής και προεπεξεργασίας δεδομένων έχει σχεδιαστεί ώστε να διασφαλίζει την ακεραιότητα και την αξιοπιστία των χαρακτηριστικών που χρησιμοποιούνται για την ταυτοποίηση των χρηστών.

\subsection{Πηγή Δεδομένων}

Τα δεδομένα που χρησιμοποιήθηκαν αντλήθηκαν από δημόσια διαθέσιμες πηγές, και συγκεκριμένα από το \textit{Kaggle\footnote{\url{https://www.kaggle.com}}
}, το οποίο προσφέρει ένα ευρύ φάσμα από σετ δεδομένων (datasets). Το συγκεκριμένο dataset περιλαμβάνει πολλαπλές εγγραφές από διάφορους χρήστες, παρέχοντας ένα πλούσιο σύνολο δεδομένων για την εξαγωγή χαρακτηριστικών και την εκπαίδευση. 

Επιπλέον, τα δεδομένα που περιλαμβάνται στο συγκεκριμένο dataset αντλούν τη θεματολογία τους από κάθε πτυχή της ανθρώπινης δραστηριότητας, διασφαλίζοντας την ποικιλομορφία στα στυλ γραφής. Η κατηγοριοποίηση βάσει χρηστών επιτρέπει τη δημιουργία εξατομικευμένων μοντέλων και εξασφαλίζει την απαραίτητη ανομοιογένεια τόσο στην εκπαίδευση των μοντέλων, όσο και στον έλεγχο και στην αξιολόγηση του συστήματος.

\subsection{Διαδικασία Προεπεξεργασίας Δεδομένων}

Η διαδικασία προεπεξεργασίας περιλάμβανε τα εξής βήματα:

\begin{enumerate}
    \item \textbf{Καθαρισμός dataset:} Αφαιρέθηκαν στήλες που παρείχαν περιττή πληροφορία σχετικά με την εργασία. Με αυτόν τον τρόπο μειώθηκε και το μέγεθος του dataset εξασφαλίζονταν ταχύτερους χρόνους ανάγνωσης και φόρτωσης των δεδομένων.
    \item \textbf{Καθαρισμός κειμένων:} Αφαιρέθηκαν από τα prompts τα URLs και HTML tags. Ο καθαρισμός διασφάλισε ότι τα δεδομένα περιείχαν μόνο πληροφοριακό περιεχόμενο.
\end{enumerate}

Η διαδικασία προεπεξεργασίας εφαρμόστηκε με χρήση βιβλιοθηκών Python, όπως οι \texttt{NLTK}, \texttt{spaCy}, και \texttt{textstat}, οι οποίες διευκόλυναν την εξαγωγή και την ανάλυση χαρακτηριστικών.

\subsection{Χαρακτηριστικά του Σετ Δεδομένων}

Το dataset\footnote{\url{https://www.kaggle.com/datasets/mmmarchetti/tweets-dataset}}
 περιλαμβάνει κείμενα από 14 διαφορετικούς χρήστες, με μέσο όρο 2.000 prompts ανά χρήστη. Ο πίνακας \ref{table:dataset_stats} συνοψίζει βασικά στατιστικά στοιχεία του dataset:

\begin{table}[h!]
\centering
\begin{tabular}{|l|r|}
\hline
\textbf{Χαρακτηριστικό} & \textbf{Τιμή} \\ \hline
Αριθμός Χρηστών & 14 \\ \hline
Συνολικά prompts ανά χρήστη & 2.000 \\ \hline
Μέσος Όρος Λέξεων ανά Κείμενο & Χ \\ \hline
Μέγιστο Μήκος Κειμένου & Χ \\ \hline
\end{tabular}
\caption{Στατιστικά Χαρακτηριστικά του Dataset}
\label{table:dataset_stats}
\end{table}

Επιπλέον, το dataset περιλαμβάνει τις παρακάτω στήλες, οι οποίες περιγράφονται στον πίνακα \ref{table:dataset_columns} και φαίνονται στο ~\autoref{fig:chapter4_datasetFormat}:

\begin{table}[h!]
\centering
\begin{tabular}{|l|l|}
\hline
\textbf{Στήλη} & \textbf{Περιγραφή} \\ \hline
\texttt{author} & Ταυτοποιητικό πεδίο του χρήστη που δημιούργησε το κείμενο. \\ \hline
\texttt{content} & Το κύριο σώμα του κειμένου, περιέχει το prompt προς ανάλυση. \\ \hline
\texttt{date\_time} & Ημερομηνία και ώρα δημιουργίας του κειμένου. \\ \hline
\texttt{id} & Μοναδικό αναγνωριστικό για κάθε prompt στο dataset. \\ \hline
\end{tabular}
\caption{Περιγραφή Στηλών του Dataset}
\label{table:dataset_columns}
\end{table}

\begin{figure}[H]
    \centering
    \includegraphics[width=0.7\textwidth]{images/chapter4/datasetFormat.png}
    \caption{Στιγμιότυπο οθόνης από το σετ δεδομένων που χρησιμοποιήθηκε - φαίνονται οι 4 στήλες που αναφέρονται παραπάνω καθώς και πολλαπλές καταχωρίσεις}
    \label{fig:chapter4_datasetFormat}
\end{figure}

Η ποικιλομορφία στα δεδομένα επιτρέπει την εκπαίδευση μοντέλων ικανά να αναγνωρίζουν διαφορετικά στυλ γραφής. Η διασφάλιση της ποιότητας και της ομοιογένειας του dataset αποτέλεσε θεμελιώδη παράγοντα για την επιτυχία των επόμενων σταδίων.

Τα δεδομένα που προκύπτουν μετά την προεπεξεργασία αποθηκεύονται στη δομή \texttt{data/filtered\_cleaned.csv} στο αποθετήριο. Η διαχείριση των δεδομένων γίνεται μέσω βιβλιοθηκών Python όπως \texttt{pandas\footnote{\url{https://pandas.pydata.org/}}
} για την ανάγνωση και εγγραφή αρχείων CSV και \texttt{os\footnote{\url{https://docs.python.org/3/library/os.html}}} για την οργάνωση των φακέλων.

