\section{Σύστημα απόφασης}
\label{sec:implementation_decision}

Το σύστημα απόφασης αποτελεί κρίσιμο μέρος της συνολικής αρχιτεκτονικής του συστήματος αυθεντικοποίησης. Αξιοποιεί τα χαρακτηριστικά που εξάγονται από τα δεδομένα, καθώς και τα αποτελέσματα των μοντέλων που εκπαιδεύτηκαν, για να παρέχει τεκμηριωμένες και αξιόπιστες αποφάσεις σχετικά με τη γνησιότητα του χρήστη.

\subsection{Εισαγωγή}
Η διαδικασία απόφασης συνδυάζει τα αποτελέσματα των εκπαιδευμένων μοντέλων με τη χρήση της συνάρτησης \emph{επιπέδου βεβαιότητας} και του αλγορίθμου \emph{σταθμισμένης πλειοψηφίας ψήφων}. Το τελικό αποτέλεσμα εξαρτάται από:
\begin{itemize}
    \item Την απόσταση από το hyperplane κάθε μοντέλου.
    \item Το επίπεδο βεβαιότητας που αντιστοιχεί σε κάθε απόσταση.
    \item Το κατώφλι απόφασης (\emph{decision threshold}) που καθορίζει τη συμπεριφορά του συστήματος και την κατηγοριοποίηση της απόφασης.
\end{itemize}

\subsection{Συνάρτηση Επιπέδου Βεβαιότητας}
Η συνάρτηση επιπέδου βεβαιότητας (\texttt{certainty\_level}) υπολογίζει το πόσο σίγουρο είναι ένα μοντέλο για την απόφασή του. Ορίζεται ως εξής:
\begin{equation}
    c(x) =
    \begin{cases} 
    \frac{|d(x)|}{d_{\text{max}}}, & \text{αν } |d(x)| < d_{\text{max}} \\
    1, & \text{αν } d(x) > d_{\text{max}} \text{ και } y = 1 \\
    -1, & \text{αν } d(x) > d_{\text{max}} \text{ και } y = -1
    \label{eq:certaintyLevel}
    \end{cases}
\end{equation}
όπου:
\begin{itemize}
    \item $d(x)$ είναι η απόσταση του σημείου εισόδου $x$ από το hyperplane του μοντέλου.
    \item $d_{\text{max}}$ είναι το μέγιστο όριο απόστασης που καθορίζει το επίπεδο βεβαιότητας.
    \item $y$ είναι η προβλεπόμενη ετικέτα (1 για γνήσιος χρήστης, -1 για εισβολέας).
\end{itemize}

\subsection{Υποσύστημα Ψηφοφορίας}
Σε αυτή την ενότητα παρουσιάζονται δύο διαφορετικές προσεγγίσεις για τη λήψη απόφασης, η απλή πλειοψηφική συνάρητηση και η σταθμισμένη πλειοψηφική συνάρτηση. Οι δύο αυτές μέθοδοι διαφέρουν ως προς τη πολυπλοκότητα και τη φιλοσοφία της διαδικασίας λήψης αποφάσεων. Η σύγκρισή τους πραγματοποιείται σε επόμενο κεφάλαιο.

\subsubsection{Απλή Πλειοψηφική Συνάρτηση}
Η πρώτη προσέγγιση συνδυάζει τα αποτελέσματα πολλαπλών μοντέλων με την ακόλουθη διαδικασία: κάθε μοντέλο $i$ παράγει μια πρόβλεψη $y_i$: 1 για γνήσιο χρήστη και -1 για εισβολέα. Το τελικό αποτέλεσμα υπολογίζεται ως εξής:
\begin{equation}
    \text{Decision} =
    \begin{cases} 
    1, & \text{αν } \sum_{i=1}^{N} y_i > 0 \\
    -1, & \text{αν } \sum_{i=1}^{N} y_i \leq 0
    \end{cases}
\end{equation}
όπου:
\begin{itemize}
    \item $N$ είναι το πλήθος των μοντέλων.
    \item $y_i$ είναι η πρόβλεψη του μοντέλου $i$.
\end{itemize}

\subsubsection{Σταθμισμένη Πλειοψηφική Συνάρτηση}
Η δεύτερη προσέγγιση συνδυάζει τα αποτελέσματα πολλαπλών μοντέλων χρησιμοποιώντας τη σταθμισμένη πλειοψηφική συνάρτηση. Κάθε μοντέλο $i$ παράγει μια πρόβλεψη $y_i$ και μια βαρύτητα ψήφου $w_i$, που βασίζεται στο επίπεδο βεβαιότητας. Το τελικό αποτέλεσμα υπολογίζεται ως εξής:
\begin{equation}
    \text{Decision} =
    \begin{cases} 
    1, & \text{αν } \sum_{i=1}^{N} w_i y_i > 0 \\
    -1, & \text{αν } \sum_{i=1}^{N} w_i y_i \leq 0
    \end{cases}
\end{equation}
όπου:
\begin{itemize}
    \item $w_i = |c_i(x)|$, το απόλυτο επίπεδο βεβαιότητας του μοντέλου $i$.
    \item $N$ είναι το πλήθος των μοντέλων.
    \item $y_i$ είναι η πρόβλεψη του μοντέλου $i$.
\end{itemize}

\subsection{Παραδείγματα Εφαρμογής}

Για να κατανοηθεί καλύτερα η λειτουργία του συστήματος λήψης αποφάσεων, παρατίθενται παραδείγματα εφαρμογής. Στο πλαίσιο αυτό, οι \(c_1, c_2, c_3, \ldots, c_n\) αντιπροσωπεύουν τις αποφάσεις που λαμβάνονται από διαφορετικά μοντέλα για ένα δείγμα \(x\). Κάθε \(c_i(x)\) εκφράζει τη βαθμολογία εμπιστοσύνης (\textit{confidence score}) που προκύπτει από το αντίστοιχο μοντέλο, με θετικές τιμές να υποδηλώνουν γνήσιο χρήστη (\textit{genuine user}) και αρνητικές τιμές να υποδηλώνουν εισβολέα (\textit{impostor}).

\begin{itemize}
    \item \textbf{Περίπτωση Γνήσιου Χρήστη:} 
    \[
    c_1(x) = 0.8, \quad c_2(x) = 0.7, \quad c_3(x) = 0.6
    \]
    - \textbf{Απλή Πλειοψηφία:} Όλα τα μοντέλα αποφασίζουν ότι πρόκειται για γνήσιο χρήστη και αναθέτουν τη τιμή 1 στα \(c_i(x)\):
      \[
      \text{Απλή Απόφαση} = 1 + 1 + 1 = 3 > 0 \quad (\text{Γνήσιος Χρήστης})
      \]
    - \textbf{Σταθμισμένη Πλειοψηφία:} Όλα τα μοντέλα αποφασίζουν ότι πρόκειται για γνήσιο χρήστη και αναθέτουν τη τιμή 1 πολλαπλασιασμένη με το βάρος της βεβαιότητας κάθε μοντέλου:
      \[
      \text{Σταθμισμένη Απόφαση} = 0.8 * 1 + 0.7 * 1 + 0.6 * 1 = 2.1 > 0 \quad (\text{Γνήσιος Χρήστης})
      \]
    
    \item \textbf{Περίπτωση Εισβολέα:} 
    \[
    c_1(x) = -0.8, \quad c_2(x) = -0.7, \quad c_3(x) = -0.6
    \]
    - \textbf{Απλή Πλειοψηφία:} Όλα τα μοντέλα αποφασίζουν ότι πρόκειται για εισβολέα και αναθέτουν τη τιμή -1 στα \(c_i(x)\):
      \[
      \text{Απλή Απόφαση} = (-1) + (-1) + (-1) = -3 < 0 \quad (\text{Εισβολέας})
      \]
    - \textbf{Σταθμισμένη Πλειοψηφία:} Όλα τα μοντέλα αποφασίζουν ότι πρόκειται για εισβολέα και αναθέτουν τη τιμή -1 πολλαπλασιασμένη με το βάρος της βεβαιότητας κάθε μοντέλου:
      \[
      \text{Σταθμισμένη Απόφαση} = 0.8 * (-1) + 0.7 * (-1) + 0.6 * (-1) = -2.1 < 0 \quad (\text{Εισβολέας})
      \]
    
    \end{itemize}

Με αυτόν τον τρόπο, η σταθμισμένη πλειοψηφική ψήφος λαμβάνει υπόψη τη βαρύτητα κάθε \(c_i(x)\), ενώ η βασική ψήφος βασίζεται αποκλειστικά στο πρόσημο της απόφασης του κάθε μοντέλου.
