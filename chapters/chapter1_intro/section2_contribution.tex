\section{Σκοπός - Συνεισφορά της Διπλωματικής Εργασίας}
\label{section:contribution}

Η παρούσα διπλωματική εργασία μελετά την ανάπτυξη και αξιολόγηση ενός συστήματος συνεχούς και έμμεσης αυθεντικοποίησης χρηστών, βασισμένου σε χαρακτηριστικά γραφής που εξάγονται μέσω τεχνικών επεξεργασίας φυσικής γλώσσας και μηχανικής μάθησης. Στόχος της εργασίας είναι η ανάπτυξη ενός συστήματος το οποίο μπορεί να ταυτοποιεί χρήστες με διακριτικό και συνεχόμενο τρόπο, αναγνωρίζοντας τον μοναδικό τρόπο γραφής τους. Η διαδικασία αυθεντικοποίησης πραγματοποιείται μέσω μοντέλων OC-SVM, τα οποία εκπαιδεύονται ώστε να αναγνωρίζουν αποκλίσεις από τη φυσιολογική γραφή του κάθε χρήστη, αποκλείοντας έτσι τους μη εξουσιοδοτημένους χρήστες από τη χρήση του εκάστοτε συστήματος.

Εξετάζεται η χρήση των τεχνικών NLP για την εξαγωγή γλωσσικών χαρακτηριστικών, όπως το μέσο μήκος λέξεων, η ποικιλία του λεξιλογίου και η δομή των προτάσεων, τα οποία μπορούν να αποδώσουν μια μοναδική ταυτότητα για κάθε χρήστη. Επιπλέον, παρουσιάζεται η εκπαίδευση και αξιολόγηση του μοντέλου OC-SVM για την έμμεση αυθεντικοποίηση, καθώς και η ανάλυση της αποτελεσματικότητας του προτεινόμενου συστήματος σε συνθήκες πραγματικής χρήσης. Η εργασία διερευνά την ακρίβεια και την απόδοση του συστήματος αυθεντικοποίησης, μετρώντας την αξιοπιστία του μέσω των δεικτών False Rejection Rate (FRR) και False Acceptance Rate (FAR).

Η εργασία συνεισφέρει στον τομέα της ψηφιακής ασφάλειας, προτείνοντας ένα σύστημα που προσφέρει διαρκή και αδιάλειπτη αυθεντικοποίηση χωρίς να απαιτεί συνεχείς ενέργειες από τον χρήστη, ενσωματώνοντας έτσι την επαλήθευση ταυτότητας στη φυσική ροή των καθημερινών δραστηριοτήτων. Παράλληλα, ανοίγει τον δρόμο για τη χρήση τεχνικών μηχανικής μάθησης και NLP στην αναγνώριση ταυτότητας χρηστών μέσω ανάλυσης γραφής, δημιουργώντας προοπτικές για την ανάπτυξη ασφαλών και ευέλικτων εφαρμογών σε περιβάλλοντα υψηλών απαιτήσεων.
