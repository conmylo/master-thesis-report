\section{Περιγραφή του Προβλήματος}
\label{section:problem_description}

Η ανάγκη για αυξημένη ασφάλεια στις ψηφιακές επικοινωνίες και την προστασία δεδομένων έχει καταστεί επιτακτική, ιδίως λόγω των συνεχώς εξελισσόμενων απειλών και της ψηφιοποίησης κάθε πτυχής της ανθρώπινης δραστηριότητας. Οι σύγχρονες τεχνολογίες απαιτούν συστήματα ταυτοποίησης που να μπορούν να παρέχουν διαρκή ασφάλεια, χωρίς να παρεμβαίνουν στη ροή ενεργειών του χρήστη. Οι παραδοσιακές μέθοδοι αυθεντικοποίησης, όπως η χρήση κωδικών πρόσβασης, παρουσιάζουν σοβαρούς περιορισμούς. Από τη μία, η πολυπλοκότητα και η συχνή ανανέωση των κωδικών απαιτεί συνεχή προσοχή από τον χρήστη, ενώ από την άλλη οι μέθοδοι αυτές είναι ευάλωτες σε επιθέσεις, όπως το phishing ή το brute-force.

Καθημερινά τεράστιος όγκος πληροφοριών δημιουργείται και ανταλλάσεται μέσω του γραπτού λόγου. Η συγκεκριμένη μορφή επικοινωνίας ωστόσο είναι ιδιαίτερα ευάλωτη σε κακόβουλες ενέργειες. Καθημερινά παραδείγματα αποτελούν τα spam emails, τα οποία επιχειρούν να εξαπατήσουν χρήστες για την αποκάλυψη ευαίσθητων δεδομένων, ενώ malicious tweets και αναρτήσεις σε κοινωνικά δίκτυα συχνά χρησιμοποιούνται για τη διασπορά παραπληροφόρησης. Παράλληλα, η διάδοση bots και deepfake τεχνικών δημιουργούν την ανάγκη για πιο εξελιγμένα μέσα ανίχνευσης και προστασίας. Σε τέτοιες περιπτώσεις, η χρήση παραδοσιακών μεθόδων αυθεντικοποίησης, όπως κωδικοί πρόσβασης ή βιομετρικά δεδομένα, δεν επαρκεί. Ακόμα και αν εξασφαλίσουν την αρχική πρόσβαση, δεν παρέχουν διαρκή προστασία καθ' όλη τη διάρκεια χρήσης της υπηρεσίας, αφήνοντας τα συστήματα ευάλωτα σε δυνητικές επιθέσεις. Είναι επιτακτική, επομένως, η ανάγκη για περισσότερο δυναμικές μεθόδους αυθεντικοποίησης, που θα ενσωματώνουν γλωσσικά χαρακτηριστικά μέσω της ανάλυσης γραφής 
και θα προσφέρουν μια διακριτική και αξιόπιστη λύση, επιτρέποντας τη συνεχή παρακολούθηση της ταυτότητας του χρήστη με βάση το προσωπικό του στυλ γραφής.

Η έμμεση αυθεντικοποίηση, η οποία αξιοποιεί χαρακτηριστικά της φυσικής συμπεριφοράς του χρήστη, προσφέρει μια πιο φυσική και ασφαλή λύση. Ειδικότερα, η ανάλυση γραφής, που ενσωματώνει στοιχεία του προσωπικού στυλ του χρήστη, επιτρέπει την αναγνώριση ταυτότητας με τρόπο διακριτικό και ανεξάρτητο. Καθώς κάθε άτομο έχει τον δικό του τρόπο διατύπωσης και χρήσης της γλώσσας, οι αποκλίσεις στη γραφή μπορούν να ανιχνευθούν μέσω ενός προσαρμοσμένου μοντέλου που εκπαιδεύεται και αναγνωρίζει τον αυθεντικό χρήστη.

Η χρήση τεχνικών Επεξεργασίας Φυσικής Γλώσσας επιτρέπει την εξαγωγή χαρακτηριστικών που μπορούν να χρησιμοποιηθούν ως μοναδικά "ψηφιακά αποτυπώματα" κάθε χρήστη, βασιζόμενα σε δείκτες όπως το μέσο μήκος λέξεων σε χαρακτήρες, η συχνότητα συγκεκριμένων συντακτικών δομών ή μερών του λόγου και η ποικιλία του λεξιλογίου. Οι τεχνικές NLP μετατρέπουν τον γραπτό λόγο σε ένα σύνολο ποσοτικών μετρήσεων που αναλύονται με τεχνικές μηχανικής μάθησης. Με αυτόν τον τρόπο, η ταυτοποίηση πραγματοποιείται με βάση διακριτά χαρακτηριστικά του γραπτού λόγου, επιτρέποντας την αδιάλειπτη επαλήθευση ταυτότητας, χωρίς να απαιτείται η άμεση παρέμβαση του χρήστη.

Η έμμεση αυθεντικοποίηση δεν προσφέρει μόνο ένα νέο επίπεδο ασφάλειας, αλλά και σημαντικά πλεονεκτήματα όσον αφορά τη χρηστικότητα. Αντί να διακόπτει την εμπειρία του χρήστη, ενσωματώνεται αδιάλειπτα στη διαδικασία της αλληλεπίδρασης με το σύστημα. Το γεγονός αυτό την καθιστά ιδανική για περιβάλλοντα όπου η συνεχής πρόσβαση στα δεδομένα είναι απαραίτητη και η διακοπή της ροής ενεργειών για λόγους ασφάλειας μπορεί να είναι επιζήμια ή ενοχλητική για τον χρήστη. Επιπλέον, η δυνατότητα της έμμεσης αυθεντικοποίησης να ανιχνεύει απειλές χωρίς να επιβαρύνει τον χρήστη προσφέρει μια πιο ολιστική προσέγγιση στην προστασία της ταυτότητας και των δεδομένων του.

Η ανάπτυξη και η εξέλιξη του τομέα της συνεχούς και έμμεσης αυθεντικοποίησης έχουν ιδιαίτερη σημασία, καθώς προσφέρουν τη δυνατότητα για πιο ανθεκτικά και προσαρμοστικά συστήματα ασφαλείας. Στο πλαίσιο αυτής της εργασίας, η έμμεση και συνεχής αυθεντικοποίηση υλοποιείται μέσω τεχνικών NLP και μηχανικής μάθησης, που επιτρέπουν την εκμάθηση και ανίχνευση μοναδικών χαρακτηριστικών του γραπτού λόγου του χρήστη. Χρησιμοποιώντας ένα μοντέλο One-Class Support Vector Machine, το σύστημα αναγνωρίζει πρότυπα γραφής του χρήστη και ανιχνεύει αποκλίσεις που μπορεί να υποδηλώνουν παραβίαση στο σύστημα. Το προτεινόμενο σύστημα καταδεικνύει τη δυνατότητα των σύγχρονων τεχνολογιών να προσφέρουν λύσεις που ανταποκρίνονται στις αυξανόμενες ανάγκες για ασφάλεια, παρέχοντας ταυτόχρονα ένα εύχρηστο και ελκυστικό προς τον χρήστη περιβάλλον.
