\newpage
\section{Διάρθρωση της Αναφοράς}
\label{section:layout}

Η διάρθρωση της παρούσας διπλωματικής εργασίας είναι η εξής:

\begin{itemize}
  \item{\textbf{Κεφάλαιο \ref{chapter:sota}:}
    Γίνεται ανασκόπηση της ερευνητικής περιοχής με έμφαση στις τεχνικές αυθεντικοποίησης και αναγνώρισης χρηστών μέσω NLP και μηχανικής μάθησης.
    }
  \item{\textbf{Κεφάλαιο \ref{chapter:theory}:}
    Παρουσιάζεται το θεωρητικό υπόβαθρο της εργασίας, με ανάλυση των βασικών εννοιών της αυθεντικοποίησης, της επεξεργασίας φυσικής γλώσσας και της μηχανικής μάθησης, καθώς και του μοντέλου One-Class SVM.
    }
  \item{\textbf{Κεφάλαιο \ref{chapter:implementations}:}
    Παρουσιάζεται η υλοποίηση του συστήματος, συμπεριλαμβανομένων των αλγορίθμων εξαγωγής χαρακτηριστικών, της επεξεργασίας δεδομένων, της εκπαίδευσης των μοντέλων και της αξιολόγησης του συστήματος.
    }
  \item{\textbf{Κεφάλαιο \ref{chapter:experiments}:}
    Παρουσιάζεται η μεθοδολογία των πειραμάτων και τα αποτελέσματα αξιολόγησης του συστήματος, περιλαμβάνοντας τους δείκτες FAR και FRR.
    }
  \item{\textbf{Κεφάλαιο \ref{chapter:conclusions}:}
    Παρουσιάζονται τα τελικά συμπεράσματα και τα προβλήματα που προέκυψαν.
    }
  \item{\textbf{Κεφάλαιο \ref{chapter:future_work}:}
    Προτείνονται θέματα για μελλοντική
    μελέτη, αλλαγές και επεκτάσεις.
    }
\end{itemize}

