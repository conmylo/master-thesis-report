\chapter{Μεθοδολογία και Υλοποίηση}
\label{chapter:implementations}

Η μεθοδολογία και η υλοποίηση αποτελούν τον πυρήνα της παρούσας εργασίας, καθώς περιγράφουν την προσέγγιση που ακολουθήθηκε για την ανάπτυξη του συστήματος αυθεντικοποίησης χρηστών. Η δομή του κεφαλαίου αναπτύσσεται με τρόπο ώστε να παρέχει μια καθολική και συστηματική παρουσίαση των διαδικασιών που ακολουθήθηκαν, από τη συλλογή των δεδομένων έως την αξιολόγηση των αποτελεσμάτων.

\subsection{Αρχιτεκτονική Υψηλού Επιπέδου}
Η συνολική αρχιτεκτονική του συστήματος συνιστά τον συνδυασμό όλων των επιμέρους ενοτήτων που περιγράφηκαν στα προηγούμενα τμήματα, σχηματίζοντας ένα ολοκληρωμένο σύστημα αυθεντικοποίησης. Η αρχιτεκτονική έχει σχεδιαστεί με τρόπο που να διασφαλίζει την ακριβή, γρήγορη και αξιόπιστη αναγνώριση χρηστών μέσω ανάλυσης γραφής. Επιπλέον, έχει ενσωματωθεί ένα απλό αλλά λειτουργικό περιβάλλον διεπαφής χρήστη (UI) με χρήση του Streamlit, επιτρέποντας την εύκολη πρόσβαση στο σύστημα και την παρακολούθηση των αποτελεσμάτων σε πραγματικό χρόνο.

Πιο συγκεκριμένα, το κεφάλαιο ξεκινά με την ανάλυση της συλλογής και της προεπεξεργασίας των δεδομένων, αναδεικνύοντας τη σημασία της ποιότητας των δεδομένων στη συνολική απόδοση του συστήματος. Στη συνέχεια, εστιάζει στην εξαγωγή χαρακτηριστικών, μια κρίσιμη διαδικασία που συνδέει τη θεωρητική βάση της επεξεργασίας φυσικής γλώσσας με την πρακτική της εφαρμογή. Η εκπαίδευση των μοντέλων μηχανικής μάθησης και η ενσωμάτωση ενός συστήματος εμπιστοσύνης σε συνδυασμό με το σύστημα απόφασης περιγράφονται με λεπτομέρεια, παρέχοντας πληροφορίες για τις τεχνικές και τις παραμέτρους που χρησιμοποιήθηκαν. Τέλος, παρουσιάζεται το σύστημα αξιολόγησης και ένα γραφικό περιβάλλον χρήστη.

Η αρχιτεκτονική υψηλού επιπέδου του συστήματος εκπαίδευσης, απόφασης και αξιολόγησης φαίνεται στο~\autoref{fig:system_architecture}.

\begin{figure}[H]
    \centering
    \includegraphics[width=\textwidth]{images/chapter4/flowchartGrande.png}
    \caption{Αρχιτεκτονική Υψηλού Επιπέδου του Συστήματος}
    \label{fig:system_architecture}
\end{figure}

\subsection{Εργαλεία και Πλατφόρμες Ανάπτυξης}
Για την ανάπτυξη του συστήματος χρησιμοποιήθηκαν τα εξής:
\begin{itemize}
    \item \textbf{Γλώσσες Προγραμματισμού}: Python (βιβλιοθήκες sklearn, NLTK, textstat, Streamlit).
    \item \textbf{Περιβάλλον Ανάπτυξης}: PyCharm και Visual Studio Code για την ανάπτυξη και δοκιμή του κώδικα. Github για διαδικασίες Continuous Integration - Continuous Deployment.
\end{itemize}


\section{Συλλογή και Προεπεξεργασία Δεδομένων}
\label{sec:implementations_data}

Η ποιότητα και η επεξεργασία των δεδομένων αποτελούν κρίσιμο στάδιο για την επιτυχία ενός συστήματος συνεχούς και έμμεσης αυθεντικοποίησης. Στο παρόν σύστημα, η διαδικασία συλλογής και προεπεξεργασίας δεδομένων έχει σχεδιαστεί ώστε να διασφαλίζει την ακεραιότητα και την αξιοπιστία των χαρακτηριστικών που χρησιμοποιούνται για την ταυτοποίηση των χρηστών.

\subsection{Πηγή Δεδομένων}

Τα δεδομένα που χρησιμοποιήθηκαν αντλήθηκαν από δημόσια διαθέσιμες πηγές, και συγκεκριμένα από το \textit{Kaggle\footnote{\url{https://www.kaggle.com}}
}, το οποίο προσφέρει ένα ευρύ φάσμα από σετ δεδομένων (datasets). Το συγκεκριμένο dataset περιλαμβάνει πολλαπλές εγγραφές από διάφορους χρήστες, παρέχοντας ένα πλούσιο σύνολο δεδομένων για την εξαγωγή χαρακτηριστικών και την εκπαίδευση. 

Επιπλέον, τα δεδομένα που περιλαμβάνται στο συγκεκριμένο dataset αντλούν τη θεματολογία τους από κάθε πτυχή της ανθρώπινης δραστηριότητας, διασφαλίζοντας την ποικιλομορφία στα στυλ γραφής. Η κατηγοριοποίηση βάσει χρηστών επιτρέπει τη δημιουργία εξατομικευμένων μοντέλων και εξασφαλίζει την απαραίτητη ανομοιογένεια τόσο στην εκπαίδευση των μοντέλων, όσο και στον έλεγχο και στην αξιολόγηση του συστήματος.

\subsection{Διαδικασία Προεπεξεργασίας Δεδομένων}

Η διαδικασία προεπεξεργασίας περιλάμβανε τα εξής βήματα:

\begin{enumerate}
    \item \textbf{Καθαρισμός dataset:} Αφαιρέθηκαν στήλες που παρείχαν περιττή πληροφορία σχετικά με την εργασία. Με αυτόν τον τρόπο μειώθηκε και το μέγεθος του dataset εξασφαλίζονταν ταχύτερους χρόνους ανάγνωσης και φόρτωσης των δεδομένων.
    \item \textbf{Καθαρισμός κειμένων:} Αφαιρέθηκαν από τα prompts τα URLs και HTML tags. Ο καθαρισμός διασφάλισε ότι τα δεδομένα περιείχαν μόνο πληροφοριακό περιεχόμενο.
\end{enumerate}

Η διαδικασία προεπεξεργασίας εφαρμόστηκε με χρήση βιβλιοθηκών Python, όπως οι \texttt{NLTK}, \texttt{spaCy}, και \texttt{textstat}, οι οποίες διευκόλυναν την εξαγωγή και την ανάλυση χαρακτηριστικών.

\subsection{Χαρακτηριστικά του Σετ Δεδομένων}

Το dataset\footnote{\url{https://www.kaggle.com/datasets/mmmarchetti/tweets-dataset}}
 περιλαμβάνει κείμενα από 14 διαφορετικούς χρήστες, με μέσο όρο 2.000 prompts ανά χρήστη. Ο πίνακας \ref{table:dataset_stats} συνοψίζει βασικά στατιστικά στοιχεία του dataset:

\begin{table}[h!]
\centering
\begin{tabular}{|l|r|}
\hline
\textbf{Χαρακτηριστικό} & \textbf{Τιμή} \\ \hline
Αριθμός Χρηστών & 14 \\ \hline
Συνολικά prompts ανά χρήστη & 2.000 \\ \hline
Μέσος Όρος Λέξεων ανά Κείμενο & Χ \\ \hline
Μέγιστο Μήκος Κειμένου & Χ \\ \hline
\end{tabular}
\caption{Στατιστικά Χαρακτηριστικά του Dataset}
\label{table:dataset_stats}
\end{table}

Επιπλέον, το dataset περιλαμβάνει τις παρακάτω στήλες, οι οποίες περιγράφονται στον πίνακα \ref{table:dataset_columns} και φαίνονται στο ~\autoref{fig:chapter4_datasetFormat}:

\begin{table}[h!]
\centering
\begin{tabular}{|l|l|}
\hline
\textbf{Στήλη} & \textbf{Περιγραφή} \\ \hline
\texttt{author} & Ταυτοποιητικό πεδίο του χρήστη που δημιούργησε το κείμενο. \\ \hline
\texttt{content} & Το κύριο σώμα του κειμένου, περιέχει το prompt προς ανάλυση. \\ \hline
\texttt{date\_time} & Ημερομηνία και ώρα δημιουργίας του κειμένου. \\ \hline
\texttt{id} & Μοναδικό αναγνωριστικό για κάθε prompt στο dataset. \\ \hline
\end{tabular}
\caption{Περιγραφή Στηλών του Dataset}
\label{table:dataset_columns}
\end{table}

\begin{figure}[H]
    \centering
    \includegraphics[width=0.7\textwidth]{images/chapter4/datasetFormat.png}
    \caption{Στιγμιότυπο οθόνης από το σετ δεδομένων που χρησιμοποιήθηκε - φαίνονται οι 4 στήλες που αναφέρονται παραπάνω καθώς και πολλαπλές καταχωρίσεις}
    \label{fig:chapter4_datasetFormat}
\end{figure}

Η ποικιλομορφία στα δεδομένα επιτρέπει την εκπαίδευση μοντέλων ικανά να αναγνωρίζουν διαφορετικά στυλ γραφής. Η διασφάλιση της ποιότητας και της ομοιογένειας του dataset αποτέλεσε θεμελιώδη παράγοντα για την επιτυχία των επόμενων σταδίων.

Τα δεδομένα που προκύπτουν μετά την προεπεξεργασία αποθηκεύονται στη δομή \texttt{data/filtered\_cleaned.csv} στο αποθετήριο. Η διαχείριση των δεδομένων γίνεται μέσω βιβλιοθηκών Python όπως \texttt{pandas\footnote{\url{https://pandas.pydata.org/}}
} για την ανάγνωση και εγγραφή αρχείων CSV και \texttt{os\footnote{\url{https://docs.python.org/3/library/os.html}}} για την οργάνωση των φακέλων.


\section{Εξαγωγή Χαρακτηριστικών}
\label{sec:implementation_features}

Η εξαγωγή χαρακτηριστικών αποτελεί ένα από τα πιο σημαντικά βήματα στη διαδικασία ανάπτυξης ενός συστήματος αυθεντικοποίησης. Τα χαρακτηριστικά που εξάγονται από τα δεδομένα γραφής περιγράφουν μοναδικές πτυχές του στυλ γραφής κάθε χρήστη, επιτρέποντας έτσι την αναγνώρισή τους. Στην παρούσα εργασία, η εξαγωγή χαρακτηριστικών βασίζεται τόσο σε γλωσσικά όσο και σε συμπεριφορικά χαρακτηριστικά, διασφαλίζοντας ότι λαμβάνονται υπόψη η δομή, το περιεχόμενο του κειμένου αλλά και οι ιδιαιτερότητες των συγγραφικών μοτίβων.

Η διαδικασία εξαγωγής περιλαμβάνει τη χρήση βιβλιοθηκών Python, όπως οι \texttt{NLTK\footnote{\url{https://www.nltk.org/}}}, \texttt{textstat\footnote{\url{https://textstat.org/}}}, και \texttt{numpy\footnote{\url{https://numpy.org/}}}, για την ανάλυση και ποσοτικοποίηση γλωσσικών και δομικών χαρακτηριστικών. Στη συνέχεια, τα χαρακτηριστικά αυτά ομαδοποιούνται σε 4 θεματικές κατηγορίες, χαρακτηριστικά που βασίζονται σε χαρακτήρες, λεξιλογικα και συντακτικά χαρακτηριστικά, δείκτες πολυπλοκότητας και αναγνωσιμότητας και δομικοί και στυλιστικοί δείκτες, όπως φαίνεται στο ~\autoref{fig:chapter4_featuresByCategory}, διευκολύνοντας την κατανόηση της συμβολής τους στη διαδικασία αυθεντικοποίησης.

\begin{figure}[H]
    \centering
    \includegraphics[width=\textwidth]{images/chapter4/featuresByCategory.png}
    \caption{Τα χαρακτηριστικά που εξάγονται στη παρούσα εργασία}
    \label{fig:chapter4_featuresByCategory}
\end{figure}

\subsection{Χαρακτηριστικά που Βασίζονται στους Χαρακτήρες}

\subsubsection{Char Count (Normalized)}

Μετρά τον συνολικό αριθμό χαρακτήρων ενός κειμένου, κανονικοποιημένο με βάση το μήκος των 100 χαρακτήρων:
\[
\text{Char Count Norm} = \frac{\text{Total Characters}}{100}
\]
Αυτή η μέτρηση εξασφαλίζει συγκρισιμότητα μεταξύ κειμένων διαφορετικών μεγεθών.

\subsubsection{Character N-grams (3-grams) Ratio}

 Υπολογίζει την αναλογία των τριγραμμάτων (3-grams) στους συνολικούς χαρακτήρες:
\[
\text{Char 3-grams Ratio} = \frac{\text{Number of 3-grams}}{\text{Total Characters}}
\]
Τα τριγράμματα είναι ακολουθίες τριών διαδοχικών χαρακτήρων (π.χ., "the", "igh"), και αποτυπώνουν μοτίβα γραφής του χρήστη.

\subsection{Λεξιλογικά και Συντακτικά Χαρακτηριστικά}

\subsubsection{Stop Word Frequency Ratio}

Υπολογίζει τον αριθμό των \textit{stop words} που περιέχονται σε ένα κείμενο ως ποσοστό του συνολικού αριθμού λέξεων. Η εξίσωση που χρησιμοποιείται είναι η εξής:
\[
\text{Stop Word Ratio} = \frac{\text{Stop Words Count}}{\text{Total Words Count}}
\]
Τα \textit{stop words} είναι λέξεις όπως: \textit{"the", "is", "in", "it", "of", "and", "to", "a", "that", "with", "as", "for", "on", "at", "by"}. Αυτές οι λέξεις είναι συχνές στη γλώσσα αλλά δεν παρέχουν σημαντικές πληροφορίες για το περιεχόμενο του κειμένου.

Παρά την έλλειψη νοηματικού βάρους, η κατανομή των \textit{stop words} μπορεί να διαφέρει σημαντικά μεταξύ των χρηστών, καθώς επηρεάζεται από τον τρόπο γραφής τους. Για παράδειγμα, ορισμένοι χρήστες μπορεί να έχουν την τάση να χρησιμοποιούν \textit{stop words} πιο συχνά για να συνδέουν φράσεις ή να δημιουργούν ροή στο κείμενο τους, γεγονός που αποτελεί χαρακτηριστικό προσωπικού ύφους.

\subsubsection{Type-Token Ratio (Lexical Diversity)}

Μετρά τη λεξιλογική ποικιλία του κειμένου:
\[
\text{Type-Token Ratio} = \frac{\text{Unique Words Count}}{\text{Total Words Count}}
\]
Μεγαλύτερη τιμή υποδηλώνει πιο ποικιλόμορφο λεξιλόγιο.

\subsubsection{Part-of-Speech Ratios}

Τα ποσοστά των μερών του λόγου (όπως ουσιαστικά, ρήματα, επίθετα, επιρρήματα) υπολογίζονται ως εξής:
\[
\text{POS Ratio} = \frac{\text{POS Count}}{\text{Total Words Count}}
\]
Για παράδειγμα, το \textit{Adjective Ratio} αφορά τη συχνότητα των επιθέτων (\textit{JJ}), ενώ το \textit{Verb Ratio} περιλαμβάνει όλες τις μορφές ρημάτων (\textit{VB, VBD, VBG, VBN, VBP, VBZ}).

\subsection{Δείκτες Πολυπλοκότητας και Αναγνωσιμότητας}

\subsubsection{Readability Score (Flesch Reading Ease)}

Ο δείκτης αναγνωσιμότητας \textit{Flesch Reading Ease~\cite{flesch1949art}} υπολογίζεται με βάση την ακόλουθη εξίσωση:
\[
\text{Flesch Score} = 206.835 - 1.015 \left( \frac{\text{Total Words}}{\text{Total Sentences}} \right) - 84.6 \left( \frac{\text{Total Syllables}}{\text{Total Words}} \right)
\]
Μεγαλύτερες τιμές υποδηλώνουν πιο εύκολα αναγνώσιμα κείμενα.

\subsubsection{Syllable Average}

Ο μέσος όρος συλλαβών ανά λέξη υπολογίζεται ως:
\[
\text{Syllable Avg} = \frac{\text{Total Syllables Count}}{\text{Total Words Count}}
\]

\subsubsection{Polysyllabic Word Ratio}

Μετρά την αναλογία των πολυσύλλαβων λέξεων:
\[
\text{Polysyllabic Ratio} = \frac{\text{Polysyllabic Words Count}}{\text{Total Words Count}}
\]
όπου πολυσύλλαβες είναι οι λέξεις με περισσότερες από τρεις συλλαβές.

\subsection{Δομικοί και Στυλιστικοί Δείκτες}

\subsubsection{Average Sentence Length}

Το μέσο μήκος προτάσεων υπολογίζεται με βάση τον αριθμό λέξεων:
\[
\text{Avg Sentence Length} = \frac{\text{Total Words}}{\text{Total Sentences}}
\]

\subsubsection{Pronoun Usage Proportion}

Υπολογίζει την αναλογία των αντωνυμιών στο κείμενο:
\[
\text{Pronoun Proportion} = \frac{\text{Pronoun Count}}{\text{Total Words Count}}
\]
Οι αντωνυμίες που περιλαμβάνονται είναι: \textit{"I", "you", "he", "she", "it", "we", "they", "me", "us", "him", "her", "them"}.

\subsubsection{Formality Measure}

Ο δείκτης επισημότητας μετράται ως εξής:
\[
\text{Formality} = \frac{\text{Noun Count} + \text{Adjective Count}}{\text{Pronoun Count} + \text{Verb Count} + 0.01}
\]
Μεγαλύτερη τιμή υποδηλώνει πιο επίσημο ύφος γραφής.

\subsection{Σύνοψη χαρακτηριστικών}
Ο πίνακας~\autoref{tab:features} περιλαμβάνει συνοπτικά όλα τα χαρακτηριστικά που εξάγονται στην παρούσα εργασία:

\begin{table}[H]
\centering
\begin{tabular}{|l|l|}
\hline
\textbf{Χαρακτηριστικό} & \textbf{Κατηγορία} \\
\hline
Char Count (Normalized) & Character-based Features \\
Character N-grams (3-grams) Ratio & Character-based Features \\
Uppercase Proportion & Character-based Features \\
Digit Proportion & Character-based Features \\
Punctuation Proportion & Character-based Features \\
Type-Token Ratio & Lexical and Syntactical Features \\
Average Word Length & Lexical and Syntactical Features \\
Part-of-Speech Ratios & Lexical and Syntactical Features \\
Pronoun Usage Proportion & Lexical and Syntactical Features \\
Past Tense Ratio & Lexical and Syntactical Features \\
Readability Score & Complexity and Readability Indicators \\
Syllable Avg & Complexity and Readability Indicators \\
Polysyllabic Word Ratio & Complexity and Readability Indicators \\
Stop Word Density & Structure and Style Indicators \\
Avg Sentence Length & Structure and Style Indicators \\
Formality Measure & Structure and Style Indicators \\
\hline
\end{tabular}
\caption{Κατηγοριοποίηση Χαρακτηριστικών Εξαγωγής}
\label{tab:features}
\end{table}

\subsection{Εργαλεία και Μέθοδοι Εξαγωγής Χαρακτηριστικών}

Η συνάρτηση \texttt{extract\_features} είναι η κύρια μέθοδος εξαγωγής χαρακτηριστικών στο σύστημα αυθεντικοποίησης χρηστών. Η συνάρτηση λαμβάνει ως είσοδο ένα κείμενο (\texttt{text}), το οποίο μπορεί να περιλαμβάνει προτάσεις, παραγράφους ή μεγαλύτερα τμήματα γραπτού λόγου. Επεξεργάζεται το κείμενο και εξάγει ένα σύνολο χαρακτηριστικών, τα οποία επιστρέφονται ως ένα πολυδιάστατο διάνυσμα (\texttt{list of feature values}).

Η έξοδος της συνάρτησης περιλαμβάνει τιμές για όλα τα χαρακτηριστικά που περιγράφηκαν στις προηγούμενες υποενότητες, όπως οι αναλογίες χαρακτήρων, λέξεων, στατιστικά στυλ, αναγνωσιμότητα και άλλοι δείκτες. Το διάνυσμα αυτό αποτελεί την είσοδο για τα επόμενα στάδια ανάλυσης, εκπαίδευσης αλλά και αξιολόγησης, επιτρέποντας την αποτελεσματική αυθεντικοποίηση χρηστών.

\section{Εκπαίδευση μοντέλων}
\label{sec:implementation_train}

Η εκπαίδευση των μοντέλων αποτελεί έναν κρίσιμο πυλώνα του συστήματος αυθεντικοποίησης. Στη συγκεκριμένη εργασία, βασιζόμαστε στον αλγόριθμο \textit{One-Class SVM}, όπως έχει περιγραφεί στην~\autoref{subsec:svm_ocsvm}, για να εκπαιδεύσουμε τα μοντέλα που αναγνωρίζουν τα πρότυπα γραφής ενός χρήστη και απορρίπτουν τυχόν απόπειρες παραβίασης. Σε αυτή την ενότητα περιγράφεται αναλυτικά η διαδικασία εκπαίδευσης, η επιλογή των υπερπαραμέτρων και οι στρατηγικές βελτιστοποίησης.

\subsection{Προεπεξεργασία Δεδομένων Εκπαίδευσης}

Η προεπεξεργασία των δεδομένων περιλαμβάνει τα παρακάτω στάδια:
\begin{enumerate}
    \item \textbf{Ανάγνωση δεδομένων}: Το αρχείο δεδομένων περιέχει τις στήλες \texttt{author} και \texttt{content}. Τα δεδομένα διαχωρίζονται με βάση τον συγγραφέα.
    \item \textbf{Διαχωρισμός σε σύνολα}:
    \begin{itemize}
        \item Σύνολο Εκπαίδευσης: 85\% των δεδομένων.
        \item Σύνολο Δοκιμών: 15\% των δεδομένων, αποθηκεύεται για αξιολόγηση.
    \end{itemize}
    \item \textbf{Κανονικοποίηση δεδομένων}: Τα εξαγόμενα χαρακτηριστικά κανονικοποιούνται χρησιμοποιώντας τον \texttt{StandardScaler}, ώστε να έχουν μέσο όρο 0 και τυπική απόκλιση 1. Αυτή η διαδικασία είναι απαραίτητη, διότι διαφορετικά χαρακτηριστικά μπορεί να έχουν διαφορετικές κλίμακες (π.χ., ο μέσος όρος συλλαβών ανά λέξη κυμαίνεται από 0 έως 1, ενώ ο αριθμός χαρακτήρων μπορεί να είναι εκατοντάδες). Εάν τα δεδομένα δεν κανονικοποιηθούν, τα χαρακτηριστικά με μεγαλύτερες αριθμητικές τιμές ενδέχεται να έχουν μεγαλύτερη επίδραση στην εκπαίδευση και απόφαση του μοντέλου, καθιστώντας το μη αντικειμενικό. Η κανονικοποίηση εξασφαλίζει ότι όλα τα χαρακτηριστικά έχουν την ίδια αριθμητική βαρύτητα και συμβάλλει στη σωστή λειτουργία των αλγορίθμων μηχανικής μάθησης, όπως το \texttt{One-Class SVM}, οι οποίοι βασίζονται στη μέτρηση αποστάσεων.
\end{enumerate}

\subsection{Εξαγωγή και Κανονικοποίηση Χαρακτηριστικών}

Τα χαρακτηριστικά που εξάγονται από τα δεδομένα εκπαίδευσης περιγράφονται αναλυτικά στο ~\autoref{sec:implementation_features}. Μετά την εξαγωγή, εφαρμόζεται κανονικοποίηση για να διασφαλιστεί η συγκρισιμότητα μεταξύ διαφορετικών χαρακτηριστικών.

\subsection{Εκπαίδευση Μοντέλων \textit{One-Class SVM}}

Η εκπαίδευση πραγματοποιείται με τη χρήση του αλγορίθμου \textit{One-Class SVM} με πυρήνα \texttt{rbf}. Ο αλγόριθμος αυτός είναι κατάλληλος για εφαρμογές όπου υπάρχουν δεδομένα μόνο από μία κατηγορία (εν προκειμένω, του χρήστη) - one-vs-all classification. Το μοντέλο μαθαίνει τα πρότυπα της κατηγορίας αυτής και απορρίπτει ανωμαλίες.

\subsubsection{Υπερπαράμετροι του \textit{One-Class SVM}}

Οι υπερπαράμετροι του \textit{One-Class SVM} που ρυθμίστηκαν είναι:
\begin{itemize}
    \item \textbf{ν (nu)}: Ελέγχει το ποσοστό των \textit{outliers} που το μοντέλο θα ανεχθεί.
    \begin{itemize}
        \item Τιμές που δοκιμάστηκαν: $0.001$, $0.005$, $0.01$.
    \end{itemize}
    \item \textbf{gamma}: Ελέγχει την ακτίνα επιρροής ενός δείγματος.
    \begin{itemize}
        \item Τιμές που δοκιμάστηκαν: $0.05$, $0.07$, $0.1$, $0.15$, $0.2$, $0.5$.
    \end{itemize}
\end{itemize}

Για κάθε χρήστη, εκπαιδεύτηκαν 18 μοντέλα για όλους τους συνδυασμούς των παραπάνω τιμών. Τα μοντέλα αποθηκεύτηκαν στη μνήμη του συστήματος μαζί με τα μοντέλα για κανονικοποίηση νέων δεδομένων.

\subsection{Hyperparameter Tuning και Threshold Tuning}

\subsubsection{Tuning των Υπερπαραμέτρων}

Οι υπερπαράμετροι ρυθμίστηκαν μέσω πειραματισμών:
\begin{itemize}
    \item \textbf{ν (nu)}: 
    \begin{itemize}
        \item Χαμηλές τιμές αυξάνουν την ευαισθησία του μοντέλου.
        \item Υψηλές τιμές επιτρέπουν μεγαλύτερη ανοχή σε \textit{outliers}.
    \end{itemize}
    \item \textbf{gamma}:
    \begin{itemize}
        \item Χαμηλές τιμές ορίζουν ευρύτερες περιοχές απόφασης.
        \item Υψηλές τιμές επικεντρώνονται σε τοπικά μοτίβα.
    \end{itemize}
\end{itemize}

\subsubsection{Threshold Tuning}

Το threshold (\emph{κατώφλι}) είναι μια κρίσιμη παράμετρος στη διαδικασία απόφασης του μοντέλου, καθώς καθορίζει πώς θα ερμηνευτούν οι προβλέψεις για νέες εισόδους. Συγκεκριμένα, το threshold θέτει το σημείο στο οποίο το μοντέλο αποφασίζει αν μια νέα είσοδος ανήκει στην κατηγορία του γνήσιου χρήστη ή ερμηνεύεται ως απόρριψη.

\paragraph{Λειτουργία του threshold}
\begin{itemize}
    \item \textbf{Positive Decision}: Αν η απόσταση ή η πρόβλεψη ενός δείγματος είναι μεγαλύτερη από το threshold, το μοντέλο το αναγνωρίζει ως γνήσιο δείγμα.
    \item \textbf{Negative Decision}: Αν είναι μικρότερη, το μοντέλο απορρίπτει το δείγμα, θεωρώντας το ως μη γνήσιο.
\end{itemize}

\paragraph{Προσαρμογή Threshold}
Η ρύθμιση του threshold γίνεται με πειραματισμό σε διάφορες τιμές μέσα σε ένα εύρος, στη συγκεκριμένη περίπτωση μεταξύ \([-0.1, 0.8]\).

\begin{itemize}
    \item \textbf{Χαμηλό Threshold}: Εάν το threshold είναι κοντά στο \(-0.1\), το μοντέλο είναι πιο δεκτικό και αποδέχεται περισσότερα δείγματα ως γνήσια, αυξάνοντας την πιθανότητα λανθασμένων θετικών προβλέψεων.
    \item \textbf{Υψηλό Threshold}: Εάν το threshold είναι κοντά στο \(0.8\), το μοντέλο γίνεται πιο αυστηρό, απορρίπτοντας περισσότερα δείγματα ως εισβολείς.Έτσι μειώνεται η πιθανότητα λανθασμένης αποδοχής δειγμάτων ως γνήσια, αλλά αυξάνεται η πιθανότητα λανθασμένης απόρριψης γνήσιων δειγμάτων.
\end{itemize}

Η βέλτιστη ρύθμιση του threshold επιτυγχάνει έναν αποδεκτό συμβιβασμό μεταξύ ασφάλειας και χρηστικότητας, καθιστώντας το σύστημα πιο αποτελεσματικό σε πραγματικές συνθήκες.


\section{Σύστημα απόφασης}
\label{sec:implementation_decision}

Το σύστημα απόφασης αποτελεί κρίσιμο μέρος της συνολικής αρχιτεκτονικής του συστήματος αυθεντικοποίησης. Αξιοποιεί τα χαρακτηριστικά που εξάγονται από τα δεδομένα, καθώς και τα αποτελέσματα των μοντέλων που εκπαιδεύτηκαν, για να παρέχει τεκμηριωμένες και αξιόπιστες αποφάσεις σχετικά με τη γνησιότητα του χρήστη.

\subsection{Εισαγωγή}
Η διαδικασία απόφασης συνδυάζει τα αποτελέσματα των εκπαιδευμένων μοντέλων με τη χρήση της συνάρτησης \emph{επιπέδου βεβαιότητας} και του αλγορίθμου \emph{σταθμισμένης πλειοψηφίας ψήφων}. Το τελικό αποτέλεσμα εξαρτάται από:
\begin{itemize}
    \item Την απόσταση από το hyperplane κάθε μοντέλου.
    \item Το επίπεδο βεβαιότητας που αντιστοιχεί σε κάθε απόσταση.
    \item Το κατώφλι απόφασης (\emph{decision threshold}) που καθορίζει τη συμπεριφορά του συστήματος και την κατηγοριοποίηση της απόφασης.
\end{itemize}

\subsection{Συνάρτηση Επιπέδου Βεβαιότητας}
Η συνάρτηση επιπέδου βεβαιότητας (\texttt{certainty\_level}) υπολογίζει το πόσο σίγουρο είναι ένα μοντέλο για την απόφασή του. Ορίζεται ως εξής:
\begin{equation}
    c(x) =
    \begin{cases} 
    \frac{|d(x)|}{d_{\text{max}}}, & \text{αν } |d(x)| < d_{\text{max}} \\
    1, & \text{αν } d(x) > d_{\text{max}} \text{ και } y = 1 \\
    -1, & \text{αν } d(x) > d_{\text{max}} \text{ και } y = -1
    \label{eq:certaintyLevel}
    \end{cases}
\end{equation}
όπου:
\begin{itemize}
    \item $d(x)$ είναι η απόσταση του σημείου εισόδου $x$ από το hyperplane του μοντέλου.
    \item $d_{\text{max}}$ είναι το μέγιστο όριο απόστασης που καθορίζει το επίπεδο βεβαιότητας.
    \item $y$ είναι η προβλεπόμενη ετικέτα (1 για γνήσιος χρήστης, -1 για εισβολέας).
\end{itemize}

\subsection{Υποσύστημα Ψηφοφορίας}
Σε αυτή την ενότητα παρουσιάζονται δύο διαφορετικές προσεγγίσεις για τη λήψη απόφασης, η απλή πλειοψηφική συνάρητηση και η σταθμισμένη πλειοψηφική συνάρτηση. Οι δύο αυτές μέθοδοι διαφέρουν ως προς τη πολυπλοκότητα και τη φιλοσοφία της διαδικασίας λήψης αποφάσεων. Η σύγκρισή τους πραγματοποιείται σε επόμενο κεφάλαιο.

\subsubsection{Απλή Πλειοψηφική Συνάρτηση}
Η πρώτη προσέγγιση συνδυάζει τα αποτελέσματα πολλαπλών μοντέλων με την ακόλουθη διαδικασία: κάθε μοντέλο $i$ παράγει μια πρόβλεψη $y_i$: 1 για γνήσιο χρήστη και -1 για εισβολέα. Το τελικό αποτέλεσμα υπολογίζεται ως εξής:
\begin{equation}
    \text{Decision} =
    \begin{cases} 
    1, & \text{αν } \sum_{i=1}^{N} y_i > 0 \\
    -1, & \text{αν } \sum_{i=1}^{N} y_i \leq 0
    \end{cases}
\end{equation}
όπου:
\begin{itemize}
    \item $N$ είναι το πλήθος των μοντέλων.
    \item $y_i$ είναι η πρόβλεψη του μοντέλου $i$.
\end{itemize}

\subsubsection{Σταθμισμένη Πλειοψηφική Συνάρτηση}
Η δεύτερη προσέγγιση συνδυάζει τα αποτελέσματα πολλαπλών μοντέλων χρησιμοποιώντας τη σταθμισμένη πλειοψηφική συνάρτηση. Κάθε μοντέλο $i$ παράγει μια πρόβλεψη $y_i$ και μια βαρύτητα ψήφου $w_i$, που βασίζεται στο επίπεδο βεβαιότητας. Το τελικό αποτέλεσμα υπολογίζεται ως εξής:
\begin{equation}
    \text{Decision} =
    \begin{cases} 
    1, & \text{αν } \sum_{i=1}^{N} w_i y_i > 0 \\
    -1, & \text{αν } \sum_{i=1}^{N} w_i y_i \leq 0
    \end{cases}
\end{equation}
όπου:
\begin{itemize}
    \item $w_i = |c_i(x)|$, το απόλυτο επίπεδο βεβαιότητας του μοντέλου $i$.
    \item $N$ είναι το πλήθος των μοντέλων.
    \item $y_i$ είναι η πρόβλεψη του μοντέλου $i$.
\end{itemize}

\subsection{Παραδείγματα Εφαρμογής}

Για να κατανοηθεί καλύτερα η λειτουργία του συστήματος λήψης αποφάσεων, παρατίθενται παραδείγματα εφαρμογής. Στο πλαίσιο αυτό, οι \(c_1, c_2, c_3, \ldots, c_n\) αντιπροσωπεύουν τις αποφάσεις που λαμβάνονται από διαφορετικά μοντέλα για ένα δείγμα \(x\). Κάθε \(c_i(x)\) εκφράζει τη βαθμολογία εμπιστοσύνης (\textit{confidence score}) που προκύπτει από το αντίστοιχο μοντέλο, με θετικές τιμές να υποδηλώνουν γνήσιο χρήστη (\textit{genuine user}) και αρνητικές τιμές να υποδηλώνουν εισβολέα (\textit{impostor}).

\begin{itemize}
    \item \textbf{Περίπτωση Γνήσιου Χρήστη:} 
    \[
    c_1(x) = 0.8, \quad c_2(x) = 0.7, \quad c_3(x) = 0.6
    \]
    - \textbf{Απλή Πλειοψηφία:} Όλα τα μοντέλα αποφασίζουν ότι πρόκειται για γνήσιο χρήστη και αναθέτουν τη τιμή 1 στα \(c_i(x)\):
      \[
      \text{Απλή Απόφαση} = 1 + 1 + 1 = 3 > 0 \quad (\text{Γνήσιος Χρήστης})
      \]
    - \textbf{Σταθμισμένη Πλειοψηφία:} Όλα τα μοντέλα αποφασίζουν ότι πρόκειται για γνήσιο χρήστη και αναθέτουν τη τιμή 1 πολλαπλασιασμένη με το βάρος της βεβαιότητας κάθε μοντέλου:
      \[
      \text{Σταθμισμένη Απόφαση} = 0.8 * 1 + 0.7 * 1 + 0.6 * 1 = 2.1 > 0 \quad (\text{Γνήσιος Χρήστης})
      \]
    
    \item \textbf{Περίπτωση Εισβολέα:} 
    \[
    c_1(x) = -0.8, \quad c_2(x) = -0.7, \quad c_3(x) = -0.6
    \]
    - \textbf{Απλή Πλειοψηφία:} Όλα τα μοντέλα αποφασίζουν ότι πρόκειται για εισβολέα και αναθέτουν τη τιμή -1 στα \(c_i(x)\):
      \[
      \text{Απλή Απόφαση} = (-1) + (-1) + (-1) = -3 < 0 \quad (\text{Εισβολέας})
      \]
    - \textbf{Σταθμισμένη Πλειοψηφία:} Όλα τα μοντέλα αποφασίζουν ότι πρόκειται για εισβολέα και αναθέτουν τη τιμή -1 πολλαπλασιασμένη με το βάρος της βεβαιότητας κάθε μοντέλου:
      \[
      \text{Σταθμισμένη Απόφαση} = 0.8 * (-1) + 0.7 * (-1) + 0.6 * (-1) = -2.1 < 0 \quad (\text{Εισβολέας})
      \]
    
    \end{itemize}

Με αυτόν τον τρόπο, η σταθμισμένη πλειοψηφική ψήφος λαμβάνει υπόψη τη βαρύτητα κάθε \(c_i(x)\), ενώ η βασική ψήφος βασίζεται αποκλειστικά στο πρόσημο της απόφασης του κάθε μοντέλου.

\section{Σύστημα Κλειδώματος/Εμπιστοσύνης}
\label{sec:implementation_lock}

\subsection{Συνάρτηση Εμπιστοσύνης}
Η ασφάλεια ενός συστήματος αυθεντικοποίησης εξαρτάται όχι μόνο από την ακρίβεια των μοντέλων μηχανικής μάθησης, αλλά και από τη δυνατότητά του να διαχειρίζεται καταστάσεις όπου οι αποφάσεις μπορεί να είναι αβέβαιες ή να υπόκεινται σε διαδοχικά λάθη. Το σύστημα κλειδώματος/εμπιστοσύνης ενσωματώνει έναν μηχανισμό παρακολούθησης του επιπέδου εμπιστοσύνης ($C$) του συστήματος προς τον χρήστη. Αυτό το επίπεδο αυξάνεται ή μειώνεται ανάλογα με την απόδοση του χρήστη, και σε περίπτωση που το $C$ πέσει κάτω από ένα όριο ($\text{confidence\_threshold}$), το σύστημα ενεργοποιεί μηχανισμούς κλειδώματος για την προστασία από κακόβουλη χρήση.

\subsubsection{Συνάρτηση Εμπιστοσύνης}
Η συνάρτηση εμπιστοσύνης περιγράφεται μαθηματικά ως:

\begin{figure}[H]
    \centering
    \includegraphics[width=\textwidth]{images/chapter4/confidenceLevelFunc.png}
    \caption{Συνάρτηση Confidence Level}
    \label{fig:confidenceLevelFunction}
\end{figure}

% \[
% C = 
% \begin{cases} 
% C + \text{base}_{\text{increase}} 
% + 
% \begin{cases} 
% \text{high\_certainty\_boost\_increase}, & \text{if } \text{certaintyScore} > \text{high\_certainty\_threshold} \\
% 0, & \text{otherwise}
% \end{cases} 
% +
% \begin{cases} 
% \text{consecutive\_genuine\_boost}, & \text{if } \text{consec}_{\text{genuine}} \mod 3 = 0 \\
% 0, & \text{otherwise}
% \end{cases}, 
% & \text{for genuine decisions} \\[10pt]

% C - \text{base}_{\text{decrease}} 
% -
% \begin{cases} 
% \text{high\_certainty\_boost\_decrease}, & \text{if } \text{certaintyScore} > \text{high\_certainty\_threshold} \\
% 0, & \text{otherwise}
% \end{cases} 
% -
% \begin{cases} 
% \text{consecutive\_impostor\_penalty}, & \text{if } \text{consec}_{\text{impostor}} \mod 2 = 0 \\
% 0, & \text{otherwise}
% \end{cases}, 
% & \text{for impostor decisions}
% \end{cases}
% \]

\subsubsection{Βασικές Σταθερές}
Ύστερα από μεγάλο αριθμό δοκιμών, οι τιμές των βασικών παραμέτρων της συνάρτησης του σχήματος~\ref{fig:confidenceLevelFunction} καθορίστηκαν στις παρακάτω. Σε επόμενο κεφάλαιο συγκρίνεται η απόδοση της συνάρτησης εμπιστοσύνης με διαφορετικές τιμές των βασικών παραμέτρων.

\begin{itemize}
    \item $\text{base}_{\text{increase}} = 0.06$
    \item $\text{base}_{\text{decrease}} = 0.12$
    \item $\text{high\_certainty\_threshold} = 0.7$
    \item $\text{high\_certainty\_boost\_factor} = 0.4$
    \item $\text{consecutive\_genuine\_boost} = 0.04$
    \item $\text{consecutive\_impostor\_penalty} = 0.05$
    \item $\text{confidence\_threshold} = 0.3$
    \item αρχικό επίπεδο εμπιστοσύνης: $C_0 = 0.6$
\end{itemize}

\subsubsection{Υπολογισμός Ενισχύσεων Υψηλής Βεβαιότητας}

Οι ενισχύσεις λόγω υψηλής βεβαιότητας υπολογίζονται ως:

\[
\text{high\_certainty\_boost\_increase} = \text{base}_{\text{increase}} \times \text{high\_certainty\_boost\_factor} = 0.06 \times 0.4 = 0.024
\]

\[
\text{high\_certainty\_boost\_decrease} = \text{base}_{\text{decrease}} \times \text{high\_certainty\_boost\_factor} = 0.12 \times 0.4 = 0.048
\]

\paragraph{Επίπεδο Βεβαιότητας ($\text{certaintyScore}$)}

Το επίπεδο βεβαιότητας υπολογίζεται με βάση την \texttt{certainty\_level\_function} ως:

\[
\text{certaintyScore} = \frac{\lvert \text{απόσταση απόφασης} \rvert}{\text{μέγιστη απόσταση απόφασης}}
\]

\subsubsection{Ενίσχυση Διαδοχικών Αποφάσεων}
Ο όρος της ενίσχυσης διαδοχικών αποφάσεων προστίθεται ως ένα μέσο σταθερότητας στη συνάρτηση, ώστε να αποτρέπεται η υπερβολική ευαισθησία του συστήματος σε μεμονωμένες ανωμαλίες ή θορύβους. Με αυτόν τον τρόπο ενισχύεται η εμπιστοσύνη στον χρήστη όσο περισσότερο ο ίδιος χρησιμοποιεί το σύστημα, ενώ μειώνεται η εμπιστοσύνη όσο συχνότερα το χρησιμποιεί κάποιος εισβολέας. Βελτιώνεται, συνεπώς, και η συνολικότερη ακρίβεια του συστήματος.


- Για τις γνήσιες αποφάσεις, η ενίσχυση λόγω διαδοχικών γνήσιων αποφάσεων εφαρμόζεται όταν ο αριθμός των διαδοχικών γνήσιων αποφάσεων ($\text{consec}_{\text{genuine}}$) είναι πολλαπλάσιο του 3:

\[
\text{consecutive\_genuine\_boost} = 
\begin{cases} 
0.04, & \text{if } \text{consec}_{\text{genuine}} \mod 3 = 0 \\
0, & \text{otherwise}
\end{cases}
\]

- Για τις αποφάσεις εισβολέα, η ποινή λόγω διαδοχικών αποφάσεων εισβολέα εφαρμόζεται όταν ο αριθμός των διαδοχικών αποφάσεων εισβολέα ($\text{consec}_{\text{impostor}}$) είναι πολλαπλάσιο του 2:

\[
\text{consecutive\_impostor\_penalty} = 
\begin{cases} 
0.05, & \text{if } \text{consec}_{\text{impostor}} \mod 2 = 0 \\
0, & \text{otherwise}
\end{cases}
\]

\subsubsection{Μηχανισμός Κλειδώματος}
Ο μηχανισμός κλειδώματος ενεργοποιείται όταν $C < \text{confidence\_threshold}$. Σε αυτή την περίπτωση:

\begin{itemize}
    \item Το σύστημα απαιτεί επανεξουσιοδότηση μέσω πρόσθετων στοιχείων ταυτοποίησης.
    \item Ο δείκτης $C$ επανέρχεται στο $C_0$ μετά από επιτυχημένη επανεξουσιοδότηση.
\end{itemize}

\subsection{Παραδείγματα Λειτουργίας}
Παρακάτω παρουσιάζονται εκτενώς παραδείγματα λειτουργίας του συστήματος κλειδώματος/εμπιστοσύνης για διαφορετικά σενάρια. Κάθε σενάριο περιλαμβάνει ακολουθία αποφάσεων, το επίπεδο βεβαιότητας ($\text{certaintyScore}$), την αλλαγή στο επίπεδο εμπιστοσύνης ($\Delta C$), και το τελικό επίπεδο εμπιστοσύνης ($C$).

\paragraph{Παράδειγμα 1: Διαδοχικές Γνήσιες Αποφάσεις}
Σε αυτό το παράδειγμα, ο χρήστης λαμβάνει διαδοχικές γνήσιες αποφάσεις με διαφορετικά επίπεδα βεβαιότητας ($\text{certaintyScore}$). Παρατηρούμε ότι κάθε τρίτη διαδοχική γνήσια απόφαση ενισχύεται με τον παράγοντα $\text{consecutive\_genuine\_boost}$:
\[
\Delta C = \text{base}_{\text{increase}} + \text{high\_certainty\_boost\_increase} + \text{consecutive\_genuine\_boost}.
\]

Η επίδραση των τιμών φαίνεται στον πίνακα \ref{tab:trust_examples_1}, όπου το επίπεδο εμπιστοσύνης αυξάνεται σημαντικά μετά από κάθε απόφαση.

\begin{table}[H]
\centering
\begin{tabular}{|c|c|c|c|c|c|}
\hline
\textbf{Απόφαση} & \textbf{$\text{Certainty}$} & \textbf{Διαδοχικές} & \textbf{True Label} & $\Delta C$ & $C$ \\
\hline
Γνήσιος & 0.8 & 1 & Γνήσιος & $+0.06 + 0.024$ & 0.684 \\
Γνήσιος & 0.9 & 2 & Γνήσιος & $+0.06 + 0.024$ & 0.768 \\
Γνήσιος & 0.85 & 3 & Γνήσιος & $+0.06 + 0.024 + 0.04$ & 0.892 \\
Εισβολεάς & 0.82 & 1 & Εισβολεάς & $-0.12 - 0.048$ & 0.724 \\
Εισβολεάς & 0.6 & 2 & Γνήσιος & $-0.12 - 0.05$ & 0.554 \\
Γνήσιος & 0.55 & 1 & Εισβολεάς & $+0.06$ & 0.614 \\
Εισβολεάς & 0.8 & 1 & Εισβολεάς & $-0.12 - 0.048$ & 0.446 \\
\hline
\end{tabular}
\caption{Παραδείγματα Ενημέρωσης Εμπιστοσύνης}
\label{tab:trust_examples_1}
\end{table}

\paragraph{Παράδειγμα 2: Εναλλαγή Γνήσιων και Εισβολέων}
Σε αυτή την περίπτωση, εναλλάσσονται γνήσιες και απατηλές αποφάσεις. Ο πίνακας \ref{tab:trust_examples_2} δείχνει τη σταδιακή μείωση του επιπέδου εμπιστοσύνης λόγω αποφάσεων εισβολέα:

\begin{table}[H]
\centering
\begin{tabular}{|c|c|c|c|c|c|}
\hline
\textbf{Απόφαση} & \textbf{$\text{Certainty}$} & \textbf{Διαδοχικές} & \textbf{True Label} & $\Delta C$ & $C$ \\
\hline
Γνήσιος & 0.75 & 1 & Γνήσιος & $+0.06 + 0.024$ & 0.684 \\
Εισβολεάς & 0.65 & 1 & Γνήσιος & $-0.12$ & 0.564 \\
Γνήσιος & 0.8 & 1 & Γνήσιος & $+0.06 + 0.024$ & 0.648 \\
Εισβολεάς & 0.55 & 1 & Γνήσιος & $-0.12$ & 0.528 \\
Γνήσιος & 0.9 & 1 & Γνήσιος & $+0.06 + 0.024$ & 0.612 \\
Εισβολεάς & 0.78 & 1 & Εισβολεάς & $-0.12 - 0.048$ & 0.444 \\
\hline
\end{tabular}
\caption{Παραδείγματα Εναλλαγής Γνήσιων και Απατηλών Αποφάσεων}
\label{tab:trust_examples_2}
\end{table}

\paragraph{Παράδειγμα 3: Απώλεια Εμπιστοσύνης και Επανεξουσιοδότηση}
Εάν το επίπεδο εμπιστοσύνης πέσει κάτω από το κατώφλι $\text{confidence\_threshold} = 0.3$, ενεργοποιείται ο μηχανισμός κλειδώματος, όπως φαίνεται στον πίνακα \ref{tab:trust_examples_3}. Επίσης, φαίνεται πώς ο μηχανισμός επαναφέρει το επίπεδο εμπιστοσύνης μετά από επιτυχημένη επανεξουσιοδότηση:

\begin{table}[H]
\centering
\begin{tabular}{|c|c|c|c|c|c|}
\hline
\textbf{Απόφαση} & \textbf{$\text{Certainty}$} & \textbf{Διαδοχικές} & \textbf{True Label} & $\Delta C$ & $C$ \\
\hline
Εισβολεάς & 0.6 & 2 & Εισβολεάς & $-0.12 - 0.048 - 0.05$ & 0.102 \\
Κλείδωμα & - & - & - & - & 0.6 \\
Γνήσιος & 0.85 & 1 & Γνήσιος & $+0.06 + 0.024$ & 0.684 \\
\hline
\end{tabular}
\caption{Παραδείγματα Ενεργοποίησης Μηχανισμού Κλειδώματος}
\label{tab:trust_examples_3}
\end{table}

\paragraph{Παράδειγμα 4: Συνεχής Ενίσχυση λόγω Υψηλής Βεβαιότητας}
Σε αυτό το σενάριο, όλες οι αποφάσεις είναι γνήσιες και συνοδεύονται από υψηλή βεβαιότητα ($\text{certaintyScore} > \text{high\_certainty\_threshold}$). Στον πίνακα \ref{tab:trust_examples_4} παρατηρούμε τη σημαντική ενίσχυση του επιπέδου εμπιστοσύνης:

\begin{table}[H]
\centering
\begin{tabular}{|c|c|c|c|c|c|}
\hline
\textbf{Απόφαση} & \textbf{$\text{Certainty}$} & \textbf{Διαδοχικές} & \textbf{True Label} & $\Delta C$ & $C$ \\
\hline
Γνήσιος & 0.9 & 1 & Γνήσιος & $+0.06 + 0.024$ & 0.684 \\
Γνήσιος & 0.95 & 2 & Γνήσιος & $+0.06 + 0.024$ & 0.768 \\
Γνήσιος & 0.92 & 3 & Γνήσιος & $+0.06 + 0.024 + 0.04$ & 0.892 \\
Γνήσιος & 0.94 & 1 & Γνήσιος & $+0.06 + 0.024$ & 0.976 \\
Γνήσιος & 0.97 & 2 & Γνήσιος & $+0.06 + 0.024$ & 1.060 \\
\hline
\end{tabular}
\caption{Παραδείγματα Συνεχούς Ενίσχυσης Λόγω Υψηλής Βεβαιότητας}
\label{tab:trust_examples_4}
\end{table}

\paragraph{Παράδειγμα 5: Επανάληψη Κλειδώματος Λόγω Απατηλών Αποφάσεων}
Σε αυτό το σενάριο, ο χρήστης λαμβάνει συστηματικά αποφάσεις εισβολέα, προκαλώντας επαναλαμβανόμενη ενεργοποίηση του μηχανισμού κλειδώματος, όπως φαίνεται στον πίνακα \ref{tab:trust_examples_5}:

\begin{table}[H]
\centering
\begin{tabular}{|c|c|c|c|c|c|}
\hline
\textbf{Απόφαση} & \textbf{$\text{Certainty}$} & \textbf{Διαδοχικές} & \textbf{True Label} & $\Delta C$ & $C$ \\
\hline
Εισβολεάς & 0.86 & 1 & Εισβολεάς & $-0.12 - 0.048$ & 0.432 \\
Εισβολεάς & 0.92 & 2 & Εισβολεάς & $-0.12 - 0.048 - 0.05$ & 0.214 \\
Κλείδωμα & - & - & - & - & 0.6 \\
Εισβολεάς & 0.91 & 1 & Εισβολεάς & $-0.12 - 0.048$ & 0.432 \\
Εισβολεάς & 0.83 & 2 & Εισβολεάς & $-0.12 - 0.048 - 0.05$ & 0.214 \\
Κλείδωμα & - & - & - & - & 0.6 \\
\hline
\end{tabular}
\caption{Παραδείγματα Επαναλαμβανόμενου Κλειδώματος}
\label{tab:trust_examples_5}
\end{table}


Το παράδειγμα δείχνει ότι η συνεχής απώλεια εμπιστοσύνης λόγω αποφάσεων εισβολέα οδηγεί σε συχνή ενεργοποίηση του μηχανισμού κλειδώματος. Αυτό εξασφαλίζει την προστασία του συστήματος από κακόβουλη χρήση.

\subsection{Παρατηρήσεις}
Τα παραπάνω παραδείγματα αναδεικνύουν τη λειτουργία του συστήματος εμπιστοσύνης σε διαφορετικά σενάρια χρήσης. Οι μαθηματικοί υπολογισμοί και οι διαδοχικές αποφάσεις παρουσιάζουν τη δυναμική φύση του μηχανισμού εμπιστοσύνης, ο οποίος μπορεί να προσαρμοστεί σε διαφορετικά μοτίβα συμπεριφοράς χρηστών. Ο συνδυασμός υψηλής βεβαιότητας, διαδοχικών αποφάσεων και μηχανισμού κλειδώματος διασφαλίζει την ασφάλεια και την αξιοπιστία του συστήματος.

Το σύστημα κλειδώματος/εμπιστοσύνης παρέχει ένα δυναμικό μέσο διαχείρισης αποφάσεων, ενισχύοντας την ασφάλεια και την αξιοπιστία. Η μαθηματική του θεμελίωση το καθιστά ικανό να προσαρμόζεται σε διαφορετικά σενάρια χρήσης.

\section{Παρουσίαση Διεπαφής Χρήστη}
\label{sec:implementations_streamlit}

Το κεφάλαιο αυτό παρουσιάζει το γραφικό περιβάλλον χρήστη που αναπτύχθηκε μέσω της βιβλιοθήκης Streamlit\footnote{\url{https://streamlit.io/}}, το οποίο σχεδιάστηκε για να υλοποιεί τη διαδικασία συνεχούς και έμμεσης αυθεντικοποίησης. Το περιβάλλον αυτό επιτρέπει την αλληλεπίδραση του χρήστη με το σύστημα μέσω απλών και κατανοητών λειτουργιών.

Η ενσωμάτωση του Streamlit UI επιτρέπει την άμεση αλληλεπίδραση των χρηστών με το σύστημα. Οι βασικές λειτουργίες περιλαμβάνουν:
\begin{itemize}
    \item \textbf{Εισαγωγή Username}: Ο χρήστης εισάγει ένα username μέσω του UI.
    \item \textbf{Εισαγωγή Prompt}: Ο χρήστης εισάγει ένα prompt μέσω του UI.
    \item \textbf{Κουμπί Αυθεντικοποίησης}: Το σύστημα εμφανίζει τα αποτελέσματα της απόφασης (γνήσιος χρήστης ή εισβολεάς) και το επίπεδο εμπιστοσύνης.
\end{itemize}

\begin{figure}[H]
    \centering
    \includegraphics[width=0.75\textwidth]{images/chapter4/streamlitUI.png}
    \caption{Περιβάλλον Χρήστη του Streamlit UI}
    \label{fig:streamlit_ui}
\end{figure}

Η διαδικασία αυθεντικοποίησης βασίζεται στην πρόβλεψη που πραγματοποιεί το μοντέλο, ενώ η εφαρμογή επιστρέφει είτε "Access Granted" είτε "Access Denied" συνοδευόμενη από τη βαθμολογία βεβαιότητας (\textit{certainty score}) της απόφασης. Ο πλήρης κώδικας βρίσκεται στο \emph{Github}\footnote{\href{https://github.com/conmylo/master-thesis/tree/main/final}{https://github.com/conmylo/master-thesis/tree/main/final}}.

\subsection{Παραδείγματα Χρήσης}
Παρακάτω παρουσιάζονται παραδείγματα από τη λειτουργία του Streamlit UI.

\begin{figure}[H]
    \centering
    \begin{subfigure}{\textwidth}
        \centering
        \includegraphics[width=0.75\textwidth]{images/chapter4/streamlitGrant.png}
        \caption{Παράδειγμα έγκρισης αυθεντικοποίησης με υψηλό certainty score.}
        \label{fig:streamlitGrant}
    \end{subfigure}
    \hfill
    \begin{subfigure}{\textwidth}
        \centering
        \includegraphics[width=0.75\textwidth]{images/chapter4/streamlitDeny.png}
        \caption{Παράδειγμα απόρριψης αυθεντικοποίησης με χαμηλό certainty score.}
        \label{fig:streamlitDeny}
    \end{subfigure}
    \caption{Στιγμιότυπα οθόνης από το streamlit UI για επιβεβαίωση και απόρριψη αυθεντικοποίησης}
    \label{fig:subgraphs}
\end{figure}

Στην εικόνα \ref{fig:streamlitGrant}, ο χρήστης "barackobama" εισάγει ένα prompt που το σύστημα αναγνωρίζει ως έγκυρο, παραχωρώντας την πρόσβαση με certainty score 0.52.

Στην εικόνα \ref{fig:streamlitDeny}, ο ίδιος χρήστης εισάγει ένα prompt που το σύστημα αξιολογεί ως μη έγκυρο, απορρίπτοντας την πρόσβαση με certainty score 0.35.


