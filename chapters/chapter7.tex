\chapter{Μελλοντικές επεκτάσεις}
\label{chapter:future_work}

Οι μελλοντικές επεκτάσεις της παρούσας εργασίας αποσκοπούν στη συνεχή βελτίωση, διεύρυνση και αναβάθμιση του συστήματος καθώς και στην εφαρμογή του σε πιο απαιτητικά και πολυδιάστατα περιβάλλοντα. Ορμώμενοι από τις δυνατότητες που παρέχει η τρέχουσα προσέγγιση, αναλογιζόμαστε τη συνεισφορά ενός τέτοιου μοντέλου στον ευρύτερο κλάδο της τεχνολογίας.

Αρχικά, ο εμπλουτισμός των δεδομένων αποτελεί έναν από τους βασικούς πυλώνες εξέλιξης του συστήματος. Η ενσωμάτωση νέων κειμενικών δεδομένων, όπως διαφορετικές μορφές γραφής ή μεγαλύτερης πολυπλοκότητας κείμενα, μπορεί να επεκτείνει τη λειτουργικότητα του μοντέλου. Παράλληλα, η υποστήριξη μακρύτερων κειμένων και διαφορετικών γλωσσών μπορεί να ενισχύσει την ικανότητα του συστήματος να επεξεργάζεται και να αναλύει δεδομένα από ευρύτερο φάσμα πηγών. Τέτοιες προσθήκες επιτρέπουν στο μοντέλο να προσαρμόζεται καλύτερα σε πιο σύνθετα περιβάλλοντα, αυξάνοντας έτσι τη συνολική του χρηστικότητα. Προτείνεται, μάλιστα, και ο συνδυασμός μοντέλων, π.χ. OCSVM σε συνδυασμό με autoencoders, για τη καλύτερη απόδοση και προσαρμοστικότητα του συστήματος.

Η ενσωμάτωση νέων χρηστών και η δυνατότητα προσαρμοστικής εκπαίδευσης αποτελούν επίσης κρίσιμες βελτιώσεις. Το μοντέλο θα πρέπει να έχει τη δυνατότητα να ενσωματώνει νέους χρήστες άμεσα, εξαλείφοντας την ανάγκη εκτεταμένης αρχικής εκπαίδευσης. Επιπλέον, η συνεχής επανεκπαίδευση στις γραφικές συνήθειες των υφιστάμενων χρηστών, ειδικά μετά από κάθε γνήσια προτροπή, μπορεί να βοηθήσει το σύστημα να παρακολουθεί δυναμικά τις αλλαγές στα πρότυπα των χρηστών και να προσαρμόζεται αναλόγως. Αυτό εξασφαλίζει μεγαλύτερη ευελιξία και εξατομίκευση στη χρήση του συστήματος. 

Η βελτιστοποίηση της διαδικασίας εξαγωγής χαρακτηριστικών είναι επίσης καίριας σημασίας για την περαιτέρω αναβάθμιση του μοντέλου. Η εισαγωγή νέων χαρακτηριστικών, όπως ανάλυση συναισθημάτων και θεματική κατηγοριοποίηση, μπορεί να συμβάλει στη βελτίωση της ακρίβειας των προβλέψεων. Επιπλέον, η ανάπτυξη εξατομικευμένων ορίων (personalized thresholds) για κάθε χρήστη ενισχύει την προσαρμοστικότητα του συστήματος στις ιδιαίτερες ανάγκες και συμπεριφορές των χρηστών.

Μια κρίσιμη μελλοντική επέκταση αφορά τη βελτίωση της συνάρτησης επίπεδου εμπιστοσύνης (confidence level function). Η υπάρχουσα λειτουργικότητα μπορεί να επεκταθεί ώστε η τιμή της ενίσχυσης ή της αποδυνάμωσης του επιπέδου εμπιστοσύνης να μεταβάλλεται δυναμικά, ανάλογα με τον αριθμό διαδοχικών γνήσιων ή μη γνήσιων προτροπών (genuine or impostor prompts). Συγκεκριμένα, για κάθε διαδοχική γνήσια προτροπή, η συνάρτηση μπορεί να αυξάνει τη θετική ενίσχυση εκθετικά ή με βάση έναν προκαθορισμένο συντελεστή, ενισχύοντας την εμπιστοσύνη στη γνησιότητα του χρήστη. Αντίστοιχα, για διαδοχικές μη γνήσιες προτροπές, η αποδυνάμωση της εμπιστοσύνης μπορεί να γίνεται ταχύτερα, διασφαλίζοντας την έγκαιρη ανίχνευση πιθανών παραβιάσεων.

Μια περαιτέρω δυνατότητα βελτίωσης είναι η ενσωμάτωση εξειδικευμένων παραμέτρων που λαμβάνουν υπόψη τον ρυθμό εμφάνισης των διαδοχικών προτροπών. Για παράδειγμα:
\begin{itemize}
    \item{Εξισορρόπηση μέσω προσαρμοστικών τιμών}: Το επίπεδο εμπιστοσύνης μπορεί να επανέρχεται σταδιακά σε ουδέτερη κατάσταση όταν παρατηρούνται εναλλαγές μεταξύ γνήσιων και μη γνήσιων προτροπών, ώστε να αποφεύγονται οι ψευδείς συναγερμοί.
    \item{Προσαρμογή ανά χρήστη}: Το σύστημα μπορεί να επιτρέπει την παραμετροποίηση της συνάρτησης ανά χρήστη, προσαρμόζοντας τη δυναμική της μεταβολής της εμπιστοσύνης στις ιδιαίτερες συνήθειες και τη συμπεριφορά του.
\end{itemize}

Επιπλέον, μπορεί να διερευνηθεί η χρήση της confidence level function ως εργαλείου πρόβλεψης, όπου το σύστημα θα επιχειρεί να προβλέψει τις επόμενες ενέργειες του χρήστη, βασιζόμενο σε ιστορικά δεδομένα. Αυτό μπορεί να περιλαμβάνει την ενσωμάτωση μηχανισμών μηχανικής μάθησης που θα βελτιστοποιούν τη λειτουργία της συνάρτησης με βάση μεγάλα σύνολα δεδομένων, επιτρέποντας έτσι στο σύστημα να προσαρμόζεται με μεγαλύτερη ακρίβεια και ταχύτητα στις ανάγκες του εκάστοτε χρήστη. Τέτοιες βελτιώσεις ενισχύουν τη συνολική απόδοση του συστήματος, παρέχοντας ένα πιο ασφαλές και αξιόπιστο περιβάλλον χρήσης.

Μια σημαντική μελλοντική επέκταση αφορά τη λειτουργικότητα real-time monitoring και απόκρισης. Η δυνατότητα αυτή περιλαμβάνει την παρακολούθηση σε πραγματικό χρόνο της χρήσης του συστήματος και την άμεση απόκριση σε ασυνήθιστες ή ύποπτες δραστηριότητες. Συγκεκριμένα, το σύστημα μπορεί να επεκταθεί με την εισαγωγή αυτόματων ειδοποιήσεων στον χρήστη όταν εντοπίζονται αποκλίσεις από τις συνηθισμένες του γραφικές συνήθειες. Επιπλέον, η δυνατότητα παρεμβάσεων σε πραγματικό χρόνο, όπως προσωρινό κλείδωμα του συστήματος σε περιπτώσεις ύποπτης δραστηριότητας ή παροχή καθοδήγησης στον χρήστη, μπορεί να αυξήσει το επίπεδο ασφάλειας και να ενισχύσει την εμπιστοσύνη στη λειτουργικότητα του συστήματος.

Όλες οι παραπάνω επεκτάσεις στοχεύουν στη μετατροπή του συστήματος σε μια προηγμένη και πολυδιάστατη πλατφόρμα που μπορεί να προσαρμόζεται δυναμικά στις ανάγκες διαφορετικών χρηστών και περιβαλλόντων. Με τη σταδιακή υλοποίηση αυτών των βελτιώσεων, το μοντέλο μπορεί να εδραιωθεί ως ένα καινοτόμο εργαλείο που εξυπηρετεί ποικίλες εφαρμογές και κλάδους.
