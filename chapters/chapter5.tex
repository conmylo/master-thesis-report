\chapter{Πειράματα - Αποτελέσματα}
\label{chapter:experiments}

Η παρούσα ενότητα επικεντρώνεται στην παρουσίαση των αποτελεσμάτων που προέκυψαν από τα πειράματα και στην αξιολόγηση του συστήματος συνολικά. Σκοπός της ενότητας είναι να προσφέρει μια σαφή εικόνα της απόδοσης του συστήματος υπό διαφορετικές συνθήκες και παραμέτρους, αλλά και να συγκρίνει τις διαφορετικές προσεγγίσεις που χρησιμοποιήθηκαν.

Αρχικά, περιγράφεται η διαδικασία αξιολόγησης και αναλύονται οι μετρικές που χρησιμοποιούνται για τη μέτρηση της απόδοσης του συστήματος, όπως το \textit{False Rejection Rate (FRR)}, το \textit{False Acceptance Rate (FAR)}, και οι μέσοι αριθμοί προτροπών για γνήσιους χρήστες και εισβολείς. Η επιλογή των συγκεκριμένων μετρικών βασίζεται στη δυνατότητά τους να αποτυπώνουν με ακρίβεια την απόδοση του συστήματος σε διαφορετικά σενάρια.

Η ανάλυση των αποτελεσμάτων δομείται σε διαφορετικές φάσεις, κάθε μία από τις οποίες εστιάζει σε συγκεκριμένες πτυχές της απόδοσης του συστήματος.

Στην πρώτη φάση, αξιολογείται η απόδοση του συστήματος όταν χρησιμοποιείται ένα μοντέλο OC-SVM ανά χρήστη. Παρουσιάζονται τα αποτελέσματα για διάφορους συνδυασμούς των υπερπαραμέτρων $nu$, $\gamma$, και \textit{threshold}, αποτυπώνοντας την ευαισθησία του μοντέλου στις παραμέτρους αυτές.

Η δεύτερη φάση αφορά την τεχνική \textit{basic majority voting}, όπου συνδυάζονται πολλαπλά μοντέλα OC-SVM ανά χρήστη. Παρουσιάζονται οι βελτιώσεις στην απόδοση και αναλύεται η συνεισφορά της πλειοψηφικού μοντέλου.

Η τρίτη φάση ενσωματώνει τη μέθοδο \textit{weighted majority voting} και τη συνάρτηση επιπέδου βεβαιότητας (\textit{certainty level}), αναδεικνύοντας πώς ο συνδυασμός των τεχνικών αυτών βελτιώνει τη συνολική απόδοση του συστήματος. Γίνεται χρήση της σταθμισμένης πλειοψηφίας, ώστε κάθε μοντέλο να συνεισφέρει στο τελικό αποτέλεσμα με βάση τη βεβαιότητά του για την απόφαση.

Στην τέταρτη φάση, παρουσιάζεται η εισαγωγή της συνάρτησης επιπέδου εμπιστοσύνης (\textit{confidence level function}). Αναλύεται πώς συμβάλλουν στην σταθεροποίηση του συστήματος και την μεγαλύτερη ακρίβειά του τόσο η ενίσχυση των διαδοχικών αποφάσεων, όσο και ο διαχωρισμός μεταξύ των περισσότερο βέβαιων αποφάσεων από τις υπόλοιπες.

Στην πέμπτη φάση, αξιολογείται η γενίκευση του συστήματος μέσω της τεχνικής Leave-One-Subject-Out Cross Validation. Τα αποτελέσματα αποτυπώνουν την απόδοση του συστήματος όταν εκπαιδεύεται σε δεδομένα άλλων χρηστών και δοκιμάζεται στα δεδομένα του αποκλεισμένου χρήστη.

Τέλος, πραγματοποιείται σύγκριση των αποτελεσμάτων από όλες τις φάσεις, αναδεικνύοντας τις βελτιώσεις που προκύπτουν από κάθε μέθοδο. Η συνολική εικόνα που παρέχεται επιτρέπει την αποτίμηση της απόδοσης του συστήματος και τη διερεύνηση δυνατοτήτων περαιτέρω βελτίωσης.

Για λόγους πληρότητας, ακολουθεί η περιγραφή του συστήματος που χρησιμοποιήθηκε στην εκτέλεση των πειραμάτων:
\begin{center}
\small
\begin{tabular}{ | l | l | l | l | l | l | l | }
  \hline
  \rowcolor{Gray}
  Μονάδα & \# πυρήνων & RAM \\
  \hline
  i7-1260p & 16 & 32 \\
  \hline
\end{tabular}
\end{center}

\section{Αξιολόγηση Συστήματος}
\label{sec:exmperiments_metrics}

\subsection{Εισαγωγή}
Το σύστημα δοκιμών έχει σχεδιαστεί για να αξιολογήσει την απόδοση και την αξιοπιστία του συστήματος αυθεντικοποίησης σε διαφορετικά σενάρια. Εστιάζουμε στη συμπεριφορά του συστήματος όταν διαχειρίζεται γνήσιους χρήστες και εισβολείς, αναλύοντας την ακρίβεια και την αποτελεσματικότητά του.

\subsection{Δεδομένα Δοκιμών}
Τα δεδομένα που χρησιμοποιήθηκαν στις δοκιμές περιλαμβάνουν:
\begin{itemize}
    \item \textbf{Γνήσια δεδομένα χρηστών}: Κείμενα από γνήσιους χρήστες που εκπαιδεύτηκαν στο σύστημα.
    \item \textbf{Δεδομένα εισβολέων}: Κείμενα από άλλους χρήστες του dataset εκτός του προφίλ του εκάστοτε γνήσιου χρήστη.
\end{itemize}

\subsection{Μετρικές Αξιολόγησης}
\label{subsec:metrics}
Για την αξιολόγηση του συστήματος χρησιμοποιούνται οι παρακάτω μετρικές:
\begin{itemize}
    \item \textbf{F1 Score):}

    Η μετρική F1 Score είναι μια σταθμισμένη μέση τιμή της ανάκλησης (\textbf{Recall}) και της ακρίβειας (\textbf{Precision}), η οποία χρησιμοποιείται για την αξιολόγηση της απόδοσης ενός συστήματος. Στο πλαίσιο του συστήματος αυθεντικοποίησης, η F1 υπολογίζεται ως εξής:

    \[
    \text{F1} = 2 \cdot \frac{\text{Precision} \cdot \text{Recall}}{\text{Precision} + \text{Recall}}
    \]
    
    Όπου:
    \begin{itemize}
    \item \textbf{Precision (Ακρίβεια)}:
    \[
    \text{Precision} = \frac{\text{True Positives (TP)}}{\text{True Positives (TP)} + \text{False Positives (FP)}}
    \]
    Η ακρίβεια αντιπροσωπεύει το ποσοστό των προτροπών που ταξινομήθηκαν ως γνήσιες (\textit{genuine}) και ήταν πράγματι γνήσιες. Υψηλή ακρίβεια σημαίνει ότι το σύστημα αποφεύγει λανθασμένες αποδοχές εισβολέων (\textit{impostors}).
    
    \item \textbf{Recall (Ανάκληση)}:
    \[
    \text{Recall} = \frac{\text{True Positives (TP)}}{\text{True Positives (TP)} + \text{False Negatives (FN)}}
    \]
    Η ανάκληση μετρά το ποσοστό των πραγματικά γνήσιων προτροπών που αναγνωρίστηκαν σωστά από το σύστημα. Υψηλή ανάκληση δείχνει ότι το σύστημα αποφεύγει να απορρίψει γνήσιες προτροπές ως εισβολείς.
    
    \item \textbf{True Positives (TP)}:
    Προτροπές που το σύστημα ταξινόμησε σωστά ως γνήσιες.
    
    \item \textbf{False Positives (FP)}:
    Προτροπές που το σύστημα ταξινόμησε λανθασμένα ως γνήσιες ενώ ήταν εισβολείς.
    
    \item \textbf{False Negatives (FN)}:
    Προτροπές που το σύστημα ταξινόμησε λανθασμένα ως εισβολείς ενώ ήταν γνήσιες.
    \end{itemize}
    
    \item \textbf{False Rejection Rate (FRR)}: Υπολογίζεται ως:
    \[
    \text{FRR} = \frac{\text{False Rejections}}{\text{Total Genuine Prompts}}
    \]
    \item \textbf{False Acceptance Rate (FAR)}: Υπολογίζεται ως:
    \[
    \text{FAR} = \frac{\text{False Acceptances}}{\text{Total Impostor Prompts}}
    \]
    \item \textbf{Accuracy (Ακρίβεια Συνολική):}
    \[\text{Accuracy} = \frac{\text{True Positives (TP)} + \text{True Negatives (TN)}}{\text{Total Predictions}}\]
    Η συνολική ακρίβεια μετρά το ποσοστό των προτροπών που ταξινομήθηκαν σωστά, είτε ως γνήσιες είτε ως εισβολείς. Υψηλή τιμή ακρίβειας δείχνει τη συνολική αποτελεσματικότητα του συστήματος.

    \item \textbf{MAE (Μέσο Απόλυτο Σφάλμα):}
    \[\text{MAE} = \frac{\sum_{i=1}^{n} |\hat{y}_i - y_i|}{n}\]
    Το MAE μετρά το μέσο απόλυτο σφάλμα μεταξύ των προβλεπόμενων τιμών (\(\hat{y}_i\)) και των πραγματικών τιμών (\(y_i\)). Μια χαμηλή τιμή MAE υποδηλώνει ότι οι προβλέψεις του συστήματος είναι κοντά στις πραγματικές τιμές, καταδεικνύοντας την ακρίβεια και την αξιοπιστία του μοντέλου.
    
    \item \textbf{Μέσος Αριθμός Προτροπών για Γνήσιους Χρήστες}:
    \[
    \text{Mean Genuine Prompts Before Lock} = \frac{\text{Σύνολο Προτροπών Γνήσιων Χρηστών}}{\text{Συνολικός Αριθμός Locks για Γνήσιους}}
    \]
    \item \textbf{Μέσος Αριθμός Προτροπών για εισβολείς}:
    \[
    \text{Mean Impostor Prompts Before Lock} = \frac{\text{Σύνολο Προτροπών εισβολέων}}{\text{Συνολικός Αριθμός Locks για εισβολείς}}
    \]

\end{itemize}

Οι μετρικές FAR και FRR είναι αντιστρόφως ανάλογες που σημαίνει πως όταν η μία αυξάνεται, η άλλη μειώνεται όπως φαίνεται στο παρακάτω σχήμα:
\begin{figure}[H]
    \centering
    \includegraphics[width=0.7\textwidth]{images/chapter4/farVSfar.png}
    \caption{Ποιοτική απεικόνηση των FAR και FRR.}
    \label{fig:chapter4_farVSfar}
\end{figure}
Το κόκκινο σημείο ονομάζεται Equal Error Rate - EER και αναπαριστά το επίπεδο ασφάλειας για το οποίο οι τιμές των FAR και FRR είναι ίσες, όπως φαίνεται στο~\autoref{fig:chapter4_farVSfar}. Ερμηνεύοντας το σχήμα από μια πιο πρακτική προσέγγιση, γίνεται αντιληπτό πως όσο ασφαλέστερο είναι ένα σύστημα, τόσο λιγότερο βολικό θα είναι για τον χρήστη καθώς θα κλειδώνει συχνότερα και, αντίστοιχα, όσο λιγότερη ασφάλεια το χαρακτηρίζει, τόσο ευκολότερο θα είναι για τον πραγματικό χρήστη να το χειριστεί, αλλά και για τον υποκλοπέα να αυθεντικοποιηθεί από το σύστημα.

\subsection{Ροή Διαδικασίας Δοκιμών}
Η διαδικασία δοκιμών ακολουθεί συγκεκριμένα βήματα, διασφαλίζοντας την αξιολόγηση κάθε μοντέλου και χρήστη ξεχωριστά. Παρακάτω περιγράφονται τα στάδια της ροής:

\subsubsection{Προετοιμασία δεδομένων} 
Η προετοιμασία περιλαμβάνει τη δημιουργία δύο συνόλων δεδομένων για κάθε χρήστη:
\begin{itemize}
    \item \textbf{Διαχωρισμός Δεδομένων Χρήστη:} Για κάθε χρήστη, το 15\% των δεδομένων του αφαιρείται από το σύνολο εκπαίδευσης και διατίθεται ως σύνολο δοκιμών (\emph{test set}). Το υπόλοιπο 85\% έχει ήδη χρησιμοποιηθεί για την εκπαίδευση των μοντέλων, όπως περιγράφεται στο Κεφάλαιο~\ref{sec:implementation_train}.
    \item \textbf{Δημιουργία Impostor Dataset:} Για κάθε χρήστη, επιλέγεται ένα σύνολο από κείμενα που ανήκουν σε άλλους χρήστες. Αυτά τα δεδομένα σχηματίζουν το \emph{impostor dataset}, το οποίο χρησιμοποιείται για την αξιολόγηση της ικανότητας του συστήματος να εντοπίζει μη γνήσιους χρήστες. Το μέγεθος του impostor test set αυτού είναι ίσο με το μέγεθος του genuine test set του κάθε χρήστη (50-50 split).
\end{itemize}

\subsubsection{Χρήση Εκπαιδευμένων Μοντέλων}
Τα εκπαιδευμένα μοντέλα χρησιμοποιούνται για να αξιολογήσουν τα κείμενα του \emph{genuine test set} κάθε χρήστη και του \emph{impostor test set}. Για κάθε κείμενο:
\begin{itemize}
    \item Εξάγονται τα χαρακτηριστικά της κάθε εγγραφής των 2 test set (genuine \& impostor) μέσω της συνάρτησης \texttt{extract\_features}, όπως περιγράφεται στο Κεφάλαιο~\ref{sec:implementation_features}.
    \item Τα χαρακτηριστικά εισάγονται στα εκπαιδευμένα μοντέλα (\emph{One-Class SVM}) για την παραγωγή απόφασης και του επιπέδου βεβαιότητας (\emph{certainty level}).
\end{itemize}

\subsubsection{Συλλογή Αποτελεσμάτων}
Η συλλογή των αποτελεσμάτων γίνεται σε δύο επίπεδα:
\begin{itemize}
    \item \textbf{Αποτελέσματα Ανά Χρήστη:} Για κάθε χρήστη καταγράφονται:
    \begin{itemize}
        \item Οι τιμές των μετρικών αξιολόγησης (\emph{FRR}, \emph{FAR}, μέσος αριθμός προτροπών πριν το κλείδωμα για γνήσιους χρήστες, μέσος αριθμός προτροπών πριν το κλείδωμα για εισβολείς).
        \item Οι τιμές βεβαιότητας (\emph{certainty scores}) για κάθε κείμενο.
    \end{itemize}
    \item \textbf{Συνολικά Αποτελέσματα:} Τα αποτελέσματα όλων των χρηστών συνδυάζονται για την εξαγωγή συνολικών στατιστικών, παρέχοντας μια ολοκληρωμένη εικόνα της απόδοσης του συστήματος.
\end{itemize}

Η διαδικασία δοκιμών διασφαλίζει την αντικειμενική αξιολόγηση της απόδοσης του συστήματος και την καταγραφή των μετρικών σε επίπεδο χρήστη αλλά και συνολικά για το σύστημα.
\section{Πρώτη Φάση Πειραμάτων: 1 Μοντέλο Ανά Χρήστη}
\label{sec:experiments_phase1}

\subsection{Πειραματική Διαδικασία}
Στη πρώτη φάση πειραμάτων, χρησιμοποιούμε ένα μοντέλο OC-SVM ανά χρήστη με σαφείς καθορισμένες παραμέτρους. Η πειραματική διαδικασία μπορεί να αναλυθεί σε:
\begin{itemize}
  \item{Προετοιμασία train \& test set για κάθε χρήστη.}
  \item{Εκπαίδευση μοντέλων με διαφορετικούς συνδυασμούς nu \& gamma για όλους τους χρήστες.}
  \item{Έλεγχος test prompts για όλους τους χρήστες με διαφορετικές τιμές για το threshold.}
  \item{Μετρικές αξιολόγησης - F1 \& FAR \& FRR - για συνδυασμούς παραμέτρων.}
\end{itemize}

\subsection{Hyperparameter \& Threshold Tuning}
\label{subsec:hypertuning}

\paragraph{Διαδικασία Tuning}
Η διαδικασία tuning είχε στόχο τη βελτιστοποίηση των παραμέτρων $nu$, $\gamma$ και του κατωφλιού (\textit{threshold}) για την εξασφάλιση της μέγιστης απόδοσης του συστήματος. Αξιοποιήθηκε η τεχνική grid search, κατά την οποία δίνουμε στο σύστημα ένα πλέγμα τιμών (grid) και μας επιστρέφει αποτελέσματα για κάθε δυνατό συνδυασμό τιμών του πλέγματος. Οι παράμετροι δοκιμάστηκαν στις εξής τιμές:
\begin{itemize}
    \item $nu$: \{0.0001, 0.001, 0.005, 0.01, 0.05, 0.1\}
    \item $\gamma$: \{0.05, 0.1, 0.2, 0.5, 1.0\}
    \item Κατώφλι: \{-0.01, 0.0, 0.01, 0.02, 0.03, 0.04, 0.05, 0.06, 0.07, 0.08\}
\end{itemize}

Οι μετρικές που αξιολογήθηκαν, όπως αναφέρεται και στο \autoref{sec:exmperiments_metrics}, περιλαμβάνουν:
\begin{itemize}
    \item \textbf{F1 Score:} Ο σταθμισμένος μέσος όρος της ακρίβειας και της ανάκλησης.
    \item \textbf{False Acceptance Rate (FAR):} Ποσοστό προτροπών εισβολέων που έγιναν δεκτές.
    \item \textbf{False Rejection Rate (FRR):} Ποσοστό γνήσιων προτροπών που απορρίφθηκαν.
\end{itemize}

\paragraph{Αποτελέσματα Tuning}
Τα αποτελέσματα της αναζήτησης πλέγματος παρουσιάζονται συνοπτικά στον Πίνακα~\ref{tab:balanced_results}. Οι καλύτερες τιμές των μετρικών για κάθε συνδυασμό παραμέτρων εμφανίζονται με έμφαση.

\begin{table}[H]
\centering
\begin{tabular}{|c|c|c|c|c|c|}
\hline
\textbf{$\nu$} & \textbf{$\gamma$} & \textbf{Threshold} & \textbf{F1 Score} & \textbf{FAR (\%)} & \textbf{FRR (\%)} \\ \hline
0.01 & 0.5 & 0.05 & \textbf{0.49707} & \textbf{44.18} & \textbf{49.58} \\ \hline
0.005 & 0.2 & 0.07 & \textbf{0.51769} & \textbf{43.85} & \textbf{46.72} \\ \hline
0.01 & 0.1 & 0.05 & 0.67351 & 86.02 & 5.73 \\ \hline
0.005 & 0.1 & 0.07 & 0.67279 & 85.75 & 5.97 \\ \hline
0.001 & 0.2 & 0.0 & 0.68526 & 86.38 & 3.18 \\ \hline
0.01 & 0.2 & 0.03 & 0.67170 & 81.88 & 8.37 \\ \hline
0.01 & 0.5 & 0.07 & \textbf{0.41699} & \textbf{35.20} & \textbf{39.90} \\ \hline
0.005 & 0.1 & 0.03 & 0.47475 & 41.65 & 46.38 \\ \hline
0.01 & 0.2 & 0.05 & \textbf{0.66543} & \textbf{39.85} & \textbf{38.92} \\ \hline
0.01 & 0.5 & 0.06 & 0.49714 & 41.58 & 45.62 \\ \hline
\end{tabular}
\caption{Επιλεγμένα Αποτελέσματα Tuning Παραμέτρων με Ισορροπία FAR και FRR.}
\label{tab:balanced_results}
\end{table}

\subsubsection{Παραδείγματα Ισορροπίας FAR και FRR}
\begin{itemize}
    \item \textbf{(ν = 0.01, γ = 0.5, threshold = 0.05):} 
    Παρουσιάζεται ισχυρή ισορροπία μεταξύ FAR (44.18\%) και FRR (49.58\%), αν καιτο F1 είναι σχετικά χαμηλό. Υποδεικνύεται ότι το σύστημα διαχειρίζεται ομοιόμορφα τους genuine και impostor χρήστες, ωστόσο τα ποσοστά των μετρικών δεν είναι ικανοποιητικά χαμηλά.
    
    \item \textbf{(ν = 0.01, γ = 0.5, threshold = 0.07):}
    Τα FAR και FRR είναι πιο χαμηλά (35.20\%, 39.90\% αντίστοιχα), υποδεικνύοντας καλύτερη διαχείριση τόσο των γνήσιων όσο και των εισβολέων χρηστών.

    \item \textbf{(ν = 0.01, γ = 0.2, threshold = 0.05):}
    Το παράδειγμα αυτό παρουσιάζει εξαιρετικά ισορροπημένες τιμές, με FAR (39.85\%) και FRR (38.92\%).
\end{itemize}

\subsubsection{Παραδείγματα Ακραίων Τιμών}
\begin{itemize}
    \item \textbf{(ν = 0.001, γ = 0.2, threshold = 0.0):}
    Παρόλο που το F1 είναι σχετικά υψηλό (0.68526), το FAR (86.38\%) είναι πολύ μεγαλύτερο από το FRR (3.18\%). Αυτό δείχνει ότι το σύστημα είναι πολύ επιεικές με genuine prompts, αλλά δυσκολεύεται να απορρίψει impostors.
    \item \textbf{(nu: 0.001, gamma: 0.05, threshold: 0.01):}
    Σε αυτή τη περίπτωση το FAR είναι εξαιρετικά χαμηλό, υπονοώντας πως το σύστημα καταφέρνει με επιτυχία να αναγνωρίσει impostors, αλλά η τιμή του FRR (97.62\%) υποδηλωνει πως το σύστημα σπάνια αναγνωρίζει τα genuine prompts. Το F1 παραμένει υψηλό.
\end{itemize}

\begin{figure}[H]
    \centering
    \includegraphics[width=\textwidth]{images/chapter5/5.1.2.png}
    \caption{FAR, FRR, F1 για διαφορετικά πλέγματα nu, gamma \& threshold}
    \label{fig:chapter5_image512}
\end{figure}

Στο~\autoref{fig:chapter5_image512} βλέπουμε τις συνολικές τιμές των μετρικών για διαφορετικά πλέγματα παραμέτρων.

\subsection{Αποτελέσματα Ανά Χρήστη με συγκεκριμένο πλέγμα}
\label{sec:results_threshold_0.05}

Ο παρακάτω πίνακας παρουσιάζει τις επιδόσεις του συστήματος για κάθε χρήστη με ρύθμιση κατωφλίου $0.05$, τη τιμή του nu $0.001$ και του gamma $0.05$. Οι μετρικές περιλαμβάνουν την ακρίβεια (\emph{Precision}), την ανάκληση (\emph{Recall}), το F1 Score, το \emph{False Acceptance Rate} (FAR), και το \emph{False Rejection Rate} (FRR). Επιπλέον, εμφανίζονται τα συνολικά FAR και FRR.

\begin{table}[H]
\centering
\begin{tabular}{|l|c|c|c|c|c|}
\hline
\textbf{User} & \textbf{Precision} & \textbf{Recall} & \textbf{F1 Score} & \textbf{FAR (\%)} & \textbf{FRR (\%)} \\ \hline
BarackObama & 1.00000 & 0.00000 & 0.00000 & 0.00 & 100.00 \\ \hline
Cristiano & 0.50828 & 0.95722 & 0.66399 & 92.60 & 4.28 \\ \hline
TheEllenShow & 0.52645 & 0.96759 & 0.68189 & 87.04 & 3.24 \\ \hline
Twitter & 0.52127 & 0.96898 & 0.67787 & 88.99 & 3.10 \\ \hline
YouTube & 0.52500 & 0.94508 & 0.67502 & 85.50 & 5.49 \\ \hline
britneyspears & 0.51703 & 0.95274 & 0.67030 & 88.99 & 4.73 \\ \hline
cnnbrk & 0.70671 & 0.49186 & 0.58003 & 20.41 & 50.81 \\ \hline
instagram & 0.65110 & 0.56267 & 0.60366 & 30.15 & 43.73 \\ \hline
jimmyfallon & 0.47433 & 0.07401 & 0.12805 & 8.20 & 92.60 \\ \hline
jtimberlake & 1.00000 & 0.00000 & 0.00000 & 0.00 & 100.00 \\ \hline
justinbieber & 0.88235 & 0.00763 & 0.01514 & 0.10 & 99.24 \\ \hline
ladygaga & 0.27027 & 0.00435 & 0.00856 & 1.17 & 99.57 \\ \hline
shakira & 0.54099 & 0.95644 & 0.69109 & 81.15 & 4.36 \\ \hline
taylorswift13 & 0.51259 & 0.92723 & 0.66020 & 88.17 & 7.28 \\ \hline
\textbf{Overall} & - & - & - & \textbf{61.68} & \textbf{55.82} \\ \hline
\end{tabular}
\caption{Αποτελέσματα Ανά Χρήστη με Threshold 0.05.}
\label{tab:threshold_0.05_results}
\end{table}

Όπως φαίνεται στον ~\ref{tab:threshold_0.05_results}, το σύστημα παρουσιάζει σημαντικές διαφοροποιήσεις στις επιδόσεις του μεταξύ των χρηστών. Οι διακυμάνσεις στις τιμές FAR και FRR είναι σημαντικές, υποδηλώνοντας την επίδραση της κατανομής δεδομένων και της ποικιλομορφίας στα χαρακτηριστικά γραφής του κάθε χρήστη. Στο~\autoref{fig:chapter5_image511} παρουσιάζονται τα παραπάνω αποτελέσματα.

\begin{figure}[H]
    \centering
    \includegraphics[width=\textwidth]{images/chapter5/5.1.1.png}
    \caption{FAR \& FRR ανά χρήστη για nu: 0.01, gamma: 0.05 και threshold: 0.05}
    \label{fig:chapter5_image511}
\end{figure}

\subsection{Παρατηρήσεις}
\begin{itemize}
    \item Τα καλύτερα αποτελέσματα για ισορροπία παρατηρούνται όταν $\nu$ είναι σχετικά μικρό ($0.005$–$0.01$), $\gamma$ είναι χαμηλό ($0.2$–$0.5$), και το threshold είναι προσεκτικά ρυθμισμένο κοντά στο $0.05$–$0.07$.
    \item Τα παραδείγματα του \autoref{tab:balanced_results} με \textbf{bold} υποδεικνύουν ιδανικές ισορροπίες για εφαρμογές όπου το FAR και το FRR πρέπει να είναι ισοδύναμα.
\end{itemize}

\section{Δεύτερη Φάση Πειραμάτων: Basic Majority Voting}
\label{sec:experiments_phase2}

\subsection{Πειραματική Διαδικασία}
Η πειραματική διαδικασία της δεύτερης φάσης επικεντρώνεται στη χρήση πολλαπλών μοντέλων για κάθε χρήστη και την εφαρμογή του απλού πλειοψηφικού μοντέλου (basic majority voting) όπως έχει παρουσιαστεί στο \autoref{sec:implementation_decision}. Παρακάτω περιγράφονται αναλυτικά τα βήματα της διαδικασίας.

\subsubsection{Διαδικασία Ελέγχου}
Για κάθε χρήστη του συστήματος, δημιουργούνται και εκπαιδεύονται πολλαπλά μοντέλα \emph{One-Class SVM} με διαφορετικούς συνδυασμούς υπερπαραμέτρων. Οι παράμετροι που χρησιμοποιούνται είναι:

\begin{itemize}
    \item \textbf{Νu} (\emph{nu}): Η υπερπαράμετρος \emph{nu} ελέγχει το ποσοστό των υποδειγμάτων που θεωρούνται outliers. Οι τιμές που εξετάζονται είναι:
    \[
    \nu \in \{0.001, 0.005, 0.01\}
    \]
    \item \textbf{Γάμμα} ($\gamma$): Η παράμετρος $\gamma$ επηρεάζει τη μορφή της συνάρτησης πυρήνα (\emph{kernel function}). Οι τιμές που δοκιμάζονται είναι:
    \[
    \gamma \in \{0.01, 0.05, 0.1\}
    \]
    \item \textbf{Κατώφλι Αποδοχής} (\emph{Acceptance Threshold}): Σύμφωνα με το \autoref{sec:experiments_phase1} καθορίζεται σε:
    \[
    \text{Threshold} = 0.05
    \]
\end{itemize}


\subsubsection{Καταγραφή και Ανάλυση Αποτελεσμάτων}
Για κάθε χρήστη, καταγράφονται:
\begin{itemize}
    \item Οι τιμές των μετρικών \emph{False Rejection Rate (FRR)} και \emph{False Acceptance Rate (FAR)}.
    \item Ο αριθμός από γνήσια και μη γνήσια prompts στον οποίο εφαρμόζεται ο έλεγχος.
\end{itemize}

\subsection{Αποτελέσματα Ανά Χρήστη από τη Συνάρτηση Basic Majority Voting}
\label{subsec:basic_majority_results}

Στον παρακάτω πίνακα παρουσιάζονται τα αποτελέσματα για διαφορετικούς χρήστες, όπως καταγράφηκαν κατά τη διάρκεια των πειραμάτων:

\begin{table}[H]
\centering
\begin{tabular}{|l|c|c|c|c|}
\hline
\textbf{User} & \textbf{FAR (\%)} & \textbf{FRR (\%)} & \textbf{Total Genuine Tests} & \textbf{Total Impostor Tests} \\ \hline
justinbieber & 92.54 & 4.75 & 295 & 295 \\ \hline
taylorswift13 & 88.78 & 4.95 & 303 & 303 \\ \hline
BarackObama & 33.49 & 3.02 & 430 & 430 \\ \hline
YouTube & 79.00 & 4.55 & 462 & 462 \\ \hline
ladygaga & 86.09 & 4.35 & 345 & 345 \\ \hline
TheEllenShow & 74.21 & 5.07 & 473 & 473 \\ \hline
Twitter & 74.13 & 4.65 & 344 & 344 \\ \hline
jtimberlake & 86.29 & 4.84 & 372 & 372 \\ \hline
britneyspears & 87.50 & 6.25 & 416 & 416 \\ \hline
Cristiano & 85.11 & 3.19 & 376 & 376 \\ \hline
cnnbrk & 25.27 & 3.97 & 277 & 277 \\ \hline
jimmyfallon & 84.43 & 6.61 & 469 & 469 \\ \hline
shakira & 76.78 & 3.96 & 379 & 379 \\ \hline
instagram & 59.43 & 4.65 & 387 & 387 \\ \hline
\end{tabular}
\caption{Αποτελέσματα από τη χρήση της Basic Majority Voting για διαφορετικούς χρήστες.}
\label{tab:tab_basic_majority_results}
\end{table}

\begin{figure}[H]
    \centering
    \includegraphics[width=\textwidth]{images/chapter5/5.2.png}
    \caption{FAR \& FRR ανά χρήστη με τη χρήση της Basic Majority Voting}
    \label{fig:chapter5_image52}
\end{figure}

\subsection{Παρατηρήσεις}
Η παραπάνω ανάλυση δείχνει σημαντική ποικιλομορφία στις επιδόσεις του συστήματος, κυρίως στη μετρική FAR, ανάλογα με τον χρήστη. Συγκεκριμένα:
\begin{itemize}
    \item Παρατηρείται μεγάλη ανισορροπία μεταξύ των τιμών FAR (πολύ υψηλές) και των τιμών FRR (ικανοποιητικά χαμηλές). Η μεταβλητότητα στις μετρικές δείχνει την ανάγκη για πιο εξελιγμένες τεχνικές λήψης απόφασης, όπως η \emph{Weighted Majority Voting}, που αναλύεται στα επόμενα υποκεφάλαια.
    \item Χρήστες όπως οι \texttt{justinbieber} και \texttt{taylorswift13} παρουσιάζουν πολύ υψηλά ποσοστά \emph{FAR}, κάτι που δείχνει αυξημένη πιθανότητα αποδοχής εισβολέων.
    \item Οι χρήστες \texttt{cnnbrk} και \texttt{BarackObama} έχουν εξαιρετικά χαμηλά ποσοστά \emph{FAR}, κάτι που υποδεικνύει ότι τα μοντέλα τους είναι πιο αυστηρά απέναντι σε εισβολείς.
\end{itemize}

Τα παραπάνω αποτελέσματα υποδηλώνουν την ανάγκη βελτιστοποίησης του συστήματος μέσω διαφορετικών τεχνικών.
\section{Τρίτη Φάση Πειραμάτων: Weighted Majority Voting}
\label{sec:experiments_phase3}

Η χρήση της μεθόδου \emph{Weighted Majority Voting} σε συνδυασμό με τη συνάρτηση \emph{Certainty Level} παρέχει μια πιο λεπτομερή και προσαρμοστική προσέγγιση στη διαδικασία λήψης αποφάσεων. Σε αυτό το υποκεφάλαιο παρουσιάζονται τα αποτελέσματα που προέκυψαν από τη χρήση των παραπάνω μεθόδων και η αξιολόγηση της απόδοσής τους σε σύγκριση με την \emph{Basic Majority Voting}.

\subsection{Πειραματική Διαδικασία}
\subsubsection{Διαδικασία Ελέγχου}
Με παρόμοιο τρόπο με νωρίτερα, για κάθε χρήστη του συστήματος, δημιουργούνται και εκπαιδεύονται πολλαπλά μοντέλα \emph{One-Class SVM} με διαφορετικούς συνδυασμούς υπερπαραμέτρων. Οι παράμετροι που χρησιμοποιούνται καθ' όλη τη διάρκεια της διαδικασίας είναι:

\begin{itemize}
    \item \textbf{Νu} (\emph{nu}): Οι τιμές που εξετάζονται είναι:
    \[
    \nu \in \{0.001, 0.005, 0.01, 0.05, 0.13, 0.25\}
    \]
    \item \textbf{Γάμμα} ($\gamma$): Οι τιμές που εξετάζονται είναι:
    \[
    \gamma \in \{0.00005, 0.0009, 0.0005, 0.001, 0.005, 0.01, 0.05, 0.07, 0.1, 0.15, 0.2, 0.3, 0.5\}
    \]
\end{itemize}

\subsubsection{Καταγραφή και Ανάλυση Αποτελεσμάτων}
Τα αποτελέσματα που προέκυψαν από τη μέθοδο \emph{Weighted Majority Voting} με τη χρήση της συνάρτησης \emph{Certainty Level} παρουσιάζονται στους Πίνακες ~\ref{tab:results_weighted_voting}, ~\ref{tab:second_grid_results}, ~\ref{tab:third_grid_results}. Οι μετρικές που αξιολογούνται περιλαμβάνουν:
\begin{itemize}
    \item \emph{False Acceptance Rate (FAR)}: Ποσοστό προτροπών εισβολέων που λανθασμένα χαρακτηρίστηκαν ως γνήσιες.
    \item \emph{False Rejection Rate (FRR)}: Ποσοστό γνήσιων προτροπών που λανθασμένα απορρίφθηκαν.
    \item Μέσος αριθμός μη γνήσιων προτροπών που γίνονται αποδεκτές πριν από το κλείδωμα.
\end{itemize}

\subsection{Αποτελέσματα}
\begin{table}[H]
\centering
\begin{tabular}{|l|c|c|c|}
\hline
\textbf{User} & \textbf{FAR (\%)} & \textbf{FRR (\%)} & \textbf{Mean Accepted Prompts by Impostor} \\ \hline
justinbieber & 88.81 & 7.80 & $\infty$ \\ \hline
taylorswift13 & 86.14 & 6.60 & $\infty$ \\ \hline
\textbf{BarackObama} & \textbf{28.60} & \textbf{3.49} & \textbf{3.77} \\ \hline
YouTube & 73.59 & 5.41 & $\infty$ \\ \hline
ladygaga & 80.29 & 6.38 & $\infty$ \\ \hline
TheEllenShow & 67.44 & 7.82 & $\infty$ \\ \hline
Twitter & 68.60 & 7.56 & $\infty$ \\ \hline
jtimberlake & 80.11 & 9.41 & $\infty$ \\ \hline
britneyspears & 83.89 & 8.17 & $\infty$ \\ \hline
Cristiano & 78.72 & 9.31 & $\infty$ \\ \hline
\textbf{cnnbrk} & \textbf{20.58} & \textbf{5.42} & \textbf{2.00} \\ \hline
jimmyfallon & 78.25 & 9.17 & $\infty$ \\ \hline
shakira & 71.24 & 6.86 & $\infty$ \\ \hline
\textbf{instagram} & \textbf{52.20} & \textbf{7.24} & \textbf{29.00} \\ \hline
\end{tabular}
\caption{Αποτελέσματα για $\nu \in \{0.001, 0.005\}$ και $\gamma \in \{0.05, 0.1\}$ που δείχνουν τις μετρικές FAR, FRR και τον μέσο αριθμό αποδεκτών προτροπών από εισβολείς.}
\label{tab:results_weighted_voting}
\end{table}


\begin{table}[H]
\centering
\begin{tabular}{|l|c|c|c|}
\hline
\textbf{User} & \textbf{FAR (\%)} & \textbf{FRR (\%)} & \textbf{Mean Accepted Prompts by Impostor} \\ \hline
justinbieber & 57.63 & 19.66 & $\infty$ \\ \hline
taylorswift13 & 65.68 & 29.04 & $\infty$ \\ \hline
\textbf{BarackObama} & \textbf{11.63} & \textbf{27.21} & \textbf{0.87} \\ \hline
YouTube & 39.39 & 30.95 & 10.31 \\ \hline
ladygaga & 51.01 & 30.14 & $\infty$ \\ \hline
\textbf{TheEllenShow} & \textbf{32.98} & \textbf{29.39} & \textbf{5.74} \\ \hline
Twitter & 41.57 & 29.94 & 11.00 \\ \hline
jtimberlake & 52.96 & 34.68 & 16.50 \\ \hline
britneyspears & 53.12 & 28.61 & $\infty$ \\ \hline
Cristiano & 49.20 & 30.85 & 6.00 \\ \hline
\textbf{cnnbrk} & \textbf{4.69} & \textbf{40.43} & \textbf{0.29} \\ \hline
jimmyfallon & 42.43 & 33.90 & 16.00 \\ \hline
shakira & 33.77 & 32.19 & 5.62 \\ \hline
\textbf{instagram} & \textbf{23.00} & \textbf{29.46} & \textbf{2.50} \\ \hline
\end{tabular}
\caption{Αποτελέσματα για $\nu \in \{0.0001, 0.0005, 0.001\}$ και $\gamma \in \{0.05, 0.1, 0.5\}$ που δείχνουν τις μετρικές FAR, FRR και τον μέσο αριθμό αποδεκτών προτροπών από εισβολείς.}
\label{tab:second_grid_results}
\end{table}


\begin{table}[H]
\centering
\begin{tabular}{|l|c|c|c|}
\hline
\textbf{User} & \textbf{FAR (\%)} & \textbf{FRR (\%)} & \textbf{Mean Accepted Prompts by Impostor} \\ \hline
justinbieber & 62.37 & 21.69 & 23.00 \\ \hline
taylorswift13 & 65.02 & 25.74 & 28.14 \\ \hline
\textbf{BarackObama} & \textbf{9.53} & \textbf{29.07} & \textbf{0.34} \\ \hline
YouTube & 46.54 & 27.92 & 4.89 \\ \hline
ladygaga & 51.59 & 27.54 & 6.85 \\ \hline
TheEllenShow & 38.69 & 26.22 & 2.86 \\ \hline
Twitter & 41.86 & 32.56 & 3.51 \\ \hline
jtimberlake & 52.42 & 33.60 & 7.80 \\ \hline
britneyspears & 56.25 & 26.92 & 11.14 \\ \hline
Cristiano & 51.06 & 29.79 & 7.11 \\ \hline
\textbf{cnnbrk} & \textbf{5.42} & \textbf{37.91} & \textbf{0.18} \\ \hline
jimmyfallon & 43.71 & 30.92 & 4.02 \\ \hline
shakira & 34.04 & 32.19 & 2.22 \\ \hline
instagram & 23.77 & 31.27 & 1.15 \\ \hline
\end{tabular}
\caption{Αποτελέσματα για $\nu \in \{0.001, 0.005, 0.01\}$ και $\gamma \in \{0.05, 0.07, 0.1, 0.2, 0.5\}$: FAR, FRR και Μέσος Αριθμός Αποδεκτών Προτροπών από Εισβολείς.}
\label{tab:third_grid_results}
\end{table}


\begin{figure}[H]
    \centering
    \includegraphics[width=\textwidth]{images/chapter5/5.3FAR.png}
    \caption{FAR ανά χρήστη για διαφορετικά πλέγματα παραμέτρων}
    \label{fig:chapter5_image53FAR}
\end{figure}

\begin{figure}[H]
    \centering
    \includegraphics[width=\textwidth]{images/chapter5/5.3MAPBL.png}
    \caption{Mean Accepted Prompts Before Locking for Impostors ανά χρήστη για διαφορετικά πλέγματα παραμέτρων}
    \label{fig:chapter5_image53MAPBL}
\end{figure}

\subsection{Παρατηρήσεις}
Η ανάλυση των αποτελεσμάτων για τη λειτουργία της μεθόδου weighted majority voting σε συνδυασμό με τη certainty level function αποδεικνύει τη σταδιακή βελτίωση της απόδοσης σε σχέση με τις προηγούμενες φάσεις. Οι τιμές των μετρικών FAR (False Acceptance Rate) και FRR (False Rejection Rate) αναδεικνύουν τη δυνατότητα του συστήματος να επιτυγχάνει καλύτερη ισορροπία μεταξύ αποδοχής γνήσιων προτροπών και απόρριψης μη γνήσιων. Μάλιστα το FAR μειώθηκε από 80.31\% σε 49.11\%, παρότι το FRR αυξήθηκε από 4,54\% σε 24.00\%.

Συγκεκριμένα, στο τρίτο πλέγμα παραμέτρων ($\nu \in {0.001, 0.005, 0.01}$ και $\gamma \in {0.05, 0.07, 0.1, 0.2, 0.5}$), παρατηρούνται εντυπωσιακά αποτελέσματα, όπως για τον χρήστη \textit{cnnbrk}, όπου η τιμή FAR μειώθηκε στο εξαιρετικά χαμηλό επίπεδο του 5.42\%, με μέση αποδοχή 0.18 prompts από impostors, ενώ το FRR παραμένει σχετικά υψηλό στο 37.91\%. Αυτό υποδεικνύει τη δυνατότητα του συστήματος να εντοπίζει με ακρίβεια μη έγκυρες εισόδους.

Η προσαρμογή των παραμέτρων $\nu$ και $\gamma$ φαίνεται να επηρεάζει άμεσα την απόδοση του συστήματος. Χαμηλότερες τιμές $\nu$ οδηγούν σε χαμηλότερο FAR, ενώ υψηλότερες τιμές $\gamma$ συμβάλλουν στη μείωση του FRR, με προφανή αντίκτυπο στην ακρίβεια και την ευαισθησία του συστήματος. Συνολικά, τα αποτελέσματα αναδεικνύουν την πρόοδο που έχει επιτευχθεί, με την ενσωμάτωση της weighted majority voting, και την επίτευξη βέλτιστων συνδυασμών παραμέτρων που εξισορροπούν τις ανάγκες ακρίβειας και ευαισθησίας.

\section{Τέταρτη Φάση Πειραμάτων: Confidence Level Function}
\label{sec:experiments_phase4}

Η τέταρτη φάση των πειραμάτων επικεντρώνεται στην ενσωμάτωση της συναρτήσης \textit{confidence level}, η οποία αποσκοπεί στη βελτίωση της ακρίβειας και της απόδοσης του συστήματος λήψης αποφάσεων. Η προσέγγιση αυτή επεκτείνει τη μέθοδο της σταθμισμένης πλειοψηφίας, προσθέτοντας έναν επιπλέον δείκτη αξιοπιστίας στις αποφάσεις, βασιζόμενο σε ένα δυναμικά μεταβαλλόμενο επίπεδο εμπιστοσύνης του συστήματος.

\subsection{Πειραματική Διαδικασία}
Για την αξιολόγηση της ενσωμάτωσης της συναρτήσης \textit{confidence level}, ακολουθήθηκε η εξής διαδικασία:
\begin{itemize}
    \item Χρήση του πλέγματος υπερπαραμέτρων:
    \[
    \nu \in \{0.001, 0.005, 0.01\}, \quad \gamma \in \{0.05, 0.07, 0.1, 0.2, 0.5\}
    \]
    \item Εκτέλεση της διαδικασίας ελέγχου για κάθε συνδυασμό υπερπαραμέτρων και καταγραφή των παρακάτω μετρικών:
    \begin{itemize}
        \item \textbf{MAPBL-G}: Μέσος αριθμός αποδεκτών προτροπών από γνήσιους χρήστες πριν την ενεργοποίηση του μηχανισμού κλειδώματος.
        \item \textbf{MAPBL-I}: Μέσος αριθμός αποδεκτών προτροπών από μη γνήσιους χρήστες (impostors) πριν την ενεργοποίηση του μηχανισμού κλειδώματος.
    \end{itemize}
    \item Αξιολόγηση της ακρίβειας και της αξιοπιστίας του συστήματος με βάση τα παραπάνω δεδομένα.
\end{itemize}

\subsection{Αποτελέσματα}
Τα αποτελέσματα της πειραματικής διαδικασίας συνοψίζονται στους παρακάτω πίνακες.

Αρχικά βλέπουμε τα αποτελέσματα των περιαμάτων για κάθε χρήστη μεμονωμένα για μια συγκεκριμένη confidence function.
\begin{table}[H]
\centering
\begin{tabular}{|c|c|c|c|c|}
\hline
\textbf{User} & \textbf{FAR (\%)} & \textbf{FRR (\%)} & \textbf{MAPBL-Genuine} & \textbf{MAPBL-Impostor} \\ \hline
justinbieber & 62.37 & 21.69 & 64.00 & 23.00 \\ \hline
taylorswift13 & 65.02 & 25.74 & 78.00 & 28.14 \\ \hline
\textbf{BarackObama} & \textbf{9.53} & \textbf{29.07} & \textbf{125.00} & \textbf{0.34} \\ \hline
YouTube & 46.54 & 27.92 & 64.50 & 4.89 \\ \hline
ladygaga & 51.59 & 27.54 & 95.00 & 6.85 \\ \hline
\textbf{TheEllenShow} & \textbf{38.69} & \textbf{26.22} & \textbf{124.00} & \textbf{2.86} \\ \hline
Twitter & 41.86 & 32.56 & 56.00 & 3.51 \\ \hline
jtimberlake & 52.42 & 33.60 & 20.83 & 7.80 \\ \hline
britneyspears & 56.25 & 26.92 & 112.00 & 11.14 \\ \hline
Cristiano & 51.06 & 29.79 & 56.00 & 7.11 \\ \hline
\textbf{cnnbrk} & \textbf{5.42} & \textbf{37.91} & \textbf{13.12} & \textbf{0.18} \\ \hline
jimmyfallon & 43.71 & 30.92 & 145.00 & 4.02 \\ \hline
shakira & 34.04 & 32.19 & 61.00 & 2.22 \\ \hline
instagram & 23.77 & 31.27 & 60.50 & 1.15 \\ \hline
\end{tabular}
\caption{Αποτελέσματα πειραμάτων για την ενσωμάτωση της συνάρτησης \textit{confidence level}.}
\label{tab:confidence_level_results}
\end{table}

Αλλάζοντας τις τιμές των μεταβλητών της συνάρτησης \textit{confidence function}, μπορούμε να προσαρμόσουμε την αυστηρότητα της συνάρτησης και τις μετρικές MAPBL-G \& MAPBL-I. Ακολουθούν πίνακες των μετρικών με διαφορετικές τιμές παραμέτρων της confidence level:

\begin{itemize}
    \item Στον~\autoref{tab:chapter5_primaryResults}, οι τιμές των παραμέτρων είναι:
    \begin{itemize}
        \item Base Increase: $0.09$
        \item Base Decrease: $0.08$
        \item High-Certainty Boost factor: $0.4$
        \item Confidence Threshold: $0.3$
    \end{itemize}

    \item Στον~\autoref{tab:chapter5_tighten}, οι τιμές των παραμέτρων είναι:
    \begin{itemize}
        \item Base Increase: $0.07$
        \item Base Decrease: $0.1$
        \item High-Certainty Boost factor: $0.5$
        \item Confidence Threshold: $0.4$
    \end{itemize}

    \item Στον~\autoref{tab:chapter5_boostValues}, οι τιμές των παραμέτρων είναι:
    \begin{itemize}
        \item Base Increase: $0.07$
        \item Base Decrease: $0.1$
        \item High-Certainty Boost factor: $1.00$
        \item Confidence Threshold: $0.4$
    \end{itemize}

     \item Στον~\autoref{tab:chapter5_relaxationBases}, οι τιμές των παραμέτρων είναι:
    \begin{itemize}
        \item Base Increase: $0.1$
        \item Base Decrease: $0.06$
        \item High-Certainty Boost factor: $0.5$
        \item Confidence Threshold: $0.4$
    \end{itemize}

     \item Στον~\autoref{tab:chapter5_relaxationAll}, οι τιμές των παραμέτρων είναι:
    \begin{itemize}
        \item Base Increase: $0.12$
        \item Base Decrease: $0.09$
        \item High-Certainty Boost factor: $0.4$
        \item Confidence Threshold: $0.3$
    \end{itemize}
\end{itemize}

\begin{table}[H]
\centering
\begin{tabular}{|l|c|}
\hline
\textbf{Μετρική}                                 & \textbf{Τιμή}   \\ \hline
Total Users Tested                               & 14             \\ \hline
Mean FRR (\%)                                    & 31.94          \\ \hline
Mean FAR (\%)                                    & 42.55          \\ \hline
Mean Genuine Rejected Prompts Before Lock       & 75.41          \\ \hline
Mean Impostor Accepted Prompts Before Lock      & 6.90           \\ \hline
\end{tabular}
\caption{Primary Results}
\label{tab:chapter5_primaryResults}
\end{table}

\begin{table}[H]
\centering
\begin{tabular}{|l|c|}
\hline
\textbf{Μετρική}                                 & \textbf{Τιμή}   \\ \hline
Total Users Tested                               & 14             \\ \hline
Mean FRR (\%)                                    & 39.48          \\ \hline
Mean FAR (\%)                                    & 34.40          \\ \hline
Mean Genuine Rejected Prompts Before Lock       & 18.55          \\ \hline
Mean Impostor Accepted Prompts Before Lock      & 3.03           \\ \hline
\end{tabular}
\caption{Tighten Variables' Values}
\label{tab:chapter5_tighten}
\end{table}

\begin{table}[H]
\centering
\begin{tabular}{|l|c|}
\hline
\textbf{Μετρική}                                 & \textbf{Τιμή}   \\ \hline
Total Users Tested                               & 14             \\ \hline
Mean FRR (\%)                                    & 48.57          \\ \hline
Mean FAR (\%)                                    & 26.87          \\ \hline
Mean Genuine Rejected Prompts Before Lock       & 7.35           \\ \hline
Mean Impostor Accepted Prompts Before Lock      & 1.70           \\ \hline
\end{tabular}
\caption{Maximized boost value in confidence function}
\label{tab:chapter5_boostValues}
\end{table}

\begin{table}[H]
\centering
\begin{tabular}{|l|c|}
\hline
\textbf{Μετρική}                                 & \textbf{Τιμή}   \\ \hline
Total Users Tested                               & 14             \\ \hline
Mean FRR (\%)                                    & 31.12          \\ \hline
Mean FAR (\%)                                    & 43.74          \\ \hline
Mean Genuine Rejected Prompts Before Lock       & 49.87          \\ \hline
Mean Impostor Accepted Prompts Before Lock      & 3.66           \\ \hline
\end{tabular}
\caption{Relaxation of base increase and base decrease variables' values}
\label{tab:chapter5_relaxationBases}
\end{table}

\begin{table}[H]
\centering
\begin{tabular}{|l|c|}
\hline
\textbf{Μετρική}                                 & \textbf{Τιμή}   \\ \hline
Total Users Tested                               & 14             \\ \hline
Mean FRR (\%)                                    & 31.12          \\ \hline
Mean FAR (\%)                                    & 43.74          \\ \hline
Mean Genuine Rejected Prompts Before Lock       & 123.25         \\ \hline
Mean Impostor Accepted Prompts Before Lock      & 6.85           \\ \hline
\end{tabular}
\caption{Relaxation of all values}
\label{tab:chapter5_relaxationAll}
\end{table}

\begin{figure}[H]
    \centering
    \includegraphics[width=\textwidth]{images/chapter5/5.4variablesValues.png}
    \caption{FAR, FRR, MAPBL-G, MAPBL-I για διαφορετικές τιμές των μεταβλητών της συνάρτησης Confidence Level}
    \label{fig:chapter5_image54variablesValues}
\end{figure}

Ενδιαφέρουσα είναι και η περίπτωση αλλαγής της υπερπαραμέτρου \textit{gamma} σε συνάρτηση με τις μετρικές που μελετάμε. Παρακάτω ακολουθούν κάποια αποτελέσματα από τέτοιες περιπτώσεις.

% Πίνακας 7: Αλλαγή στις Τιμές Gamma (Όλες οι Τιμές)
\begin{table}[H]
\centering
\begin{tabular}{|l|c|}
\hline
\textbf{Μετρική}                                 & \textbf{Τιμή}   \\ \hline
Total Users Tested                               & 14             \\ \hline
Mean FRR (\%)                                    & 31.12          \\ \hline
Mean FAR (\%)                                    & 43.74          \\ \hline
Mean Genuine Rejected Prompts Before Lock       & 72.38          \\ \hline
Mean Impostor Accepted Prompts Before Lock      & 6.76           \\ \hline
\end{tabular}
\caption{All gamma values}
\end{table}

% Πίνακας 8: Conf3.py με 0.3 στο gamma grid
\begin{table}[H]
\centering
\begin{tabular}{|l|c|}
\hline
\textbf{Μετρική}                                 & \textbf{Τιμή}   \\ \hline
Total Users Tested                               & 14             \\ \hline
Mean FRR (\%)                                    & 37.42          \\ \hline
Mean FAR (\%)                                    & 36.83          \\ \hline
Mean Genuine Rejected Prompts Before Lock       & 40.25          \\ \hline
Mean Impostor Accepted Prompts Before Lock      & 2.61           \\ \hline
\end{tabular}
\caption{Include 0.3 gamma value on the grid}
\end{table}

% Πίνακας 9: Conf3.py με 0.15 και 0.3 στο gamma grid
\begin{table}[H]
\centering
\begin{tabular}{|l|c|}
\hline
\textbf{Μετρική}                                 & \textbf{Τιμή}   \\ \hline
Total Users Tested                               & 14             \\ \hline
Mean FRR (\%)                                    & 35.87          \\ \hline
Mean FAR (\%)                                    & 38.40          \\ \hline
Mean Genuine Rejected Prompts Before Lock       & 42.22          \\ \hline
Mean Impostor Accepted Prompts Before Lock      & 3.03           \\ \hline
\end{tabular}
\caption{Include 0.15 and 0.3 on the gamma grid}
\end{table}

% Πίνακας 10: Conf3.py με 0.15 στο gamma grid
\begin{table}[H]
\centering
\begin{tabular}{|l|c|}
\hline
\textbf{Μετρική}                                 & \textbf{Τιμή}   \\ \hline
Total Users Tested                               & 14             \\ \hline
Mean FRR (\%)                                    & 30.20          \\ \hline
Mean FAR (\%)                                    & 44.29          \\ \hline
Mean Genuine Rejected Prompts Before Lock       & 86.85          \\ \hline
Mean Impostor Accepted Prompts Before Lock      & 6.08           \\ \hline
\end{tabular}
\caption{Include 0.15 on the gamma grid}
\end{table}

% Πίνακας 11: Gamma χωρίς 0.5
\begin{table}[H]
\centering
\begin{tabular}{|l|c|}
\hline
\textbf{Μετρική}                                 & \textbf{Τιμή}   \\ \hline
Total Users Tested                               & 14             \\ \hline
Mean FRR (\%)                                    & 17.80          \\ \hline
Mean FAR (\%)                                    & 59.52          \\ \hline
Mean Genuine Rejected Prompts Before Lock       & 70.62          \\ \hline
Mean Impostor Accepted Prompts Before Lock      & 68.00          \\ \hline
\end{tabular}
\caption{Remove 0.5 from the gamma grid}
\end{table}

\begin{figure}[H]
    \centering
    \includegraphics[width=\textwidth]{images/chapter5/5.4gamma.png}
    \caption{FAR, FRR, MAPBL-G, MAPBL-I για διαφορετικές τιμές της υπερπαραμέτρου gamma}
    \label{fig:chapter5_image54gamma}
\end{figure}

\subsection{Παρατηρήσεις}
Η ενσωμάτωση της συναρτήσης \textit{confidence level} προσέφερε ποσοτικά σημαντικά οφέλη στις επιδόσεις του συστήματος:
\begin{itemize}
    \item \textbf{Μείωση FAR:} Η μέση τιμή του FAR μειώθηκε από 42.55\% σε 26.87\% κατά τις πειραματικές δοκιμές, αποδεικνύοντας ότι το σύστημα μπορεί να αποφεύγει με μεγαλύτερη ακρίβεια τις λανθασμένες αποδοχές μη γνήσιων χρηστών.
    \item \textbf{Ισορροπία μεταξύ FRR και MAPBL:} Το FRR παρέμεινε σε διαχειρίσιμα επίπεδα με μέση τιμή 31.12\%, ενώ οι μέσοι αριθμοί αποδεκτών προτροπών από impostors (\textbf{MAPBL-I}) μειώθηκαν σημαντικά, επιτυγχάνοντας καλύτερη ισορροπία.
\end{itemize}

Η παραπάνω προσέγγιση επιβεβαιώνει τη σημασία της χρήσης μιας δυναμικής μεθόδου όπως το \textit{confidence level}.


\section{Πέμπτη Φάση Πειραμάτων: LOSO Cross Validation}
\label{sec:experiments_phase5}

Η πέμπτη φάση αφορά την τεχνική Leave-One-Subject-Out Cross Validation, η οποία στοχεύει στην ανάδειξη της δυνατότητας γενίκευσης του μοντέλου.

Η τεχνική Leave-One-Subject-Out Cross-Validation (LOSO-CV) αποτελεί μια εξειδικευμένη προσέγγιση του ευρύτερου πλαισίου της διαδικασίας \textbf{Cross-Validation}. Το Cross-Validation χρησιμοποιείται συχνά στη μηχανική μάθηση για την αξιολόγηση της δυνατότητας γενίκευσης ενός μοντέλου. Σκοπός του είναι να επιβεβαιώσει πως τα αποτελέσματα που παράγονται από ένα μοντέλο δεν είναι προσαρμοσμένα αποκλειστικά στο σύνολο των δεδομένων εκπαίδευσης αλλά μπορούν να γενικευτούν και σε άγνωστα δεδομένα.

Κατά τη διαδικασία Cross-Validation, το σύνολο δεδομένων διαχωρίζεται σε υποκατηγορίες (\emph{folds}). Ένα από τα \emph{folds} χρησιμοποιείται για την αξιολόγηση (\emph{validation set}), ενώ τα υπόλοιπα χρησιμοποιούνται για την εκπαίδευση (\emph{training set}). Αυτή η διαδικασία επαναλαμβάνεται ώσπου κάθε \emph{fold} να έχει χρησιμοποιηθεί ως σύνολο αξιολόγησης. 

Συγκεκριμένα στη τεχνική LOSO, αντί για τυχαία \emph{folds}, κάθε επανάληψη του Cross-Validation επικεντρώνεται στον διαχωρισμό όλων των δεδομένων ενός συγκεκριμένου χρήστη από όλων των υπολοίπων, δηλαδή ολόκληρο το προφίλ ενός χρήστη αποκλείεται κατά τη διάρκεια της εκπαίδευσης και χρησιμοποιείται αποκλειστικά για την αξιολόγηση. Η προσέγγιση αυτή βρίσκει εφαρμογή και στην ανάλυση βιομετρικών δεδομένων, όπου η γενίκευση σε άγνωστα προφίλ χρηστών αποτελεί κριτήριο για την αποδοτικότητα του συστήματος.

\subsubsection{Περιγραφή της Διαδικασίας LOSO}
Η διαδικασία LOSO λειτουργεί ως εξής:
\begin{enumerate}
    \item Το σύνολο δεδομένων χωρίζεται με βάση τον χρήστη. Κάθε χρήστης θεωρείται μία μοναδική κατηγορία δεδομένων.
    \item Σε κάθε επανάληψη, τα δεδομένα ενός χρήστη (\emph{testing subject}) αφαιρούνται πλήρως από το σύνολο εκπαίδευσης και χρησιμοποιούνται αποκλειστικά για την αξιολόγηση.
    \item Τα υπόλοιπα δεδομένα των υπόλοιπων χρηστών (\emph{training subjects}) χρησιμοποιούνται για την εκπαίδευση των μοντέλων.
    \item Μετά την εκπαίδευση, τα μοντέλα αξιολογούνται στα δεδομένα του αποκλεισμένου χρήστη.
    \item Η διαδικασία επαναλαμβάνεται για κάθε χρήστη, διασφαλίζοντας ότι όλοι οι χρήστες έχουν χρησιμοποιηθεί μία φορά ως \emph{testing subjects}.
\end{enumerate}

Στο ~\autoref{fig:LOSOcv} φαίνεται η διαδικασία της τεχνικής LOSO για πολυδιάστατα δεδομένα (π.χ. δεδομένα από πολλαπλές πηγές ή συσκευές), ενώ στο ~\autoref{fig:exampleLOSOcv} παρουσιάζεται η βασική αρχή αυτής της τεχνικής, όπου κάθε χρήστης αποκλείεται διαδοχικά για να χρησιμοποιηθούν τα δεδομένα του ως (\emph{testing subject}).

\begin{figure}[H]
    \centering
    \begin{subfigure}{0.45\textwidth}
        \centering
        \includegraphics[width=0.8\textwidth]{images/chapter4/LOSOcv.jpg}
        \caption{Επεξήγηση της διαδικασίας LOSO για δεδομένα από διαφορετικές πηγές. Η κάθε πηγή δεδομένων αποκλείεται διαδοχικά και τα υπόλοιπα δεδομένα χρησιμοποιούνται για εκπαίδευση.}
        \label{fig:LOSOcv}
    \end{subfigure}
    \hfill
    \begin{subfigure}{0.45\textwidth}
        \centering
        \includegraphics[width=0.8\textwidth]{images/chapter4/exampleLOSOcv.png}
        \caption{Παραδείγματα του τρόπου λειτουργίας της τεχνικής LOSO. Κάθε χρήστης αποκλείεται διαδοχικά από το σύνολο εκπαίδευσης και χρησιμοποιείται ως σύνολο αξιολόγησης.}
        \label{fig:exampleLOSOcv}
    \end{subfigure}
    \caption{Γραφήματα επεξήγησης τεχνικής LOSO-CV}
    \label{fig:subgraphs}
\end{figure}

\paragraph{Υπολογισμένες Μετρικές}
Οι παρακάτω μετρικές χρησιμοποιούνται για την αξιολόγηση της απόδοσης της τεχνικής LOSO-CV:
\begin{itemize}
    \item \textbf{Accuracy:}
    Μετρά το ποσοστό των σωστών προβλέψεων, είτε για γνήσιους χρήστες είτε για εισβολείς, σε σχέση με το σύνολο των προβλέψεων, όπως αναλύεται στην~\autoref{subsec:metrics}.

    \item \textbf{MAE:}
    Αντιπροσωπεύει την ικανότητα του συστήματος να μειώνει τα σφάλματα πρόβλεψης, διατηρώντας την ακρίβεια στις εκτιμήσεις του. Ο υπολογισμός του αναλύεται στην~\autoref{subsec:metrics}.  
\end{itemize}

\subsection{Αποτελέσματα}
\label{sec:results_phase5}

Η τεχνική \textit{Leave-One-Subject-Out Cross Validation} αξιολογήθηκε με δύο μετρικές: την \textbf{Accuracy (Ακρίβεια)} και το \textbf{Mean Absolute Error (MAE)}. Οι μετρικές αυτές προσφέρουν μια ξεκάθαρη εικόνα για την απόδοση του συστήματος αυθεντικοποίησης σε επίπεδο χρήστη. Παρακάτω παρουσιάζονται τα αποτελέσματα:

\subsubsection{Ακρίβεια (Accuracy)} Η ακρίβεια δείχνει το ποσοστό των συνολικών προτροπών που ταξινομήθηκαν σωστά από το σύστημα. Υψηλότερες τιμές αντιπροσωπεύουν καλύτερη απόδοση στην αναγνώριση των γνήσιων χρηστών και στον αποκλεισμό των εισβολέων. Τα αποτελέσματα της ακρίβειας για κάθε χρήστη φαίνονται στο~\autoref{fig:loso_accuracy}.

\begin{figure}[H]
    \centering
    \includegraphics[width=\textwidth]{images/chapter5/losoAccu.png}
    \caption{Αποτελέσματα ακρίβειας (Accuracy) ανά χρήστη για την τεχνική LOSO.}
    \label{fig:loso_accuracy}
\end{figure}

\subsubsection{Μέσο Απόλυτο Σφάλμα (Mean Absolute Error - MAE)} Το MAE μετρά το μέσο σφάλμα μεταξύ των προβλεπόμενων και των πραγματικών τιμών και παρέχει μια ένδειξη για το επίπεδο αστοχίας της πρόβλεψης. Χαμηλότερες τιμές MAE αντιπροσωπεύουν καλύτερη απόδοση του συστήματος. Τα αποτελέσματα του MAE για κάθε χρήστη παρουσιάζονται στο~\autoref{fig:loso_mae}.

\begin{figure}[H]
    \centering
    \includegraphics[width=\textwidth]{images/chapter5/losoMae.png}
    \caption{Αποτελέσματα μέσου απόλυτου σφάλματος (MAE) ανά χρήστη για την τεχνική LOSO.}
    \label{fig:loso_mae}
\end{figure}

Επίσης, στο~\autoref{fig:loso_accu_mae} παρουσιάζονται οι μέσοι όροι των μετρικών Accuracy και ΜΑΕ. 

\begin{figure}[H]
    \centering
    \includegraphics[width=\textwidth]{images/chapter5/losoAccuMae.png}
    \caption{Μέσος όρος Accuracy \& MAE.}
    \label{fig:loso_accu_mae}
\end{figure}


%\newpage
\section{Συγκριτικά Αποτελέσματα}
\label{sec:experiments_comparative}

Η τέταρτη φάση των πειραμάτων έδειξε σημαντικές βελτιώσεις αλλά και προκλήσεις στη χρήση της συνάρτησης \textit{confidence level}. Παρακάτω παρουσιάζονται ποσοτικοποιημένες παρατηρήσεις βασισμένες στα πειραματικά δεδομένα:

\begin{itemize}
    \item \textbf{Μείωση του FAR και διατήρηση του FRR:}
    \begin{itemize}
        \item Στο βασικό πλέγμα υπερπαραμέτρων 
        \[
        \nu \in \{0.001, 0.005, 0.01\}, \quad \gamma \in \{0.05, 0.07, 0.1, 0.2, 0.5\}
        \]
        παρατηρήθηκε \textbf{μέσο FAR 42.55\%} και \textbf{μέσο FRR 31.94\%}.
        \item Όταν εισήχθησαν χαμηλότερες τιμές \textit{gamma} (\(0.15\)), το \textbf{FAR αυξήθηκε σε 44.29\%}, ενώ το \textbf{FRR μειώθηκε σε 30.20\%}, δείχνοντας βελτίωση ευαισθησίας για γνήσιους χρήστες, αλλά αύξηση της πιθανότητας αποδοχής impostors.
        \item Με την αφαίρεση της τιμής \(\gamma = 0.5\), το \textbf{FAR μειώθηκε σε 59.52\%} και το \textbf{FRR μειώθηκε σε 17.80\%}, καταδεικνύοντας ότι η αφαίρεση μεγάλων τιμών \textit{gamma} μειώνει την αυστηρότητα χωρίς αύξηση των ψευδών αποδοχών.
    \end{itemize}

    \item \textbf{Βελτίωση MAPBL-G και MAPBL-I:}
    \begin{itemize}
        \item Η χρήση των παραμέτρων \textit{base increase} 0.09 και \textit{base decrease} 0.08 βελτίωσε τη σταθερότητα του συστήματος. Το \textbf{MAPBL-G} μειώθηκε από \textbf{75.41} στο βασικό πλέγμα σε \textbf{49.87}, ενώ το \textbf{MAPBL-I} παρέμεινε σταθερό γύρω στο \textbf{6.85}.
        \item Όταν χρησιμοποιήθηκαν πιο αυστηρές ρυθμίσεις (\textit{confidence threshold} 0.3), το \textbf{MAPBL-G} μειώθηκε δραματικά σε \textbf{7.35}, ενώ το \textbf{MAPBL-I} έπεσε σε μόλις \textbf{1.70}, γεγονός που υποδηλώνει αυξημένη αυστηρότητα έναντι impostors, εις βάρος των γνήσιων χρηστών.
    \end{itemize}

    \item \textbf{Σύγκριση ρυθμίσεων γ-τιμών:}
    \begin{itemize}
        \item Όταν χρησιμοποιήθηκαν οι τιμές \textit{gamma} \(0.3\) και \(0.15\) ταυτόχρονα, το \textbf{FAR μειώθηκε σε 38.40\%}, αλλά το \textbf{FRR αυξήθηκε σε 35.87\%}.
        \item Με μόνο την τιμή \(\gamma = 0.15\), το \textbf{FAR αυξήθηκε σε 44.29\%}, δείχνοντας ότι οι χαμηλές τιμές \textit{gamma} χωρίς συνδυασμό με άλλες παραμέτρους αυξάνουν τις ψευδείς αποδοχές.
        \item Στην περίπτωση που η τιμή \(\gamma = 0.3\) προστέθηκε στο πλέγμα, το \textbf{MAPBL-I μειώθηκε σε 2.61}, ενώ το \textbf{MAPBL-G} βελτιώθηκε σε \textbf{40.25}, γεγονός που υποδηλώνει καλύτερη απόδοση του συστήματος συνολικά.
    \end{itemize}

    \item \textbf{Ανάλυση ευαισθησίας συστήματος:}
    \begin{itemize}
        \item Με βάση τα διαφορετικά πλέγματα υπερπαραμέτρων, οι αυστηρότερες τιμές \textit{confidence threshold} μειώνουν δραστικά τις αποδεκτές προτροπές από impostors, όπως φάνηκε με \textbf{MAPBL-I 1.70} στην πιο αυστηρή ρύθμιση.
        \item Οι πιο χαλαρές ρυθμίσεις (\textit{confidence boost} +0.03) αύξησαν το \textbf{MAPBL-G} σε \textbf{123.25}, ενώ διατήρησαν το \textbf{MAPBL-I} κοντά στο \textbf{6.85}, ενισχύοντας την αποδοτικότητα για γνήσιους χρήστες.
    \end{itemize}
\end{itemize}

\begin{figure}[H]
    \centering
    \includegraphics[width=\textwidth]{images/chapter5/5.5compareChapters1-3.png}
    \caption{Σύγκριση FAR \& FRR στο \autoref{sec:experiments_phase1} και στο \autoref{sec:experiments_phase3}}
    \label{fig:chapter5_image55comparisonChapters13}
\end{figure}

\begin{figure}[H]
    \centering
    \includegraphics[width=\textwidth]{images/chapter5/55comparisonMAPBL.png}
    \caption{Σύγκριση Mean Accepted Prompts Before Locking for Impostors στο \autoref{sec:experiments_phase3} με το αρχικό πλέγμα υπερπαραμέτρων, στο \autoref{sec:experiments_phase3} με το δεύτερο πλέγμα υπερπαραμέτρων και στο στο \autoref{sec:experiments_phase4} με την ενσωμάτωση της συνάρτησης Confidence Level}
    \label{fig:chapter5_image55comparisonMAPBL}
\end{figure}

\textbf{Συμπέρασμα:} Η συνάρτηση \textit{confidence level} απέδειξε την ευελιξία της στη βελτίωση του συστήματος. Οι αλλαγές στις τιμές των παραμέτρων επηρεάζουν δραστικά το FAR και το FRR, ενώ παρέχουν δυνατότητα προσαρμογής στις απαιτήσεις ασφάλειας ή χρηστικότητας, καθιστώντας το σύστημα κατάλληλο για διάφορα σενάρια χρήσης.

