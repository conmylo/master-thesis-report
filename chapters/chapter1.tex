\chapter{Εισαγωγή}
\label{chapter:intro}

Η ασφάλεια στη σύγχρονη εποχή έχει καταστεί μείζον ζήτημα, καθώς οι ψηφιακές υπηρεσίες, οι εφαρμογές και οι συσκευές που χρησιμοποιούμε σε καθημερινή βάση αποθηκεύουν και επεξεργάζονται σημαντικό όγκο προσωπικών, επαγγελματικών και ευαίσθητων δεδομένων. Από τις τραπεζικές συναλλαγές μέχρι τις επικοινωνίες και από την εργασία μέχρι την ψυχαγωγία, οι άνθρωποι στηρίζονται σε ψηφιακά μέσα που απαιτούν ασφαλή πρόσβαση και προστασία από κακόβουλες επιθέσεις. Ο αυξανόμενος κίνδυνος παραβίασης δεδομένων, απάτης και ψηφιακής κατασκοπείας, έχει ενισχύσει την ανάγκη για ισχυρά και αξιόπιστα μέτρα ασφάλειας στον κυβερνοχώρο.

Μία από τις σημαντικότερες παραμέτρους της ασφάλειας των πληροφοριών είναι η αυθεντικοποίηση, δηλαδή η διαδικασία ταυτοποίησης ενός χρήστη πριν του επιτραπεί η πρόσβαση σε έναν πόρο ή σε μια υπηρεσία. Οι παραδοσιακές μέθοδοι αυθεντικοποίησης περιλαμβάνουν συνήθως τη χρήση κωδικών πρόσβασης, προσωπικών αναγνωριστικών (PIN) ή βιομετρικών δεδομένων, όπως δακτυλικά αποτυπώματα και αναγνώριση προσώπου. Παρόλο που αυτές οι μέθοδοι αποτελούν αναπόσπαστο μέρος της ψηφιακής ασφάλειας, παρουσιάζουν ορισμένα προβλήματα και περιορισμούς. Οι κωδικοί πρόσβασης, για παράδειγμα, μπορεί να είναι εύκολο να υποκλαπούν μέσω επιθέσεων phishing, ενώ οι χρήστες συχνά δυσκολεύονται να θυμούνται πολλούς και διαφορετικούς κωδικούς για κάθε πλατφόρμα. Επιπλέον, οι βιομετρικές μέθοδοι, παρότι είναι ισχυρότερες από τους κωδικούς πρόσβασης, ενδέχεται να μην είναι πάντοτε πρακτικές και δεν αναιρούν το γεγονός πως η εκάστοτε ψηφιακή υπηρεσία παραμένει εκτεθειμένη μετά από την πρώτη και μοναδική φορά αυθεντικοποίησης του χρήστη (one-time authentication).

Στις μέρες μας, μεγάλος όγκος πληροφορίας ανταλλάσσεται μέσω κειμένου, είτε με τη μορφή συνομιλιών (chats) και μηνυμάτων  ηλεκτρονικού ταχυδρομίου (emails), είτε μέσω αναρτήσεων σε μέσα κοινωνικής δικτύωσης. Αυτή η τάση φανερώνει τη σημασία της ανάλυσης του γραπτού λόγου στη κατανόηση της αλληλεπίδρασης μεταξύ χρηστών και συστημάτων. Ταυτόχρονα, οι επιθέσεις με τη χρήση bots, ψεύτικων προφίλ και τεχνικών όπως deepfake εντείνονται διαρκώς, αναδεικνύοντας την ανάγκη για πιο εξελιγμένα συστήματα αυθεντικοποίησης. Μια τέτοια νέα προσέγγιση θα πρέπει επομένως να ενσωματώνει γλωσσικά χαρακτηριστικά, ικανά να προσδιορίσουν τον χρήστη και να τον ταυτοποιήσουν με ασφάλεια.

Με τη ραγδαία ανάπτυξη της τεχνολογίας και την αυξημένη εξάρτηση από ψηφιακά μέσα, τα συστήματα αυθεντικοποίησης γίνονται ολοένα και πιο περίπλοκα. Οι απαιτήσεις ασφαλείας αυξάνονται, ενώ ταυτόχρονα υπάρχει η ανάγκη για μεθόδους που είναι εύχρηστες και δεν απαιτούν συνεχή παρέμβαση από τους χρήστες. Αυτή η ανάγκη καθοδηγεί την έρευνα για νέες, πιο δυναμικές και ευέλικτες μεθόδους ταυτοποίησης, οι οποίες θα μπορούν να προσαρμοστούν στις απαιτήσεις των σύγχρονων ψηφιακών εφαρμογών και στις αυξημένες απαιτήσεις ασφαλείας του σημερινού διαδικτυακού περιβάλλοντος.

Τα τελευταία χρόνια, η Μηχανική Μάθηση και η Επεξεργασία Φυσικής Γλώσσας (Natural Language Processing - NLP) έχουν αρχίσει να παίζουν καθοριστικό ρόλο στην ανάπτυξη συστημάτων ασφάλειας και ταυτοποίησης. Η Μηχανική Μάθηση επιτρέπει τη δημιουργία έξυπνων συστημάτων που μπορούν να εκπαιδεύονται και να αναγνωρίζουν πρότυπα και συμπεριφορές, ενώ οι τεχνικές NLP προσφέρουν την ικανότητα ανάλυσης και επεξεργασίας του γραπτού λόγου, δημιουργώντας νέες δυνατότητες για την ανίχνευση μοναδικών χαρακτηριστικών σε επίπεδο χρήστη. Αυτά τα εργαλεία επιτρέπουν τη δημιουργία εξατομικευμένων προφίλ, τα οποία βασίζονται σε χαρακτηριστικά όπως η γλωσσική συμπεριφορά και το στυλ γραφής του χρήστη.

Η ανάγκη για ισχυρά συστήματα ασφάλειας έχει δημιουργήσει τις προϋποθέσεις για νέες μορφές αυθεντικοποίησης, οι οποίες συνδυάζουν την ευελιξία και τη διακριτική λειτουργία με την ισχυρή ασφάλεια. Το προτεινόμενο σύστημα στην εργασία αυτή αξιοποιεί αυτές τις εξελίξεις στην τεχνολογία και ενσωματώνει μεθόδους που βασίζονται στη μηχανική μάθηση και την επεξεργασία φυσικής γλώσσας, ώστε να δημιουργήσει ένα ισχυρό πλαίσιο για την ταυτοποίηση των χρηστών.



\section{Περιγραφή του Προβλήματος}
\label{section:problem_description}

Η ανάγκη για αυξημένη ασφάλεια στις ψηφιακές επικοινωνίες και την προστασία δεδομένων έχει καταστεί επιτακτική, ιδίως λόγω των συνεχώς εξελισσόμενων απειλών και της ψηφιοποίησης κάθε πτυχής της ανθρώπινης δραστηριότητας. Οι σύγχρονες τεχνολογίες απαιτούν συστήματα ταυτοποίησης που να μπορούν να παρέχουν διαρκή ασφάλεια, χωρίς να παρεμβαίνουν στη ροή ενεργειών του χρήστη. Οι παραδοσιακές μέθοδοι αυθεντικοποίησης, όπως η χρήση κωδικών πρόσβασης, παρουσιάζουν σοβαρούς περιορισμούς. Από τη μία, η πολυπλοκότητα και η συχνή ανανέωση των κωδικών απαιτεί συνεχή προσοχή από τον χρήστη, ενώ από την άλλη οι μέθοδοι αυτές είναι ευάλωτες σε επιθέσεις, όπως το phishing ή το brute-force.

Καθημερινά τεράστιος όγκος πληροφοριών δημιουργείται και ανταλλάσεται μέσω του γραπτού λόγου. Η συγκεκριμένη μορφή επικοινωνίας ωστόσο είναι ιδιαίτερα ευάλωτη σε κακόβουλες ενέργειες. Καθημερινά παραδείγματα αποτελούν τα spam emails, τα οποία επιχειρούν να εξαπατήσουν χρήστες για την αποκάλυψη ευαίσθητων δεδομένων, ενώ malicious tweets και αναρτήσεις σε κοινωνικά δίκτυα συχνά χρησιμοποιούνται για τη διασπορά παραπληροφόρησης. Παράλληλα, η διάδοση bots και deepfake τεχνικών δημιουργούν την ανάγκη για πιο εξελιγμένα μέσα ανίχνευσης και προστασίας. Σε τέτοιες περιπτώσεις, η χρήση παραδοσιακών μεθόδων αυθεντικοποίησης, όπως κωδικοί πρόσβασης ή βιομετρικά δεδομένα, δεν επαρκεί. Ακόμα και αν εξασφαλίσουν την αρχική πρόσβαση, δεν παρέχουν διαρκή προστασία καθ' όλη τη διάρκεια χρήσης της υπηρεσίας, αφήνοντας τα συστήματα ευάλωτα σε δυνητικές επιθέσεις. Είναι επιτακτική, επομένως, η ανάγκη για περισσότερο δυναμικές μεθόδους αυθεντικοποίησης, που θα ενσωματώνουν γλωσσικά χαρακτηριστικά μέσω της ανάλυσης γραφής 
και θα προσφέρουν μια διακριτική και αξιόπιστη λύση, επιτρέποντας τη συνεχή παρακολούθηση της ταυτότητας του χρήστη με βάση το προσωπικό του στυλ γραφής.

Η έμμεση αυθεντικοποίηση, η οποία αξιοποιεί χαρακτηριστικά της φυσικής συμπεριφοράς του χρήστη, προσφέρει μια πιο φυσική και ασφαλή λύση. Ειδικότερα, η ανάλυση γραφής, που ενσωματώνει στοιχεία του προσωπικού στυλ του χρήστη, επιτρέπει την αναγνώριση ταυτότητας με τρόπο διακριτικό και ανεξάρτητο. Καθώς κάθε άτομο έχει τον δικό του τρόπο διατύπωσης και χρήσης της γλώσσας, οι αποκλίσεις στη γραφή μπορούν να ανιχνευθούν μέσω ενός προσαρμοσμένου μοντέλου που εκπαιδεύεται και αναγνωρίζει τον αυθεντικό χρήστη.

Η χρήση τεχνικών Επεξεργασίας Φυσικής Γλώσσας επιτρέπει την εξαγωγή χαρακτηριστικών που μπορούν να χρησιμοποιηθούν ως μοναδικά "ψηφιακά αποτυπώματα" κάθε χρήστη, βασιζόμενα σε δείκτες όπως το μέσο μήκος λέξεων σε χαρακτήρες, η συχνότητα συγκεκριμένων συντακτικών δομών ή μερών του λόγου και η ποικιλία του λεξιλογίου. Οι τεχνικές NLP μετατρέπουν τον γραπτό λόγο σε ένα σύνολο ποσοτικών μετρήσεων που αναλύονται με τεχνικές μηχανικής μάθησης. Με αυτόν τον τρόπο, η ταυτοποίηση πραγματοποιείται με βάση διακριτά χαρακτηριστικά του γραπτού λόγου, επιτρέποντας την αδιάλειπτη επαλήθευση ταυτότητας, χωρίς να απαιτείται η άμεση παρέμβαση του χρήστη.

Η έμμεση αυθεντικοποίηση δεν προσφέρει μόνο ένα νέο επίπεδο ασφάλειας, αλλά και σημαντικά πλεονεκτήματα όσον αφορά τη χρηστικότητα. Αντί να διακόπτει την εμπειρία του χρήστη, ενσωματώνεται αδιάλειπτα στη διαδικασία της αλληλεπίδρασης με το σύστημα. Το γεγονός αυτό την καθιστά ιδανική για περιβάλλοντα όπου η συνεχής πρόσβαση στα δεδομένα είναι απαραίτητη και η διακοπή της ροής ενεργειών για λόγους ασφάλειας μπορεί να είναι επιζήμια ή ενοχλητική για τον χρήστη. Επιπλέον, η δυνατότητα της έμμεσης αυθεντικοποίησης να ανιχνεύει απειλές χωρίς να επιβαρύνει τον χρήστη προσφέρει μια πιο ολιστική προσέγγιση στην προστασία της ταυτότητας και των δεδομένων του.

Η ανάπτυξη και η εξέλιξη του τομέα της συνεχούς και έμμεσης αυθεντικοποίησης έχουν ιδιαίτερη σημασία, καθώς προσφέρουν τη δυνατότητα για πιο ανθεκτικά και προσαρμοστικά συστήματα ασφαλείας. Στο πλαίσιο αυτής της εργασίας, η έμμεση και συνεχής αυθεντικοποίηση υλοποιείται μέσω τεχνικών NLP και μηχανικής μάθησης, που επιτρέπουν την εκμάθηση και ανίχνευση μοναδικών χαρακτηριστικών του γραπτού λόγου του χρήστη. Χρησιμοποιώντας ένα μοντέλο One-Class Support Vector Machine, το σύστημα αναγνωρίζει πρότυπα γραφής του χρήστη και ανιχνεύει αποκλίσεις που μπορεί να υποδηλώνουν παραβίαση στο σύστημα. Το προτεινόμενο σύστημα καταδεικνύει τη δυνατότητα των σύγχρονων τεχνολογιών να προσφέρουν λύσεις που ανταποκρίνονται στις αυξανόμενες ανάγκες για ασφάλεια, παρέχοντας ταυτόχρονα ένα εύχρηστο και ελκυστικό προς τον χρήστη περιβάλλον.

\section{Σκοπός - Συνεισφορά της Διπλωματικής Εργασίας}
\label{section:contribution}

Η παρούσα διπλωματική εργασία μελετά την ανάπτυξη και αξιολόγηση ενός συστήματος συνεχούς και έμμεσης αυθεντικοποίησης χρηστών, βασισμένου σε χαρακτηριστικά γραφής που εξάγονται μέσω τεχνικών επεξεργασίας φυσικής γλώσσας και μηχανικής μάθησης. Στόχος της εργασίας είναι η ανάπτυξη ενός συστήματος το οποίο μπορεί να ταυτοποιεί χρήστες με διακριτικό και συνεχόμενο τρόπο, αναγνωρίζοντας τον μοναδικό τρόπο γραφής τους. Η διαδικασία αυθεντικοποίησης πραγματοποιείται μέσω μοντέλων OC-SVM, τα οποία εκπαιδεύονται ώστε να αναγνωρίζουν αποκλίσεις από τη φυσιολογική γραφή του κάθε χρήστη, αποκλείοντας έτσι τους μη εξουσιοδοτημένους χρήστες από τη χρήση του εκάστοτε συστήματος.

Εξετάζεται η χρήση των τεχνικών NLP για την εξαγωγή γλωσσικών χαρακτηριστικών, όπως το μέσο μήκος λέξεων, η ποικιλία του λεξιλογίου και η δομή των προτάσεων, τα οποία μπορούν να αποδώσουν μια μοναδική ταυτότητα για κάθε χρήστη. Επιπλέον, παρουσιάζεται η εκπαίδευση και αξιολόγηση του μοντέλου OC-SVM για την έμμεση αυθεντικοποίηση, καθώς και η ανάλυση της αποτελεσματικότητας του προτεινόμενου συστήματος σε συνθήκες πραγματικής χρήσης. Η εργασία διερευνά την ακρίβεια και την απόδοση του συστήματος αυθεντικοποίησης, μετρώντας την αξιοπιστία του μέσω των δεικτών False Rejection Rate (FRR) και False Acceptance Rate (FAR).

Η εργασία συνεισφέρει στον τομέα της ψηφιακής ασφάλειας, προτείνοντας ένα σύστημα που προσφέρει διαρκή και αδιάλειπτη αυθεντικοποίηση χωρίς να απαιτεί συνεχείς ενέργειες από τον χρήστη, ενσωματώνοντας έτσι την επαλήθευση ταυτότητας στη φυσική ροή των καθημερινών δραστηριοτήτων. Παράλληλα, ανοίγει τον δρόμο για τη χρήση τεχνικών μηχανικής μάθησης και NLP στην αναγνώριση ταυτότητας χρηστών μέσω ανάλυσης γραφής, δημιουργώντας προοπτικές για την ανάπτυξη ασφαλών και ευέλικτων εφαρμογών σε περιβάλλοντα υψηλών απαιτήσεων.

\newpage
\section{Διάρθρωση της Αναφοράς}
\label{section:layout}

Η διάρθρωση της παρούσας διπλωματικής εργασίας είναι η εξής:

\begin{itemize}
  \item{\textbf{Κεφάλαιο \ref{chapter:sota}:}
    Γίνεται ανασκόπηση της ερευνητικής περιοχής με έμφαση στις τεχνικές αυθεντικοποίησης και αναγνώρισης χρηστών μέσω NLP και μηχανικής μάθησης.
    }
  \item{\textbf{Κεφάλαιο \ref{chapter:theory}:}
    Παρουσιάζεται το θεωρητικό υπόβαθρο της εργασίας, με ανάλυση των βασικών εννοιών της αυθεντικοποίησης, της επεξεργασίας φυσικής γλώσσας και της μηχανικής μάθησης, καθώς και του μοντέλου One-Class SVM.
    }
  \item{\textbf{Κεφάλαιο \ref{chapter:implementations}:}
    Παρουσιάζεται η υλοποίηση του συστήματος, συμπεριλαμβανομένων των αλγορίθμων εξαγωγής χαρακτηριστικών, της επεξεργασίας δεδομένων, της εκπαίδευσης των μοντέλων και της αξιολόγησης του συστήματος.
    }
  \item{\textbf{Κεφάλαιο \ref{chapter:experiments}:}
    Παρουσιάζεται η μεθοδολογία των πειραμάτων και τα αποτελέσματα αξιολόγησης του συστήματος, περιλαμβάνοντας τους δείκτες FAR και FRR.
    }
  \item{\textbf{Κεφάλαιο \ref{chapter:conclusions}:}
    Παρουσιάζονται τα τελικά συμπεράσματα και τα προβλήματα που προέκυψαν.
    }
  \item{\textbf{Κεφάλαιο \ref{chapter:future_work}:}
    Προτείνονται θέματα για μελλοντική
    μελέτη, αλλαγές και επεκτάσεις.
    }
\end{itemize}


