\chapter{Θεωρητικό Υπόβαθρο}
\label{chapter:theory}

Η συνεχής και έμμεση αυθεντικοποίηση μέσω ανάλυσης συμπεριφοράς αποτελεί μια σύγχρονη προσέγγιση που συνδυάζει την \emph{επεξεργασία φυσικής γλώσσας} και τις τεχνικές \emph{μηχανικής μάθησης}. Στόχος του κεφαλαίου αυτού είναι να παρουσιάσει τα θεωρητικά θεμέλια που καθιστούν δυνατή την υλοποίηση αυτής της προσέγγισης. Οι ενότητες που ακολουθούν αναλύουν τη διαδικασία της αυθεντικοποίησης, τις τεχνικές επεξεργασίας φυσικής γλώσσας και εξαγωγής χαρακτηριστικών, τις τεχνικές μηχανικής μάθησης, και τους αλγόριθμους που χρησιμοποιούνται για την ταξινόμηση δεδομένων και τον εντοπισμό αποκλίσεων.

\section{Αυθεντικοποίηση}
\label{sec:theory_auth}

Η αυθεντικοποίηση αποτελεί θεμελιώδη διαδικασία στον τομέα της ασφάλειας πληροφοριακών συστημάτων. Στόχος της είναι η επαλήθευση της ταυτότητας ενός χρήστη ή μιας συσκευής προτού παραχωρηθεί πρόσβαση σε δεδομένα ή υπηρεσίες. Η ανάγκη για αξιόπιστες μεθόδους αυθεντικοποίησης γίνεται ολοένα και πιο επιτακτική, εξαιτίας της αυξανόμενης πολυπλοκότητας των απειλών κυβερνοασφάλειας και των επιτιθέμενων που αναζητούν διαρκώς τρόπους να παρακάμψουν τα παραδοσιακά συστήματα ελέγχου ταυτότητας.


\subsection{Ορισμός Αυθεντικοποίησης}
Η αυθεντικοποίηση αναφέρεται στη διαδικασία επαλήθευσης της ταυτότητας ενός χρήστη ή συσκευής προτού επιτραπεί η πρόσβαση σε ένα σύστημα. Η διαδικασία περιλαμβάνει την ταυτοποίηση, δηλαδή τη δήλωση της ταυτότητας, και την επαλήθευση, που επιβεβαιώνει την ακρίβεια της δήλωσης. Για παράδειγμα, ένας χρήστης μπορεί να δηλώσει την ταυτότητά του μέσω του ονόματος χρήστη (ταυτοποίηση) και να την επαληθεύσει μέσω ενός κωδικού πρόσβασης ή μιας βιομετρικής μεθόδου (επαλήθευση).


\subsection{Κατηγορίες Τεχνικών Αυθεντικοποίησης}
Η εξέλιξη της τεχνολογίας έχει οδηγήσει στην ανάπτυξη ποικίλων συστημάτων αυθεντικοποίησης, που απαιτούν από τον χρήστη κάτι διαφορετικό κάθε φορά για την επαλήθευση της ταυτότητάς του. Η κατηγοριοποίηση αυτή επομένως μπορεί να αναλυθεί σε:
\begin{itemize}
    \item \textbf{Γνώση}: Αφορά κάτι που ο χρήστης γνωρίζει και συμπεριλαμβάνει κωδικούς πρόσβασης / PIN, μοτίβο ή απάντηση σε κάποια ερώτηση.
    \item \textbf{Κατοχή}: Αφορά κάτι που ο χρήστης κατέχει και συμπεριλαμβάνει φυσικά αντικείμενα όπως κάρτες ή tokens.
    \item \textbf{Βιομετρικά χαρακτηριστικά}: Αφορά έμφυτα φυσιολογικά ή συμπεριφορικά χαρακτηριστικά, γνωστά και ως βιομετρικά, που είναι μοναδικά για κάθε άτομο. Οι βιομετρικές τεχνικές χωρίζονται σε δύο κύριες κατηγορίες:
\begin{enumerate}
    \item \textbf{Φυσιολογικές Μέθοδοι}:
    \begin{itemize}
        \item \textit{Αναγνώριση προσώπου}: Χρήση χαρακτηριστικών του προσώπου για την ταυτοποίηση.
        \item \textit{Δακτυλικά αποτυπώματα}: Καταγραφή και αντιστοίχιση μοναδικών αποτυπωμάτων.
        \item \textit{Σάρωση ίριδας}: Εξαιρετικά ασφαλής μέθοδος, αλλά απαιτεί εξειδικευμένο εξοπλισμό.
    \end{itemize}
    \item \textbf{Συμπεριφορικές Μέθοδοι}:
    \begin{itemize}
        \item \textit{Ανάλυση γραφής}: Εξαγωγή χαρακτηριστικών από τον τρόπο που γράφει ένας χρήστης.
        \item \textit{Δυναμική πληκτρολόγηση}: Παρακολούθηση των μοτίβων πληκτρολόγησης.
        \item \textit{Συμπεριφορά πλοήγησης}: Ανάλυση τρόπων πλοήγησης σε περιβάλλοντα χρήστη.
    \end{itemize}
\end{enumerate}

Οι βιομετρικές μέθοδοι, σε αντίθεση με τις παραδοσιακές, δεν μπορούν εύκολα να παραβιαστούν, καθώς βασίζονται σε εγγενή χαρακτηριστικά του χρήστη.

\end{itemize}

Ακόμη, μπορούμε να διακρίνουμε 2 κατηγορίες τεχνικών αυθεντικοποίησης ανάλογα με τη διαφάνεια του συστήματος ως προς τον τελικό χρήστη.

\begin{itemize}
    \item \textbf{Ενεργή / Άμεση}: το σύστημα απαιτεί την εισαγωγή δεδομένων από τον χρήστη, όπως την πληκτρολόγηση ενός κωδικού ή μίας απάντησης σε μία ερώτηση ασφαλείας.
    \item \textbf{Παθητική / Έμμεση}: εκτελείται στο παρασκήνιο, χωρίς να χρειάζεται ενέργεια από τον χρήστη. Συστήματα αυτής της κατηγορίας ξεχωρίζουν για την δυνατότητά τους να εκτελούνται συνεχώς χωρίς να επεμβαίνουν στην λειτουργικότητα της συσκευής.
\end{itemize}

Σήμερα πολλές υπηρεσίες και εφαρμογές χρησιμοποιούν συνδυασμούς τεχνικών αυθεντικοποίησης (Multi Factor Authentication - MFA). Συχνότερο παράδειγμα αποτελεί η αυθεντικοποίηση 2 παραγόντων (two-factor authentication ή 2FA). Στη συγκεκριμένη κατηγορία εμπίπτουν η ανάληψη χρημάτων από το ATM με την χρήση κάρτας (κατοχή) και την πληκτρολόγηση του PIN (γνώση), αλλά και η σύνδεση σε λογαριασμούς ηλεκτρονικού ταχυδρομείου με τον κωδικό πρόσβασης (γνώση) και ένα συνθηματικό που αποστέλλεται σε κάποια συσκευή του χρήστη (κατοχή).

\subsection{Συνεχής και Έμμεση Αυθεντικοποίηση βασιζόμενη σε συμπεριφορικές μεθόδους}
Η συνεχής και έμμεση αυθεντικοποίηση αποτελεί μία καινοτόμο προσέγγιση στον τομέα της ασφάλειας πληροφοριακών συστημάτων, η οποία δίνει έμφαση στην αδιάλειπτη και μη παρεμβατική επαλήθευση της ταυτότητας του χρήστη. Σε αντίθεση με τις παραδοσιακές μεθόδους αυθεντικοποίησης, οι οποίες συχνά απαιτούν τη ρητή συμμετοχή του χρήστη, όπως η εισαγωγή κωδικών πρόσβασης ή η χρήση βιομετρικών αναγνωριστικών, η συνεχής αυθεντικοποίηση αξιοποιεί πληροφορίες που συλλέγονται από τη συμπεριφορά του χρήστη και τις αλληλεπιδράσεις του με το σύστημα.

Οι τεχνικές συνεχούς και έμμεσης αυθεντικοποίησης χρησιμοποιούν δεδομένα όπως:
\begin{itemize}
    \item Τα μοτίβα γραφής και πληκτρολόγησης, που αναλύονται μέσω τεχνικών επεξεργασίας φυσικής γλώσσας (NLP) και μηχανικής μάθησης.
    \item Τα μοτίβα πλοήγησης σε ψηφιακά περιβάλλοντα, τα οποία παρέχουν στοιχεία σχετικά με τη συμπεριφορά του χρήστη.
    \item Βιομετρικά δεδομένα χαμηλής συχνότητας, όπως η δυναμική χρήσης της συσκευής (π.χ., κλίση ή ταχύτητα κύλισης).
\end{itemize}

Η συνεχής και έμμεση αυθεντικοποίηση προσφέρει σημαντικά πλεονεκτήματα:
\begin{enumerate}
    \item \textbf{Αυξημένη ασφάλεια:} Η συνεχής παρακολούθηση καθιστά δυσκολότερη την παραβίαση του συστήματος.
    \item \textbf{Μη παρεμβατική λειτουργία:} Οι χρήστες δεν χρειάζεται να διακόπτουν τη ροή των ενεργειών τους για να επαληθεύσουν την ταυτότητά τους.
    \item \textbf{Δυναμική προσαρμογή:} Τα συστήματα μπορούν να προσαρμόζονται στις αλλαγές της συμπεριφοράς του χρήστη, βελτιώνοντας τη συνολική ακρίβεια.
    \item \textbf{Εύκολη ενσωμάτωση:} Μπορεί να ενσωματωθεί σε υπάρχον hardware, χωρίς να απαιτούνται πρόσθετα κόστη εξοπλισμού.
    \item \textbf{Ευελιξία:} Μπορούν να χρησιμοποιηθούν πολλά διαφορετικά συμπεριφορικά χαρακτηριστικά ανάλογα με τις απαιτήσεις και το τελικό προϊόν που θα εξυπηρετεί το σύστημα. 
\end{enumerate}

Ένα παράδειγμα εφαρμογής αυτής της τεχνολογίας είναι η ανάλυση γραφής για αυθεντικοποίηση σε περιβάλλοντα συνομιλιών. Η τεχνική αυτή βασίζεται στην εξαγωγή χαρακτηριστικών, όπως η επιλογή λέξεων, η σύνταξη και η δομή των προτάσεων, που αποτελούν μοναδικά χαρακτηριστικά του χρήστη. Σε συνδυασμό με τεχνολογίες όπως οι αλγόριθμοι ανίχνευσης ανωμαλιών, η συνεχής αυθεντικοποίηση μπορεί να εξασφαλίσει ένα υψηλό επίπεδο ασφάλειας, χωρίς να επηρεάζει την εμπειρία του χρήστη.

\begin{figure}[H]
    \centering
    \includegraphics[width=\textwidth]{images/chapter3/CIAtheoreticalBackground.png}
    \caption{Πτυχές της συνεχούς και έμμεσης αυθεντικοποίησης}
    \label{fig:chapter3_CIA}
\end{figure}

Η χρήση τέτοιων τεχνολογιών ανοίγει νέες προοπτικές για εφαρμογές όπως η προστασία προσωπικών δεδομένων, η ασφαλής πρόσβαση σε κρίσιμες υποδομές, και η βελτίωση της εμπειρίας των χρηστών σε ψηφιακά περιβάλλοντα.

\section{Επεξεργασία Φυσικής Γλώσσας}
\label{sec:theory_nlp}

\subsection{Εισαγωγή στην Επεξεργασία Φυσικής Γλώσσας}
Η επεξεργασία φυσικής γλώσσας αποτελεί έναν από τους πλέον εξελισσόμενους τομείς της τεχνητής νοημοσύνης. Στόχος της είναι η κατανόηση, ανάλυση και εξαγωγή χαρακτηριστικών από κείμενα, δίνοντας τη δυνατότητα στα συστήματα να ερμηνεύουν και να επεξεργάζονται τη γλώσσα. Ο εξαιρετικά μεγάλος όγκος δεδομένων που ανταλλάσσονται διαρκώς υπό τη μορφή κειμένου καθιστά επιτακτική την ανάγκη για αυξημένη ασφάλεια. Στο πλαίσιο αυτής της εργασίας, το NLP αξιοποιείται για την εξαγωγή χαρακτηριστικών που αποτυπώνουν μοναδικές γλωσσικές και συμπεριφορικές πτυχές κάθε χρήστη.

Η κατανόηση των κειμένων περιλαμβάνει διαδικασίες όπως η αναγνώριση της δομής, η σημασιολογική ανάλυση και η δημιουργία μοναδικών γλωσσικών προφίλ. Αυτές οι διαδικασίες παίζουν κρίσιμο ρόλο στη δημιουργία συστημάτων συνεχούς αυθεντικοποίησης χρηστών, όπου ο στόχος είναι η ανίχνευση μοτίβων γραφής που επιτρέπουν την ασφαλή ταυτοποίηση.

Η εξέλιξη του κλάδου NLP ξεκίνησε με τις πρώτες γλωσσολογικές προσεγγίσεις, όπως οι τεχνικές Bag-of-Words και Term Frequency - Inverse Document Frequency (TF-IDF), οι οποίες βασίζονταν σε απλή μέτρηση συχνοτήτων λέξεων. Με την έλευση της μηχανικής μάθησης, η ανάλυση κειμένων πέρασε σε πιο σύνθετα επίπεδα, όπως η εξαγωγή γλωσσολογικών και σημασιολογικών χαρακτηριστικών.

Η εξέλιξη των εργαλείων NLP είναι ραγδαία. Αρχικά, το Word2Vec~\cite{mikolov2013efficient} εισήγαγε τη δυνατότητα εκμάθησης συσχετίσεων μεταξύ διανυσμάτων αναπαράστασης των λέξεων, οδηγώντας στην ανάλυση σημασιολογικών χαρακτηριστικών, πέρα από καθαρά γλωσσικών. Επίσης, το GloVe~\cite{pennington2014glove} βελτίωσε την προσέγγιση αυτή καθώς παρείχε στατιστικά σχετικά με τη συχνότητα των λέξεων μέσω της γεωμετρικής τους απεικόνισης σε πολυδιάστατους χώρους. Τέλος, μετασχηματιστές όπως ο BERT~\cite{devlin2018bert} αναπαριστούν το κείμενο ως μια σειρά από διανύσματα χρησιμοποιώντας επιβλεπόμενη μάθηση και μαθαίνουν λανθάνουσες αναπαραστάσεις των σημείων στο πλαίσιο των συμφραζομένων τους. Η τεχνική αυτή αποτέλεσε σημαντική βελτίωση έναντι των προηγούμενων μοντέλων.

Συνολικά, το NLP διαδραματίζει κεντρικό ρόλο στη δημιουργία μοναδικών προφίλ χρηστών. Μέσω της ανάλυσης γλωσσικών μοτίβων, επιτυγχάνεται η ταυτοποίηση χρηστών βασισμένη στη γραφή τους. Συστήματα αυθεντικοποίησης βασισμένα στο NLP χρησιμοποιούνται σε εφαρμογές ασφάλειας, όπου απαιτείται υψηλή ακρίβεια και αξιοπιστία.

\subsubsection{Tokenization}
Η διαίρεση του κειμένου σε λέξεις, φράσεις ή προτάσεις αποτελεί το πρώτο στάδιο επεξεργασίας. Το Tokenization επιτρέπει την απομόνωση σημαντικών τμημάτων του κειμένου για περαιτέρω ανάλυση.

\subsubsection{Stemming και Lemmatization}
Η απλοποίηση λέξεων στην αρχική τους μορφή βελτιώνει την ακρίβεια της γλωσσικής ανάλυσης. Το Stemming αφαιρεί τα προσφύματα των λέξεων, ενώ το Lemmatization διατηρεί τη γραμματική ακεραιότητα.

\subsubsection{Part-of-Speech Tagging}
Η επισήμανση της γραμματικής κατηγορίας (π.χ. ουσιαστικά, ρήματα) παρέχει σημαντικές πληροφορίες για τη σύνταξη και τη σημασιολογία.

\subsubsection{Named Entity Recognition (NER)}
Το NER αναγνωρίζει οντότητες, όπως ονόματα, ημερομηνίες ή τοποθεσίες, βοηθώντας στη δημιουργία πλούσιων γλωσσικών προφίλ.

\subsubsection{Dependency Parsing}
Αναλύει τις συντακτικές σχέσεις μεταξύ λέξεων, αποκαλύπτοντας τη δομή του κειμένου.

\subsection{Εργαλεία και Μέθοδοι Εξαγωγής Χαρακτηριστικών}
Τα χαρακτηριστικά εξάγονται με τη χρήση εργαλείων όπως:
\begin{itemize}
    \item \textbf{NLTK\footnote{\url{https://www.nltk.org/}} και spaCy\footnote{\url{https://spacy.io/}}:} Για μορφολογική και συντακτική ανάλυση.
    \item \textbf{textstat\footnote{\url{https://texstat.org/}}:} Για δείκτες αναγνωσιμότητας και πολυπλοκότητας.
\end{itemize}


\newpage
\section{Σύγχρονη Μηχανική Μάθηση}
\label{sec:theory_ml}

Η μηχανική μάθηση (Machine Learning, ML) αποτελεί έναν από τους πλέον δυναμικά εξελισσόμενους τομείς της επιστήμης των υπολογιστών, με εκτεταμένες εφαρμογές στην ανάλυση δεδομένων, τη λήψη αποφάσεων και την κατανόηση της φυσικής γλώσσας. Η ουσία της μηχανικής μάθησης έγκειται στη δυνατότητα των συστημάτων να «μαθαίνουν» από τα δεδομένα και να βελτιώνουν την απόδοσή τους χωρίς ρητές οδηγίες προγραμματισμού. Με τη χρήση εξελιγμένων αλγορίθμων, τα μοντέλα μηχανικής μάθησης αναπτύσσουν ικανότητες για την εξαγωγή μοτίβων και τη δημιουργία προβλέψεων σε πραγματικό χρόνο.

Η εισαγωγή του κλάδου της μηχανικής μάθησης στην επιστήμη των υπολογιστών,
επέτρεψε στους υπολογιστές να μπορούν να αντιμετωπίσουν προβλήματα αντίληψης
για τον πραγματικό κόσμο, όσο και να παίρνουν υποκειμενικές αποφάσεις.

Οι αλγόριθμοι ML επιτρέπουν σε συστήματα Τεχνητής Νοημοσύνης (Artificial Intelligence, AI)
να προσαρμόζονται εύκολα σε καινούργια προβλήματα απαιτώντας ελάχιστη επέμβαση από τον άνθρωπο.
Για παράδειγμα, ένα νευρωνικό δίκτυο που έχει εκπαιδευτεί να αναγνωρίζει γάτες σε εικόνες,
δεν απαιτεί να σχεδιαστεί και να εκπαιδευτεί από το μηδέν για να έχει την ικανότητα
να αναγνωρίζει και σκύλους.

\begin{figure}[!ht]
  \centering
  \includegraphics[width=1\textwidth]{./images/chapter3/AI_1.jpg}
  \caption[Κλάδοι και εφαρμογές της επιστήμης της Τεχνητής Νοημοσύνης]{Κλάδοι και εφαρμογές της επιστήμης της Τεχνητής Νοημοσύνης}
  \label{fig:ai_1}
\end{figure}

\subsection{Κατηγορίες Αλγορίθμων Μηχανικής Μάθησης}
Πολλά προβλήματα που μέχρι πριν μερικά χρόνια λύνονταν με
“χειρόγραφη”, προγραμματισμένη από τον άνθρωπο γνώση, σήμερα επιλύονται με χρήση
αλγορίθμων ML (\autoref{fig:ai_1}). Κάποια παραδείγματα αφορούν:

\begin{itemize}
  \item{Αναγνώριση ομιλίας - Speech Recognition}
  \item{Μηχανική όραση - Computer Vision}
  \begin{itemize}
    \item{Αναγνώριση αντικειμένων σε εικόνες - Object Recognition}
    \item{Αναγνώριση και εντοπισμός της θέσης αντικειμένων σε εικόνες - Object Detection}
  \end{itemize}
  \item{Αναγνώριση ηλεκτρονικών επιθέσεων στο διαδίκτυο - Cyberattack detection}
  \item{Επεξεργασία φυσικής γλώσσας - Natural Language Processing}
  \begin{itemize}
    \item{Κατανόηση της φυσικής γλώσσας του ανθρώπου - Natural Language Understanding}
    \item{Μοντελοποίηση και παραγωγή της φυσικής γλώσσας του ανθρώπου από μηχανές - Natural Language Generation}
  \end{itemize}
  \item{Μηχανές αναζήτησης - Search Engines}
  \item{Αναπαράσταση γνώσης - Knowledge Representation}
  \item{Ρομποτική}
\end{itemize}

Η επιστήμη της μηχανικής μάθησης μπορεί να ταξινομηθεί σε τρεις κύριες κατηγορίες:
\begin{enumerate}
    \item \textbf{Εποπτευόμενη Μάθηση (Supervised Learning)}: Το μοντέλο εκπαιδεύεται σε σύνολα δεδομένων όπου υπάρχουν ετικέτες (labels) που καθοδηγούν τη διαδικασία εκμάθησης.
    \item \textbf{Μη Εποπτευόμενη Μάθηση (Unsupervised Learning)}: Το μοντέλο επιχειρεί να ανακαλύψει μοτίβα και δομές από δεδομένα χωρίς προκαθορισμένες ετικέτες.
    \item \textbf{Μάθηση Ενίσχυσης (Reinforcement Learning)}: Το μοντέλο βελτιώνει τη συμπεριφορά του μέσω επαναλαμβανόμενης αλληλεπίδρασης με το περιβάλλον και αξιολόγησης των ενεργειών του. Ένα παράδειγμα εφαρμογής
    είναι η αυτόματη πλοήγηση ενός οχήματος.
\end{enumerate}

Kάποια προβλήματα είναι υβριδικά, δηλαδή συνδυασμός των πιο πάνω.
Στο \autoref{fig:ml_venn_diagram} απεικονίζεται το διάγραμμα Venn των διαφόρων 
αλγοριθμικών κατηγοριών ML.
\begin{figure}[H]
  \centering
  \includegraphics[width=0.6\textwidth]{./images/chapter3/ml_venn_diagram.jpg}
  \caption[Διάγραμμα Venn των διαφόρων κατηγοριών μηχανικής μάθησης]{Διάγραμμα Venn των διαφόρων κατηγοριών μηχανικής μάθησης}
  \label{fig:ml_venn_diagram}
\end{figure}

Επιπλέον, οι Supervised Learning αλγόριθμοι χωρίζονται σε 2 κατηγορίες, όπως φαίνεται στο~\autoref{fig:chapter3_regressionVSclassification}, ανάλογα
με την επιθυμητή μορφή της εξόδου του αλγόριθμου ML:
\begin{itemize}
  \item{Ταξινόμησης - Classification: Πρόβλεψη μίας διακριτής κατηγορίας ή κλάσης. Η έξοδος είναι μία διακριτή ετικέτα (label).}
  \item{Παλινδρόμησης - Regression: Πρόβλεψη μίας συνεχούς μεταβλητής. Η έξοδος είναι μια συνεχής τιμή. Οι αλγόριθμοι παλινδρόμησης παράγουν ένα μοντέλο που μπορεί να προβλέψει αριθμητικές τιμές.}
\end{itemize}

\begin{figure}[H]
    \centering
    \includegraphics[width=0.6\textwidth]{images/chapter3/regressionVSclassification.png}
    \caption{Διαφορά classification \& regression}
    \label{fig:chapter3_regressionVSclassification}
\end{figure}

\subsection{Προκλήσεις}
Το \textbf{overfitting} και το \textbf{underfitting} αποτελούν δύο από τις πιο σημαντικές προκλήσεις στη μηχανική μάθηση, καθώς επηρεάζουν άμεσα την ικανότητα του μοντέλου να γενικεύει σε νέα δεδομένα. Το \textbf{overfitting} προκύπτει όταν το μοντέλο μαθαίνει υπερβολικά καλά τα δεδομένα εκπαίδευσης, συμπεριλαμβανομένων τυχόν θορύβων ή σφαλμάτων, με αποτέλεσμα να αποδίδει εξαιρετικά στα δεδομένα αυτά, αλλά να αποτυγχάνει στα δεδομένα δοκιμών ή σε νέα δεδομένα. Αυτό συμβαίνει όταν το μοντέλο είναι υπερβολικά περίπλοκο, για παράδειγμα, περιέχει πολλές παραμέτρους σε σχέση με τα διαθέσιμα δεδομένα. Αντίθετα, το \textbf{underfitting} εμφανίζεται όταν το μοντέλο αποτυγχάνει να μάθει επαρκώς τα μοτίβα των δεδομένων εκπαίδευσης, με αποτέλεσμα να έχει χαμηλή απόδοση τόσο στα δεδομένα εκπαίδευσης όσο και σε νέα δεδομένα. Αυτό συμβαίνει συχνά όταν το μοντέλο είναι υπερβολικά απλό ή οι παράμετροί του δεν έχουν ρυθμιστεί σωστά. 

Η αντιμετώπιση αυτών των προκλήσεων απαιτεί προσεκτική επιλογή της αρχιτεκτονικής του μοντέλου, της μεθόδου εκπαίδευσης και των υπερπαραμέτρων, καθώς και τη χρήση τεχνικών όπως η διασταυρούμενη επικύρωση, η κανονικοποίηση και η αύξηση του μεγέθους ή της ποικιλίας των δεδομένων εκπαίδευσης. Η \textbf{διασταυρούμενη επικύρωση (cross-validation, CV)} είναι μια τεχνική που χρησιμοποιείται για την αξιολόγηση της απόδοσης ενός μοντέλου και περιλαμβάνει τη διαίρεση των δεδομένων σε υποσύνολα (folds). Το μοντέλο εκπαιδεύεται επανειλημμένα σε διαφορετικά υποσύνολα και δοκιμάζεται σε αυτά που εξαιρούνται, ώστε να εκτιμηθεί η γενική απόδοσή του σε νέα δεδομένα. Η \textbf{κανονικοποίηση (regularization)} είναι η διαδικασία μετατροπής των δεδομένων σε μία ακολουθία κανονικών μορφών, οι οποίες αποτελούνται από απλές και σαφείς σχέσεις που δεν περιέχουν επαναλήψεις. Ως στόχο έχει να περιορίσει την πολυπλοκότητα του μοντέλου, αποτρέποντας έτσι το overfitting.

\begin{figure}[H]
    \centering
    \includegraphics[width=\textwidth]{images/chapter3/underOverBestFitting.png}
    \caption{Πρώτο γράφημα: underfitting, Δεύτερο γράφημα: best fit, Τρίτο γράφημα: overfitting}
    \label{fig:chapter3_underOverBestFitting}
\end{figure}

\subsection{Το SVM και το One-Class SVM}
\label{subsec:svm_ocsvm}

Ένας από τους θεμελιώδεις αλγορίθμους στη μηχανική μάθηση είναι το \textit{Support Vector Machine} (SVM), το οποίο είναι ιδιαίτερα ισχυρό για την επίλυση προβλημάτων ταξινόμησης (\textit{classification}) και παλινδρόμησης (\textit{regression}). Το SVM βασίζεται στη χρήση ενός υπερεπιπέδου (\textit{hyperplane}) που διαχωρίζει δεδομένα σε διαφορετικές κλάσεις στον χώρο χαρακτηριστικών (\textit{feature space}). Ο αλγόριθμος επιλέγει το υπερεπίπεδο που μεγιστοποιεί το περιθώριο (\textit{margin}) μεταξύ των δεδομένων των διαφορετικών κλάσεων, όπως φαίνεται στο~\autoref{fig:chapter3_svmGraph}.

\begin{figure}[H]
    \centering
    \includegraphics[width=0.45\textwidth]{images/chapter3/svm-graph.png}
    \caption{Λειτουργία του SVM - Κατασκευή hyperplane με το maximum margin}
    \label{fig:chapter3_svmGraph}
\end{figure}

\subsubsection{Λειτουργία του SVM}
Το SVM λειτουργεί ως εξής:
\begin{enumerate}
    \item \textbf{Μετασχηματισμός Δεδομένων}:
    Τα δεδομένα μεταφέρονται σε έναν υψηλής διάστασης χώρο μέσω μιας μη γραμμικής συνάρτησης πυρήνα (\textit{kernel function}), όπως η RBF (Radial Basis Function) ή ο πολυωνυμικός πυρήνας. Συχνά απαιτείται μετασχηματισμός των δεδομένων σε διαφορετικό σύστημα συντεταγμένων, ώστε αυτά να είναι γραμμικά διαχωρίσιμα, όπως φαίνεται στο~\autoref{fig:chapter3_representation}.
    \item \textbf{Κατασκευή Υπερεπιπέδου}:
    Το SVM επιλέγει το υπερεπίπεδο που μεγιστοποιεί το περιθώριο μεταξύ των δύο κλάσεων δεδομένων. Στο~\autoref{fig:chapter3_optimalHyperplane} με μαύρο χρώμα φαίνεται η ευθεία που μεγιστοποιεί αυτό το περιθώριο. 
    \item \textbf{Υποστήριξη Σημείων (Support Vectors)}:
    Τα δεδομένα που βρίσκονται πλησιέστερα στο υπερεπίπεδο καλούνται \textit{support vectors} και καθορίζουν τη θέση και τον προσανατολισμό του.
\end{enumerate}

\begin{figure}[H]
    \centering
    \includegraphics[width=0.45\textwidth]{images/chapter3/optimalHyperplane.png}
    \caption{Διαχωρισμός δεδομένων με τη χρήση του αλγορίθμου SVM. Με μαύρο φαίνεται η ευθεία που μεγιστοποιεί το περιθώριο ενώ με κόκκινο και πράσινο φαίνονται άλλες - μη βέλτιστες - επιλογές ευθειών διαχωρισμού}
    \label{fig:chapter3_optimalHyperplane}
\end{figure}

\begin{figure}[H]
    \centering
    \includegraphics[width=0.9\textwidth]{images/chapter3/representation_dependency.png}
    \caption{Μη γραμμικά διαχωρίσιμα δεδομένα μετασχηματίζονται σε διαφορετικό χώρο υψηλής διάστασης}
    \label{fig:chapter3_representation}
\end{figure}

Το SVM είναι κατάλληλο για προβλήματα εποπτευόμενης μάθησης με δύο ή περισσότερες κλάσεις. Ωστόσο, όταν πρόκειται για ανίχνευση ανωμαλιών ή μοτίβων σε δεδομένα χωρίς ετικέτες, εισάγεται η επέκτασή του: το \textit{One-Class SVM}.

\subsubsection{One-Class SVM: Εξειδίκευση για Ανίχνευση Ανωμαλιών}
Το \textit{One-Class Support Vector Machine} (One-Class SVM) είναι ένας αλγόριθμος μηχανικής μάθησης που ανήκει στην κατηγορία της μη εποπτευόμενης μάθησης και χρησιμοποιείται κυρίως για την ανίχνευση ανωμαλιών και την αναγνώριση μοτίβων σε δεδομένα. Αναπτύχθηκε από τους Schölkopf et al.~\cite{scholkopf2001ocsvm}, και αποτελεί μία εκτεταμένη εφαρμογή του κλασικού SVM, που έχει σχεδιαστεί για να μοντελοποιεί τη διανομή ενός μόνο κλάδου (class). Είναι μία επέκταση του SVM που χρησιμοποιείται για προβλήματα μη εποπτευόμενης μάθησης. Σκοπός του είναι να μοντελοποιήσει την κανονική κατανομή των δεδομένων και να αναγνωρίσει αποκλίσεις ή ανωμαλίες. 

\subsubsection{Βασικές Αρχές}
Το OC-SVM προσπαθεί να περικλείσει όλα τα κανονικά δεδομένα σε έναν υψηλής διάστασης χώρο μέσω μίας υπερ-επιφάνειας (\textit{hyperplane}) ή ενός υπερσφαιρικού χώρου. Οτιδήποτε βρίσκεται εκτός αυτής της περιοχής θεωρείται ανωμαλία. Στο~\autoref{fig:chapter3_svmCircle} με μπλε χρώμα φαίνονται τα δεδομένα εκπαίδευσης που καθορίζουν τον υπερσφαιρικό χώρο. Με πράσινο φαίνονται τα δεδομένα που ανήκουν σε αυτόν τον χώρο ενώ με κόκκινο φαίνονται οι ανωμαλίες.

\begin{figure}[H]
    \centering
    \includegraphics[width=0.7\textwidth]{images/chapter3/svm-circle.png}
    \caption{Επίδειξη λειτουργίας OC-SVM σε δισδιάστατο χώρο}
    \label{fig:chapter3_svmCircle}
\end{figure}

\subsubsection{Μαθηματική Διατύπωση}
Η βασική ιδέα πίσω από το One-Class SVM είναι η αναπαράσταση του συνόλου δεδομένων σε έναν υψηλής διάστασης χώρο χαρακτηριστικών μέσω μίας μη γραμμικής συνάρτησης πυρήνα (\textit{kernel function}). Στον χώρο αυτόν, το μοντέλο επιχειρεί να κατασκευάσει μία υπερ-επιφάνεια (\textit{hyperplane}) που περικλείει τη μεγαλύτερη δυνατή ποσότητα των δεδομένων, ελαχιστοποιώντας παράλληλα την απόσταση των σημείων από την επιφάνεια.

Η μαθηματική διατύπωση περιλαμβάνει την επίλυση του εξής βελτιστοποιητικού προβλήματος:
\[
\min_{\mathbf{w}, \xi, \rho} \frac{1}{2} \|\mathbf{w}\|^2 + \frac{1}{\nu N} \sum_{i=1}^N \xi_i - \rho,
\]
υπό τους περιορισμούς:
\[
(\mathbf{w} \cdot \phi(\mathbf{x}_i)) \geq \rho - \xi_i, \quad \xi_i \geq 0, \quad i = 1, \ldots, N.
\]
Όπου:
\begin{itemize}
    \item $\mathbf{w}$: Το διάνυσμα των παραμέτρων του μοντέλου.
    \item $\phi(\mathbf{x}_i)$: Η συνάρτηση που μετασχηματίζει τα δεδομένα στον χώρο χαρακτηριστικών.
    \item $\xi_i$: Οι μεταβλητές χαλάρωσης που επιτρέπουν ορισμένα σημεία να βρεθούν εκτός της επιφάνειας.
    \item $\rho$: Η παράμετρος που καθορίζει την απόσταση του υπερεπιπέδου από την αρχή.
    \item $\nu$: Ένας υπερπαράμετρος που ελέγχει το ποσοστό των σημείων που θεωρούνται εκτός της επιφάνειας.
\end{itemize}

\subsubsection{Διαδικασία Λειτουργίας}
Η διαδικασία λειτουργίας του One-Class SVM περιλαμβάνει τα εξής στάδια:
\begin{enumerate}
    \item \textbf{Εκπαίδευση}:
    Το μοντέλο εκπαιδεύεται σε ένα σύνολο δεδομένων που περιλαμβάνει μόνο τα κανονικά δεδομένα (genuine data). Ο στόχος είναι να εντοπίσει μία περιοχή στον χώρο χαρακτηριστικών που περικλείει τα δεδομένα αυτά.
    \item \textbf{Αναγνώριση}:
    Κατά τη φάση δοκιμών, τα νέα δεδομένα αξιολογούνται με βάση την απόστασή τους από την υπερ-επιφάνεια. Σημεία που βρίσκονται εκτός της καθορισμένης περιοχής θεωρούνται ανωμαλίες ή impostor δεδομένα.
    \item \textbf{Ενημέρωση}:
    Το μοντέλο μπορεί να προσαρμοστεί ώστε να ενσωματώσει νέα δεδομένα, βελτιώνοντας έτσι τη δυνατότητα ανίχνευσης μεταβαλλόμενων μοτίβων.
\end{enumerate}

\subsubsection{Πλεονεκτήματα και Εφαρμογές}
Το OC-SVM διακρίνεται για τα εξής:
\begin{itemize}
    \item \textbf{Ικανότητα Ανίχνευσης Ανωμαλιών}:
    Το OC-SVM έχει αποδειχθεί εξαιρετικά αποτελεσματικό στην ανίχνευση ανωμαλιών, καθώς μπορεί να ανιχνεύσει μοτίβα σε δεδομένα χωρίς ετικέτες, καθιστώντας το ιδανικό για ανίχνευση απάτης ή βλάβης~\cite{scholkopf2001ocsvm}.
    \item \textbf{Ευελιξία}:
    Μπορεί να εφαρμοστεί σε διάφορους τύπους δεδομένων, από κείμενο και αριθμητικά δεδομένα μέχρι εικόνες~\cite{seo2007application}.
\end{itemize}

Το One-Class SVM αποτελεί ένα θεμελιώδες εργαλείο για την ανίχνευση ανωμαλιών. Έχει εφαρμοστεί επιτυχώς σε πολλαπλά πεδία, όπως η ανίχνευση επιθέσεων σε συστήματα SCADA~\cite{laskov2004intrusion}, η ανάλυση βιομετρικών δεδομένων~\cite{hong2008fingerprint}, καθώς και σε προβλήματα ανίχνευσης και παρακολούθησης συστημάτων σε πραγματικό χρόνο~\cite{sun2017abnormal}. Η χρήση του One-Class SVM στην παρούσα εργασία επιτρέπει τη δημιουργία εξατομικευμένων μοντέλων αυθεντικοποίησης και ανίχνευσης ανωμαλιών μεταξύ των χρηστών.
