\section{Πέμπτη Φάση Πειραμάτων: LOSO Cross Validation}
\label{sec:experiments_phase5}

Η πέμπτη φάση αφορά την τεχνική Leave-One-Subject-Out Cross Validation, η οποία στοχεύει στην ανάδειξη της δυνατότητας γενίκευσης του μοντέλου.

Η τεχνική Leave-One-Subject-Out Cross-Validation (LOSO-CV) αποτελεί μια εξειδικευμένη προσέγγιση του ευρύτερου πλαισίου της διαδικασίας \textbf{Cross-Validation}. Το Cross-Validation χρησιμοποιείται συχνά στη μηχανική μάθηση για την αξιολόγηση της δυνατότητας γενίκευσης ενός μοντέλου. Σκοπός του είναι να επιβεβαιώσει πως τα αποτελέσματα που παράγονται από ένα μοντέλο δεν είναι προσαρμοσμένα αποκλειστικά στο σύνολο των δεδομένων εκπαίδευσης αλλά μπορούν να γενικευτούν και σε άγνωστα δεδομένα.

Κατά τη διαδικασία Cross-Validation, το σύνολο δεδομένων διαχωρίζεται σε υποκατηγορίες (\emph{folds}). Ένα από τα \emph{folds} χρησιμοποιείται για την αξιολόγηση (\emph{validation set}), ενώ τα υπόλοιπα χρησιμοποιούνται για την εκπαίδευση (\emph{training set}). Αυτή η διαδικασία επαναλαμβάνεται ώσπου κάθε \emph{fold} να έχει χρησιμοποιηθεί ως σύνολο αξιολόγησης. 

Συγκεκριμένα στη τεχνική LOSO, αντί για τυχαία \emph{folds}, κάθε επανάληψη του Cross-Validation επικεντρώνεται στον διαχωρισμό όλων των δεδομένων ενός συγκεκριμένου χρήστη από όλων των υπολοίπων, δηλαδή ολόκληρο το προφίλ ενός χρήστη αποκλείεται κατά τη διάρκεια της εκπαίδευσης και χρησιμοποιείται αποκλειστικά για την αξιολόγηση. Η προσέγγιση αυτή βρίσκει εφαρμογή και στην ανάλυση βιομετρικών δεδομένων, όπου η γενίκευση σε άγνωστα προφίλ χρηστών αποτελεί κριτήριο για την αποδοτικότητα του συστήματος.

\subsubsection{Περιγραφή της Διαδικασίας LOSO}
Η διαδικασία LOSO λειτουργεί ως εξής:
\begin{enumerate}
    \item Το σύνολο δεδομένων χωρίζεται με βάση τον χρήστη. Κάθε χρήστης θεωρείται μία μοναδική κατηγορία δεδομένων.
    \item Σε κάθε επανάληψη, τα δεδομένα ενός χρήστη (\emph{testing subject}) αφαιρούνται πλήρως από το σύνολο εκπαίδευσης και χρησιμοποιούνται αποκλειστικά για την αξιολόγηση.
    \item Τα υπόλοιπα δεδομένα των υπόλοιπων χρηστών (\emph{training subjects}) χρησιμοποιούνται για την εκπαίδευση των μοντέλων.
    \item Μετά την εκπαίδευση, τα μοντέλα αξιολογούνται στα δεδομένα του αποκλεισμένου χρήστη.
    \item Η διαδικασία επαναλαμβάνεται για κάθε χρήστη, διασφαλίζοντας ότι όλοι οι χρήστες έχουν χρησιμοποιηθεί μία φορά ως \emph{testing subjects}.
\end{enumerate}

Στο ~\autoref{fig:LOSOcv} φαίνεται η διαδικασία της τεχνικής LOSO για πολυδιάστατα δεδομένα (π.χ. δεδομένα από πολλαπλές πηγές ή συσκευές), ενώ στο ~\autoref{fig:exampleLOSOcv} παρουσιάζεται η βασική αρχή αυτής της τεχνικής, όπου κάθε χρήστης αποκλείεται διαδοχικά για να χρησιμοποιηθούν τα δεδομένα του ως (\emph{testing subject}).

\begin{figure}[H]
    \centering
    \begin{subfigure}{0.45\textwidth}
        \centering
        \includegraphics[width=0.8\textwidth]{images/chapter4/LOSOcv.jpg}
        \caption{Επεξήγηση της διαδικασίας LOSO για δεδομένα από διαφορετικές πηγές. Η κάθε πηγή δεδομένων αποκλείεται διαδοχικά και τα υπόλοιπα δεδομένα χρησιμοποιούνται για εκπαίδευση.}
        \label{fig:LOSOcv}
    \end{subfigure}
    \hfill
    \begin{subfigure}{0.45\textwidth}
        \centering
        \includegraphics[width=0.8\textwidth]{images/chapter4/exampleLOSOcv.png}
        \caption{Παραδείγματα του τρόπου λειτουργίας της τεχνικής LOSO. Κάθε χρήστης αποκλείεται διαδοχικά από το σύνολο εκπαίδευσης και χρησιμοποιείται ως σύνολο αξιολόγησης.}
        \label{fig:exampleLOSOcv}
    \end{subfigure}
    \caption{Γραφήματα επεξήγησης τεχνικής LOSO-CV}
    \label{fig:subgraphs}
\end{figure}

\paragraph{Υπολογισμένες Μετρικές}
Οι παρακάτω μετρικές χρησιμοποιούνται για την αξιολόγηση της απόδοσης της τεχνικής LOSO-CV:
\begin{itemize}
    \item \textbf{Accuracy:}
    Μετρά το ποσοστό των σωστών προβλέψεων, είτε για γνήσιους χρήστες είτε για εισβολείς, σε σχέση με το σύνολο των προβλέψεων, όπως αναλύεται στην~\autoref{subsec:metrics}.

    \item \textbf{MAE:}
    Αντιπροσωπεύει την ικανότητα του συστήματος να μειώνει τα σφάλματα πρόβλεψης, διατηρώντας την ακρίβεια στις εκτιμήσεις του. Ο υπολογισμός του αναλύεται στην~\autoref{subsec:metrics}.  
\end{itemize}

\subsection{Αποτελέσματα}
\label{sec:results_phase5}

Η τεχνική \textit{Leave-One-Subject-Out Cross Validation} αξιολογήθηκε με δύο μετρικές: την \textbf{Accuracy (Ακρίβεια)} και το \textbf{Mean Absolute Error (MAE)}. Οι μετρικές αυτές προσφέρουν μια ξεκάθαρη εικόνα για την απόδοση του συστήματος αυθεντικοποίησης σε επίπεδο χρήστη. Παρακάτω παρουσιάζονται τα αποτελέσματα:

\subsubsection{Ακρίβεια (Accuracy)} Η ακρίβεια δείχνει το ποσοστό των συνολικών προτροπών που ταξινομήθηκαν σωστά από το σύστημα. Υψηλότερες τιμές αντιπροσωπεύουν καλύτερη απόδοση στην αναγνώριση των γνήσιων χρηστών και στον αποκλεισμό των εισβολέων. Τα αποτελέσματα της ακρίβειας για κάθε χρήστη φαίνονται στο~\autoref{fig:loso_accuracy}.

\begin{figure}[H]
    \centering
    \includegraphics[width=\textwidth]{images/chapter5/losoAccu.png}
    \caption{Αποτελέσματα ακρίβειας (Accuracy) ανά χρήστη για την τεχνική LOSO.}
    \label{fig:loso_accuracy}
\end{figure}

\subsubsection{Μέσο Απόλυτο Σφάλμα (Mean Absolute Error - MAE)} Το MAE μετρά το μέσο σφάλμα μεταξύ των προβλεπόμενων και των πραγματικών τιμών και παρέχει μια ένδειξη για το επίπεδο αστοχίας της πρόβλεψης. Χαμηλότερες τιμές MAE αντιπροσωπεύουν καλύτερη απόδοση του συστήματος. Τα αποτελέσματα του MAE για κάθε χρήστη παρουσιάζονται στο~\autoref{fig:loso_mae}.

\begin{figure}[H]
    \centering
    \includegraphics[width=\textwidth]{images/chapter5/losoMae.png}
    \caption{Αποτελέσματα μέσου απόλυτου σφάλματος (MAE) ανά χρήστη για την τεχνική LOSO.}
    \label{fig:loso_mae}
\end{figure}

Επίσης, στο~\autoref{fig:loso_accu_mae} παρουσιάζονται οι μέσοι όροι των μετρικών Accuracy και ΜΑΕ. 

\begin{figure}[H]
    \centering
    \includegraphics[width=\textwidth]{images/chapter5/losoAccuMae.png}
    \caption{Μέσος όρος Accuracy \& MAE.}
    \label{fig:loso_accu_mae}
\end{figure}

