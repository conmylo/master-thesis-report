\section{Πρώτη Φάση Πειραμάτων: 1 Μοντέλο Ανά Χρήστη}
\label{sec:experiments_phase1}

\subsection{Πειραματική Διαδικασία}
Στη πρώτη φάση πειραμάτων, χρησιμοποιούμε ένα μοντέλο OC-SVM ανά χρήστη με σαφείς καθορισμένες παραμέτρους. Η πειραματική διαδικασία μπορεί να αναλυθεί σε:
\begin{itemize}
  \item{Προετοιμασία train \& test set για κάθε χρήστη.}
  \item{Εκπαίδευση μοντέλων με διαφορετικούς συνδυασμούς nu \& gamma για όλους τους χρήστες.}
  \item{Έλεγχος test prompts για όλους τους χρήστες με διαφορετικές τιμές για το threshold.}
  \item{Μετρικές αξιολόγησης - F1 \& FAR \& FRR - για συνδυασμούς παραμέτρων.}
\end{itemize}

\subsection{Hyperparameter \& Threshold Tuning}
\label{subsec:hypertuning}

\paragraph{Διαδικασία Tuning}
Η διαδικασία tuning είχε στόχο τη βελτιστοποίηση των παραμέτρων $nu$, $\gamma$ και του κατωφλιού (\textit{threshold}) για την εξασφάλιση της μέγιστης απόδοσης του συστήματος. Αξιοποιήθηκε η τεχνική grid search, κατά την οποία δίνουμε στο σύστημα ένα πλέγμα τιμών (grid) και μας επιστρέφει αποτελέσματα για κάθε δυνατό συνδυασμό τιμών του πλέγματος. Οι παράμετροι δοκιμάστηκαν στις εξής τιμές:
\begin{itemize}
    \item $nu$: \{0.0001, 0.001, 0.005, 0.01, 0.05, 0.1\}
    \item $\gamma$: \{0.05, 0.1, 0.2, 0.5, 1.0\}
    \item Κατώφλι: \{-0.01, 0.0, 0.01, 0.02, 0.03, 0.04, 0.05, 0.06, 0.07, 0.08\}
\end{itemize}

Οι μετρικές που αξιολογήθηκαν, όπως αναφέρεται και στο \autoref{sec:exmperiments_metrics}, περιλαμβάνουν:
\begin{itemize}
    \item \textbf{F1 Score:} Ο σταθμισμένος μέσος όρος της ακρίβειας και της ανάκλησης.
    \item \textbf{False Acceptance Rate (FAR):} Ποσοστό προτροπών εισβολέων που έγιναν δεκτές.
    \item \textbf{False Rejection Rate (FRR):} Ποσοστό γνήσιων προτροπών που απορρίφθηκαν.
\end{itemize}

\paragraph{Αποτελέσματα Tuning}
Τα αποτελέσματα της αναζήτησης πλέγματος παρουσιάζονται συνοπτικά στον Πίνακα~\ref{tab:balanced_results}. Οι καλύτερες τιμές των μετρικών για κάθε συνδυασμό παραμέτρων εμφανίζονται με έμφαση.

\begin{table}[H]
\centering
\begin{tabular}{|c|c|c|c|c|c|}
\hline
\textbf{$\nu$} & \textbf{$\gamma$} & \textbf{Threshold} & \textbf{F1 Score} & \textbf{FAR (\%)} & \textbf{FRR (\%)} \\ \hline
0.01 & 0.5 & 0.05 & \textbf{0.49707} & \textbf{44.18} & \textbf{49.58} \\ \hline
0.005 & 0.2 & 0.07 & \textbf{0.51769} & \textbf{43.85} & \textbf{46.72} \\ \hline
0.01 & 0.1 & 0.05 & 0.67351 & 86.02 & 5.73 \\ \hline
0.005 & 0.1 & 0.07 & 0.67279 & 85.75 & 5.97 \\ \hline
0.001 & 0.2 & 0.0 & 0.68526 & 86.38 & 3.18 \\ \hline
0.01 & 0.2 & 0.03 & 0.67170 & 81.88 & 8.37 \\ \hline
0.01 & 0.5 & 0.07 & \textbf{0.41699} & \textbf{35.20} & \textbf{39.90} \\ \hline
0.005 & 0.1 & 0.03 & 0.47475 & 41.65 & 46.38 \\ \hline
0.01 & 0.2 & 0.05 & \textbf{0.66543} & \textbf{39.85} & \textbf{38.92} \\ \hline
0.01 & 0.5 & 0.06 & 0.49714 & 41.58 & 45.62 \\ \hline
\end{tabular}
\caption{Επιλεγμένα Αποτελέσματα Tuning Παραμέτρων με Ισορροπία FAR και FRR.}
\label{tab:balanced_results}
\end{table}

\subsubsection{Παραδείγματα Ισορροπίας FAR και FRR}
\begin{itemize}
    \item \textbf{(ν = 0.01, γ = 0.5, threshold = 0.05):} 
    Παρουσιάζεται ισχυρή ισορροπία μεταξύ FAR (44.18\%) και FRR (49.58\%), αν καιτο F1 είναι σχετικά χαμηλό. Υποδεικνύεται ότι το σύστημα διαχειρίζεται ομοιόμορφα τους genuine και impostor χρήστες, ωστόσο τα ποσοστά των μετρικών δεν είναι ικανοποιητικά χαμηλά.
    
    \item \textbf{(ν = 0.01, γ = 0.5, threshold = 0.07):}
    Τα FAR και FRR είναι πιο χαμηλά (35.20\%, 39.90\% αντίστοιχα), υποδεικνύοντας καλύτερη διαχείριση τόσο των γνήσιων όσο και των εισβολέων χρηστών.

    \item \textbf{(ν = 0.01, γ = 0.2, threshold = 0.05):}
    Το παράδειγμα αυτό παρουσιάζει εξαιρετικά ισορροπημένες τιμές, με FAR (39.85\%) και FRR (38.92\%).
\end{itemize}

\subsubsection{Παραδείγματα Ακραίων Τιμών}
\begin{itemize}
    \item \textbf{(ν = 0.001, γ = 0.2, threshold = 0.0):}
    Παρόλο που το F1 είναι σχετικά υψηλό (0.68526), το FAR (86.38\%) είναι πολύ μεγαλύτερο από το FRR (3.18\%). Αυτό δείχνει ότι το σύστημα είναι πολύ επιεικές με genuine prompts, αλλά δυσκολεύεται να απορρίψει impostors.
    \item \textbf{(nu: 0.001, gamma: 0.05, threshold: 0.01):}
    Σε αυτή τη περίπτωση το FAR είναι εξαιρετικά χαμηλό, υπονοώντας πως το σύστημα καταφέρνει με επιτυχία να αναγνωρίσει impostors, αλλά η τιμή του FRR (97.62\%) υποδηλωνει πως το σύστημα σπάνια αναγνωρίζει τα genuine prompts. Το F1 παραμένει υψηλό.
\end{itemize}

\begin{figure}[H]
    \centering
    \includegraphics[width=\textwidth]{images/chapter5/5.1.2.png}
    \caption{FAR, FRR, F1 για διαφορετικά πλέγματα nu, gamma \& threshold}
    \label{fig:chapter5_image512}
\end{figure}

Στο~\autoref{fig:chapter5_image512} βλέπουμε τις συνολικές τιμές των μετρικών για διαφορετικά πλέγματα παραμέτρων.

\subsection{Αποτελέσματα Ανά Χρήστη με συγκεκριμένο πλέγμα}
\label{sec:results_threshold_0.05}

Ο παρακάτω πίνακας παρουσιάζει τις επιδόσεις του συστήματος για κάθε χρήστη με ρύθμιση κατωφλίου $0.05$, τη τιμή του nu $0.001$ και του gamma $0.05$. Οι μετρικές περιλαμβάνουν την ακρίβεια (\emph{Precision}), την ανάκληση (\emph{Recall}), το F1 Score, το \emph{False Acceptance Rate} (FAR), και το \emph{False Rejection Rate} (FRR). Επιπλέον, εμφανίζονται τα συνολικά FAR και FRR.

\begin{table}[H]
\centering
\begin{tabular}{|l|c|c|c|c|c|}
\hline
\textbf{User} & \textbf{Precision} & \textbf{Recall} & \textbf{F1 Score} & \textbf{FAR (\%)} & \textbf{FRR (\%)} \\ \hline
BarackObama & 1.00000 & 0.00000 & 0.00000 & 0.00 & 100.00 \\ \hline
Cristiano & 0.50828 & 0.95722 & 0.66399 & 92.60 & 4.28 \\ \hline
TheEllenShow & 0.52645 & 0.96759 & 0.68189 & 87.04 & 3.24 \\ \hline
Twitter & 0.52127 & 0.96898 & 0.67787 & 88.99 & 3.10 \\ \hline
YouTube & 0.52500 & 0.94508 & 0.67502 & 85.50 & 5.49 \\ \hline
britneyspears & 0.51703 & 0.95274 & 0.67030 & 88.99 & 4.73 \\ \hline
cnnbrk & 0.70671 & 0.49186 & 0.58003 & 20.41 & 50.81 \\ \hline
instagram & 0.65110 & 0.56267 & 0.60366 & 30.15 & 43.73 \\ \hline
jimmyfallon & 0.47433 & 0.07401 & 0.12805 & 8.20 & 92.60 \\ \hline
jtimberlake & 1.00000 & 0.00000 & 0.00000 & 0.00 & 100.00 \\ \hline
justinbieber & 0.88235 & 0.00763 & 0.01514 & 0.10 & 99.24 \\ \hline
ladygaga & 0.27027 & 0.00435 & 0.00856 & 1.17 & 99.57 \\ \hline
shakira & 0.54099 & 0.95644 & 0.69109 & 81.15 & 4.36 \\ \hline
taylorswift13 & 0.51259 & 0.92723 & 0.66020 & 88.17 & 7.28 \\ \hline
\textbf{Overall} & - & - & - & \textbf{61.68} & \textbf{55.82} \\ \hline
\end{tabular}
\caption{Αποτελέσματα Ανά Χρήστη με Threshold 0.05.}
\label{tab:threshold_0.05_results}
\end{table}

Όπως φαίνεται στον ~\ref{tab:threshold_0.05_results}, το σύστημα παρουσιάζει σημαντικές διαφοροποιήσεις στις επιδόσεις του μεταξύ των χρηστών. Οι διακυμάνσεις στις τιμές FAR και FRR είναι σημαντικές, υποδηλώνοντας την επίδραση της κατανομής δεδομένων και της ποικιλομορφίας στα χαρακτηριστικά γραφής του κάθε χρήστη. Στο~\autoref{fig:chapter5_image511} παρουσιάζονται τα παραπάνω αποτελέσματα.

\begin{figure}[H]
    \centering
    \includegraphics[width=\textwidth]{images/chapter5/5.1.1.png}
    \caption{FAR \& FRR ανά χρήστη για nu: 0.01, gamma: 0.05 και threshold: 0.05}
    \label{fig:chapter5_image511}
\end{figure}

\subsection{Παρατηρήσεις}
\begin{itemize}
    \item Τα καλύτερα αποτελέσματα για ισορροπία παρατηρούνται όταν $\nu$ είναι σχετικά μικρό ($0.005$–$0.01$), $\gamma$ είναι χαμηλό ($0.2$–$0.5$), και το threshold είναι προσεκτικά ρυθμισμένο κοντά στο $0.05$–$0.07$.
    \item Τα παραδείγματα του \autoref{tab:balanced_results} με \textbf{bold} υποδεικνύουν ιδανικές ισορροπίες για εφαρμογές όπου το FAR και το FRR πρέπει να είναι ισοδύναμα.
\end{itemize}
