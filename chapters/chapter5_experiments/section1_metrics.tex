\section{Αξιολόγηση Συστήματος}
\label{sec:exmperiments_metrics}

\subsection{Εισαγωγή}
Το σύστημα δοκιμών έχει σχεδιαστεί για να αξιολογήσει την απόδοση και την αξιοπιστία του συστήματος αυθεντικοποίησης σε διαφορετικά σενάρια. Εστιάζουμε στη συμπεριφορά του συστήματος όταν διαχειρίζεται γνήσιους χρήστες και εισβολείς, αναλύοντας την ακρίβεια και την αποτελεσματικότητά του.

\subsection{Δεδομένα Δοκιμών}
Τα δεδομένα που χρησιμοποιήθηκαν στις δοκιμές περιλαμβάνουν:
\begin{itemize}
    \item \textbf{Γνήσια δεδομένα χρηστών}: Κείμενα από γνήσιους χρήστες που εκπαιδεύτηκαν στο σύστημα.
    \item \textbf{Δεδομένα εισβολέων}: Κείμενα από άλλους χρήστες του dataset εκτός του προφίλ του εκάστοτε γνήσιου χρήστη.
\end{itemize}

\subsection{Μετρικές Αξιολόγησης}
\label{subsec:metrics}
Για την αξιολόγηση του συστήματος χρησιμοποιούνται οι παρακάτω μετρικές:
\begin{itemize}
    \item \textbf{F1 Score):}

    Η μετρική F1 Score είναι μια σταθμισμένη μέση τιμή της ανάκλησης (\textbf{Recall}) και της ακρίβειας (\textbf{Precision}), η οποία χρησιμοποιείται για την αξιολόγηση της απόδοσης ενός συστήματος. Στο πλαίσιο του συστήματος αυθεντικοποίησης, η F1 υπολογίζεται ως εξής:

    \[
    \text{F1} = 2 \cdot \frac{\text{Precision} \cdot \text{Recall}}{\text{Precision} + \text{Recall}}
    \]
    
    Όπου:
    \begin{itemize}
    \item \textbf{Precision (Ακρίβεια)}:
    \[
    \text{Precision} = \frac{\text{True Positives (TP)}}{\text{True Positives (TP)} + \text{False Positives (FP)}}
    \]
    Η ακρίβεια αντιπροσωπεύει το ποσοστό των προτροπών που ταξινομήθηκαν ως γνήσιες (\textit{genuine}) και ήταν πράγματι γνήσιες. Υψηλή ακρίβεια σημαίνει ότι το σύστημα αποφεύγει λανθασμένες αποδοχές εισβολέων (\textit{impostors}).
    
    \item \textbf{Recall (Ανάκληση)}:
    \[
    \text{Recall} = \frac{\text{True Positives (TP)}}{\text{True Positives (TP)} + \text{False Negatives (FN)}}
    \]
    Η ανάκληση μετρά το ποσοστό των πραγματικά γνήσιων προτροπών που αναγνωρίστηκαν σωστά από το σύστημα. Υψηλή ανάκληση δείχνει ότι το σύστημα αποφεύγει να απορρίψει γνήσιες προτροπές ως εισβολείς.
    
    \item \textbf{True Positives (TP)}:
    Προτροπές που το σύστημα ταξινόμησε σωστά ως γνήσιες.
    
    \item \textbf{False Positives (FP)}:
    Προτροπές που το σύστημα ταξινόμησε λανθασμένα ως γνήσιες ενώ ήταν εισβολείς.
    
    \item \textbf{False Negatives (FN)}:
    Προτροπές που το σύστημα ταξινόμησε λανθασμένα ως εισβολείς ενώ ήταν γνήσιες.
    \end{itemize}
    
    \item \textbf{False Rejection Rate (FRR)}: Υπολογίζεται ως:
    \[
    \text{FRR} = \frac{\text{False Rejections}}{\text{Total Genuine Prompts}}
    \]
    \item \textbf{False Acceptance Rate (FAR)}: Υπολογίζεται ως:
    \[
    \text{FAR} = \frac{\text{False Acceptances}}{\text{Total Impostor Prompts}}
    \]
    \item \textbf{Accuracy (Ακρίβεια Συνολική):}
    \[\text{Accuracy} = \frac{\text{True Positives (TP)} + \text{True Negatives (TN)}}{\text{Total Predictions}}\]
    Η συνολική ακρίβεια μετρά το ποσοστό των προτροπών που ταξινομήθηκαν σωστά, είτε ως γνήσιες είτε ως εισβολείς. Υψηλή τιμή ακρίβειας δείχνει τη συνολική αποτελεσματικότητα του συστήματος.

    \item \textbf{MAE (Μέσο Απόλυτο Σφάλμα):}
    \[\text{MAE} = \frac{\sum_{i=1}^{n} |\hat{y}_i - y_i|}{n}\]
    Το MAE μετρά το μέσο απόλυτο σφάλμα μεταξύ των προβλεπόμενων τιμών (\(\hat{y}_i\)) και των πραγματικών τιμών (\(y_i\)). Μια χαμηλή τιμή MAE υποδηλώνει ότι οι προβλέψεις του συστήματος είναι κοντά στις πραγματικές τιμές, καταδεικνύοντας την ακρίβεια και την αξιοπιστία του μοντέλου.
    
    \item \textbf{Μέσος Αριθμός Προτροπών για Γνήσιους Χρήστες}:
    \[
    \text{Mean Genuine Prompts Before Lock} = \frac{\text{Σύνολο Προτροπών Γνήσιων Χρηστών}}{\text{Συνολικός Αριθμός Locks για Γνήσιους}}
    \]
    \item \textbf{Μέσος Αριθμός Προτροπών για εισβολείς}:
    \[
    \text{Mean Impostor Prompts Before Lock} = \frac{\text{Σύνολο Προτροπών εισβολέων}}{\text{Συνολικός Αριθμός Locks για εισβολείς}}
    \]

\end{itemize}

Οι μετρικές FAR και FRR είναι αντιστρόφως ανάλογες που σημαίνει πως όταν η μία αυξάνεται, η άλλη μειώνεται όπως φαίνεται στο παρακάτω σχήμα:
\begin{figure}[H]
    \centering
    \includegraphics[width=0.7\textwidth]{images/chapter4/farVSfar.png}
    \caption{Ποιοτική απεικόνηση των FAR και FRR.}
    \label{fig:chapter4_farVSfar}
\end{figure}
Το κόκκινο σημείο ονομάζεται Equal Error Rate - EER και αναπαριστά το επίπεδο ασφάλειας για το οποίο οι τιμές των FAR και FRR είναι ίσες, όπως φαίνεται στο~\autoref{fig:chapter4_farVSfar}. Ερμηνεύοντας το σχήμα από μια πιο πρακτική προσέγγιση, γίνεται αντιληπτό πως όσο ασφαλέστερο είναι ένα σύστημα, τόσο λιγότερο βολικό θα είναι για τον χρήστη καθώς θα κλειδώνει συχνότερα και, αντίστοιχα, όσο λιγότερη ασφάλεια το χαρακτηρίζει, τόσο ευκολότερο θα είναι για τον πραγματικό χρήστη να το χειριστεί, αλλά και για τον υποκλοπέα να αυθεντικοποιηθεί από το σύστημα.

\subsection{Ροή Διαδικασίας Δοκιμών}
Η διαδικασία δοκιμών ακολουθεί συγκεκριμένα βήματα, διασφαλίζοντας την αξιολόγηση κάθε μοντέλου και χρήστη ξεχωριστά. Παρακάτω περιγράφονται τα στάδια της ροής:

\subsubsection{Προετοιμασία δεδομένων} 
Η προετοιμασία περιλαμβάνει τη δημιουργία δύο συνόλων δεδομένων για κάθε χρήστη:
\begin{itemize}
    \item \textbf{Διαχωρισμός Δεδομένων Χρήστη:} Για κάθε χρήστη, το 15\% των δεδομένων του αφαιρείται από το σύνολο εκπαίδευσης και διατίθεται ως σύνολο δοκιμών (\emph{test set}). Το υπόλοιπο 85\% έχει ήδη χρησιμοποιηθεί για την εκπαίδευση των μοντέλων, όπως περιγράφεται στο Κεφάλαιο~\ref{sec:implementation_train}.
    \item \textbf{Δημιουργία Impostor Dataset:} Για κάθε χρήστη, επιλέγεται ένα σύνολο από κείμενα που ανήκουν σε άλλους χρήστες. Αυτά τα δεδομένα σχηματίζουν το \emph{impostor dataset}, το οποίο χρησιμοποιείται για την αξιολόγηση της ικανότητας του συστήματος να εντοπίζει μη γνήσιους χρήστες. Το μέγεθος του impostor test set αυτού είναι ίσο με το μέγεθος του genuine test set του κάθε χρήστη (50-50 split).
\end{itemize}

\subsubsection{Χρήση Εκπαιδευμένων Μοντέλων}
Τα εκπαιδευμένα μοντέλα χρησιμοποιούνται για να αξιολογήσουν τα κείμενα του \emph{genuine test set} κάθε χρήστη και του \emph{impostor test set}. Για κάθε κείμενο:
\begin{itemize}
    \item Εξάγονται τα χαρακτηριστικά της κάθε εγγραφής των 2 test set (genuine \& impostor) μέσω της συνάρτησης \texttt{extract\_features}, όπως περιγράφεται στο Κεφάλαιο~\ref{sec:implementation_features}.
    \item Τα χαρακτηριστικά εισάγονται στα εκπαιδευμένα μοντέλα (\emph{One-Class SVM}) για την παραγωγή απόφασης και του επιπέδου βεβαιότητας (\emph{certainty level}).
\end{itemize}

\subsubsection{Συλλογή Αποτελεσμάτων}
Η συλλογή των αποτελεσμάτων γίνεται σε δύο επίπεδα:
\begin{itemize}
    \item \textbf{Αποτελέσματα Ανά Χρήστη:} Για κάθε χρήστη καταγράφονται:
    \begin{itemize}
        \item Οι τιμές των μετρικών αξιολόγησης (\emph{FRR}, \emph{FAR}, μέσος αριθμός προτροπών πριν το κλείδωμα για γνήσιους χρήστες, μέσος αριθμός προτροπών πριν το κλείδωμα για εισβολείς).
        \item Οι τιμές βεβαιότητας (\emph{certainty scores}) για κάθε κείμενο.
    \end{itemize}
    \item \textbf{Συνολικά Αποτελέσματα:} Τα αποτελέσματα όλων των χρηστών συνδυάζονται για την εξαγωγή συνολικών στατιστικών, παρέχοντας μια ολοκληρωμένη εικόνα της απόδοσης του συστήματος.
\end{itemize}

Η διαδικασία δοκιμών διασφαλίζει την αντικειμενική αξιολόγηση της απόδοσης του συστήματος και την καταγραφή των μετρικών σε επίπεδο χρήστη αλλά και συνολικά για το σύστημα.