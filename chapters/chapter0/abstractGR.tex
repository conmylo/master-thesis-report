\begin{center}
  \centering

  \vspace{0.5cm}
  \centering
  \textbf{\Large{Περίληψη}}
  \phantomsection
  \addcontentsline{toc}{section}{Περίληψη}

  \vspace{1cm}

\end{center}

 Η αυθεντικοποίηση χρηστών αποτελεί κρίσιμη πτυχή της ασφάλειας σε κάθε ψηφιακό περιβάλλον, με τις παραδοσιακές μεθόδους, όπως οι κωδικοί πρόσβασης, τα PIN και οι ερωτήσεις ασφαλείας, να παρουσιάζουν περιορισμούς όσον αφορά την απρόσκοπτη ενσωμάτωση και τη χρηστικότητα. Αντίθετα, ο έμμεσος έλεγχος ταυτότητας επιδιώκει τη συνεχή και διακριτική ταυτοποίηση των χρηστών με βάση τις μοναδικές συμπεριφορές τους, όπως το στυλ γραφής τους, επιτυγχάνοντας υψηλότερα επίπεδα ασφάλειας χωρίς να επηρεάζεται η ροή εργασίας τους. Ωστόσο, η προσέγγιση αυτή αντιμετωπίζει σημαντικές προκλήσεις, καθώς η γραφή ενός ατόμου μπορεί να επηρεάζεται από διάφορους παράγοντες, όπως η συναισθηματική κατάσταση, η κούραση ή ο τύπος του περιεχομένου που παράγει.

Η παρούσα διπλωματική εργασία επικεντρώνεται στην ανάπτυξη ενός συστήματος συνεχούς έμμεσης αυθεντικοποίησης με βάση τα γλωσσικά χαρακτηριστικά, το οποίο βρίσκει εφαρμογές στα μέσα κοινωνικής δικτύωσης, στο ηλεκτρονικό ταχυδρομείο και στα συστήματα συνομιλίας. Χρησιμοποιεί τεχνικές Επεξεργασίας Φυσικής Γλώσσας και Μηχανικής Μάθησης για την ταυτοποίηση των χρηστών μέσω της ανάλυσης των χαρακτηριστικών του γραπτού τους κειμένου. Το σύστημα αποσκοπεί στην ενίσχυση της ασφάλειας σε περιβάλλοντα όπου οι παραδοσιακές μέθοδοι ελέγχου ταυτότητας μπορεί να είναι ακατάλληλες ή μη επαρκείς. Η προτεινόμενη λύση χρησιμοποιεί μοντέλα μηχανών διανυσμάτων υποστήριξης μίας κλάσης για την εκμάθηση εξατομικευμένων μοτίβων γραφής, αξιοποιώντας χαρακτηριστικά όπως η λεξική μορφολογία, η χρήση λεξιλογίου και τα συντακτικά και σημασιολογικά πρότυπα.

Στο πλαίσιο αυτής της εργασίας, διεξήχθη εκτεταμένη ανάλυση των χαρακτηριστικών που διαχωρίζουν μοναδικά τους χρήστες μέσω της γραφής τους, συμπεριλαμβανομένης της συχνότητας χρήσης λέξεων, της σύνθεσης προτάσεων και στατιστικών μέτρων όπως δείκτες πολυπλοκότητας και ποσοστά παύσεων. Τα δεδομένα που συλλέχθηκαν από τους χρήστες ενσωματώθηκαν σε ένα σετ δεδομένων για την εκπαίδευση και την αξιολόγηση του μοντέλου. Το σύστημα αξιολογήθηκε με βάση μετρικές όπως το ποσοστό ψευδούς απόρριψης και το ποσοστό ψευδούς αποδοχής, ενώ πραγματοποιήθηκαν πρόσθετες δοκιμές για την αξιολόγηση της ανθεκτικότητάς του σε διάφορα περιβάλλοντα και σενάρια χρήσης.

Τα ευρήματα αυτής της έρευνας καταδεικνύουν ότι το σύστημα επιτυγχάνει υψηλά επίπεδα ακρίβειας και απόκρισης, ισορροπώντας αποτελεσματικά την ασφάλεια και τη χρηστικότητα. Εξαλείφοντας την ανάγκη για συνεχή παρέμβαση του χρήστη, το σύστημα προσφέρει μια διακριτική αλλά ενσωματωμένη μέθοδο για αυθεντικοποίηση, καθιστώντας το κατάλληλο για εφαρμογές σε ένα διαρκώς εξελισσόμενο ψηφιακό περιβάλλον. 

