{\fontfamily{cmr}\selectfont

\phantomsection
\addcontentsline{toc}{section}{Abstract}

\begin{center}
  \centering
  \textbf{\Large{Title}}
  \vspace{0.5cm}

  \textbf{\large{Development of a Continuous Implicit Authentication System Based on Linguistic and Behavioural Features}}

  \vspace{1cm}

  \centering
  \textbf{Abstract}
\end{center}
\begin{sloppypar}

    User authentication is a critical aspect of security in any digital environment, with traditional authentication methods, such as passwords, PINs and security questions, having limitations in terms of seamless integration and usability. In contrast, implicit authentication seeks to identify users continuously and discreetly based on their unique behaviours, such as their writing style, thus achieving higher levels of security without affecting their workflow. However, this approach faces significant challenges, as an individual's writing can be affected by a variety of factors, such as their emotional state, level of fatigue or the type of content they produce.

    This thesis focuses on the development of a continuous implicit authentication system based on linguistic features, which finds applications in social media, email and chat systems. It uses Natural Language Processing and Machine Learning techniques to identify users by analyzing the features of their written text. The system aims to enhance security in environments where traditional authentication methods may be inappropriate or insufficient. The proposed solution uses one-class support vector machine models to learn personalized writing patterns by exploiting features such as lexical morphology, vocabulary usage, and syntactic and semantic patterns.

    As part of this work, an extensive analysis of the features that can be used to uniquely identify users through their writing was conducted, including word usage frequency, sentence composition, and statistical measures, such as complexity index and pause ratio. The data collected from users was incorporated into a dataset for model training and evaluation. The system was evaluated using metrics such as false rejection rate and false acceptance rate, and additional tests were conducted to evaluate its robustness in different environments and usage scenarios.

    The findings of this research demonstrate that the system can achieve high levels of accuracy and adaptability, while effectively balancing safety and usability. By eliminating the need for constant user intervention, the proposed system offers an unobtrusive yet integrated method for authentication, making it suitable for applications in an ever-evolving digital environment. 

\end{sloppypar}
\begin{flushright}
  \vspace{2cm}
  Konstantinos Mylonas
  \\
  Electrical \& Computer Engineering Department,
  \\
  Aristotle University of Thessaloniki, Greece
  \\
  December 2024
\end{flushright}

}
