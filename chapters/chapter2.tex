% Ο συνεχής και έμμεσος έλεγχος ταυτότητας μέσω ανάλυσης γραφής αποτελεί έναν ταχύτατα αναπτυσσόμενο τομέα έρευνας στον οποίο συνδυάζονται τεχνικές επεξεργασίας φυσικής γλώσσας και μηχανικής μάθησης. Η συγκεκριμένη προσέγγιση παρέχει νέες προοπτικές ασφάλειας, καθώς είναι ιδιαίτερα αποτελεσματική για την ταυτοποίηση χρηστών σε πραγματικό χρόνο. Στην ενότητα αυτή, παρουσιάζονται προηγούμενες έρευνες που διαμόρφωσαν την περιοχή, υπογραμμίζοντας τη σημασία της ανάλυσης γραφής για την αυθεντικοποίηση χρηστών.

% Η χρήση NLP για την εξαγωγή χαρακτηριστικών έχει επιτρέψει την ποσοτικοποίηση της γλωσσικής συμπεριφοράς, καθιστώντας δυνατή την εκπαίδευση μοντέλων μηχανικής μάθησης για τη διάκριση μεταξύ χρηστών. Χαρακτηριστική είναι η εργασία των Argamon et al.~\cite{argamon2003linguistic}, η οποία παρουσίασε μια δομημένη προσέγγιση εξαγωγής χαρακτηριστικών με βάση τη γλωσσολογική ανάλυση. Παράλληλα, οι Schwartz et al.~\cite{schwartz2013authorship} χρησιμοποίησαν τεχνικές TF-IDF και SVM για τη βελτίωση της ακρίβειας αναγνώρισης συγγραφέων, ενώ οι Zhang et al.~\cite{zhang2021bert} εστίασαν στη χρήση μετασχηματιστών όπως το BERT και το GPT για ανάλυση συγγραφικής ταυτότητας.

% Η μηχανική μάθηση έχει συμβάλει καθοριστικά στην πρόοδο της περιοχής, επιτρέποντας την εφαρμογή προηγμένων τεχνικών ανίχνευσης ανωμαλιών. Οι Schölkopf et al.~\cite{scholkopf2001ocsvm} ανέπτυξαν τον αλγόριθμο One-Class SVM, ο οποίος αποτελεί θεμέλιο λίθο για την ανάπτυξη εξατομικευμένων μοντέλων αυθεντικοποίησης. Πιο πρόσφατα, οι Ferrante και Marone~\cite{ferrante2022stylometric} επικεντρώθηκαν στη χρήση βαθιάς μάθησης για τη δημιουργία στιλιστικών προφίλ συγγραφέων, ενώ οι Gopalakrishnan et al.~\cite{gopalakrishnan2021explainable} ασχολήθηκαν με την έννοια της ερμηνευσιμότητας μέσω Explainable AI.

% Παρά τις προόδους, οι προκλήσεις παραμένουν σημαντικές. Η ανάγκη για σταθερά δεδομένα γραφής, η ευαισθησία στις αλλαγές του στυλ γραφής και η δυνατότητα αντιμετώπισης της πολυγλωσσικότητας αποτελούν ανοιχτά ζητήματα που εξετάζονται από τους Karpathy et al.~\cite{karpathy2018deep}. Η αντιμετώπιση αυτών των ζητημάτων είναι κρίσιμη για τη βελτίωση των συστημάτων συνεχούς αυθεντικοποίησης.

% Ο παρακάτω πίνακας παρουσιάζει μια αναλυτική επισκόπηση των κυριότερων ερευνητών και των μεθόδων που έχουν αναπτυχθεί στον τομέα της αναγνώρισης γραφής και συγγραφικής ταυτότητας.

% Ο πίνακας καταδεικνύει τη συνεχή πρόοδο και τη δυναμική των μεθόδων που εφαρμόζονται στον τομέα της ανάλυσης γραφής. Υπογραμμίζεται η θεμελιώδης σημασία της συνδυαστικής χρήσης NLP και μηχανικής μάθησης για την επίτευξη ακριβών και αξιόπιστων συστημάτων αυθεντικοποίησης. Επιπλέον, οι τεχνικές που αναφέρονται παρέχουν τη θεωρητική βάση για την ανάπτυξη του συστήματος που προτείνεται στην παρούσα εργασία, συνεισφέροντας στην κατανόηση των δυνατοτήτων της ανάλυσης γραφής σε εφαρμογές ασφάλειας.


\chapter{Επισκόπηση της Ερευνητικής Περιοχής}
\label{chapter:sota}

Ο συνεχής και έμμεσος έλεγχος ταυτότητας μέσω ανάλυσης γραφής αποτελεί έναν ταχύτατα αναπτυσσόμενο τομέα έρευνας, στον οποίο συνδυάζονται τεχνικές Επεξεργασίας Φυσικής Γλώσσας και Μηχανικής Μάθησης. Η συγκεκριμένη προσέγγιση παρέχει νέες προοπτικές ασφάλειας, καθώς είναι ιδιαίτερα αποτελεσματική για την ταυτοποίηση χρηστών σε πραγματικό χρόνο. Στην ενότητα αυτή, παρουσιάζονται προηγούμενες έρευνες που διαμόρφωσαν την περιοχή, υπογραμμίζοντας τη χρονολογική εξέλιξή της και τη σημερινή ερευνητική κατάσταση.

Η επεξεργασία φυσικής γλώσσας συνδυάζει γλωσσολογικές και υπολογιστικές μεθόδους για την ανάλυση και κατανόηση του γραπτού λόγου. Οι Argamon et al. \cite{argamon2003linguistic} διερεύνησαν πώς το φύλο, το είδος, και το στυλ γραφής επηρεάζουν τα γλωσσικά χαρακτηριστικά, παρέχοντας σημαντικές γνώσεις για τη συσχέτιση κοινωνικών παραμέτρων με τη γραπτή έκφραση. Ο Van Haltern \cite{vanhaltern2003profiling} ανέπτυξε μεθόδους γλωσσικού προφίλ που εστιάζουν στη συχνότητα λέξεων και φράσεων, διευκολύνοντας την ταυτοποίηση συγγραφέων. Ο Hoover \cite{hoover2002wordfreq} έδειξε ότι η συχνότητα λέξεων μπορεί να είναι ένας ισχυρός δείκτης για την αναγνώριση συγγραφικού στυλ, ενώ ο Juola \cite{juola2008matchlength} πρότεινε τεχνικές που βασίζονται στο μήκος αντιστοιχιών για τη βελτίωση της ακρίβειας στην ανάλυση γραπτών. Οι Kesel και Cercone \cite{kesel2010cng} εισήγαγαν την προσέγγιση Common-N-Grams (CNG) με βαρύτητα ψήφων, συνδυάζοντας στατιστικές και υπολογιστικές μεθόδους για την ταυτοποίηση συγγραφέων. Οι Zhang et al. \cite{zhang2021bert} χρησιμοποίησαν μοντέλα μετασχηματιστών για την αναγνώριση συγγραφικού στυλ, αποδεικνύοντας τη δύναμη της βαθιάς μάθησης στην ανάλυση κειμένου. Οι Coburn και Fitzpatrick \cite{coburn2020contextual} συνδύασαν τεχνικές NLP με θεωρία δικτύων για την ταυτοποίηση συγγραφέων, ενώ οι Schwartz et al. \cite{schwartz2013authorship} ανέλυσαν τη γλώσσα κοινωνικών δικτύων, δείχνοντας πώς η προσωπικότητα και το φύλο αντικατοπτρίζονται στη χρήση της γλώσσας.

Η πρόοδος στη μηχανική μάθηση και ειδικότερα στα μοντέλα SVM έχει οδηγήσει σε σημαντικές εξελίξεις σε διάφορους τομείς ανάλυσης δεδομένων. Οι Schölkopf et al. \cite{scholkopf2001ocsvm} εισήγαγαν το μοντέλο One-Class SVM για την εκτίμηση κατανομών υψηλών διαστάσεων, προσφέροντας έναν καινοτόμο τρόπο ανίχνευσης αποκλίσεων σε δεδομένα. Οι Manevitz και Yousef \cite{manevitz2002oneclass} εξειδίκευσαν τη χρήση του One-Class SVM για την ταξινόμηση εγγράφων, δείχνοντας τη χρησιμότητά του σε προβλήματα με περιορισμένα παραδείγματα δεδομένων. Παράλληλα, οι Laskov et al. \cite{laskov2004intrusion} εφάρμοσαν kernel-based learning methods για την ανίχνευση εισβολών, εισάγοντας έναν συνδυασμό πυρήνων για τη βελτίωση της ακρίβειας και της αποτελεσματικότητας των συστημάτων ασφαλείας. Οι Ferrante και Marone \cite{ferrante2022stylometric} αξιοποίησαν την τεχνολογία SVM για την ανάλυση στυλ, αναπτύσσοντας τεχνικές βαθιάς μάθησης που συνδυάζουν παραδοσιακές και σύγχρονες μεθόδους. 

Η χρήση των SVM επεκτάθηκε περαιτέρω από τους Xu et al. \cite{xu2013novel}, οι οποίοι ανέπτυξαν συνδυαστικές μεθόδους πυρήνων για την ανίχνευση ανωμαλιών σε πολυδιάστατα δεδομένα. Οι Baronchelli και Altmann \cite{baronchelli2020entropy} παρουσίασαν μέτρα βασισμένα στην εντροπία για τη ροή πληροφορίας σε δίκτυα, αξιοποιώντας την ικανότητα των SVM να μοντελοποιούν πολύπλοκα μοτίβα. Οι Gopalakrishnan et al. \cite{gopalakrishnan2021explainable} χρησιμοποίησαν εξηγήσιμα μοντέλα SVM για την αυθεντικοποίηση χρηστών, ενισχύοντας τη διαφάνεια στις αποφάσεις των συστημάτων. Επιπλέον, οι Hong et al. \cite{hong2008fingerprint} συνδύασαν SVM με Bayesian classifiers για τη βελτίωση της ακρίβειας στην ταξινόμηση βιομετρικών δεδομένων, αποδεικνύοντας τη χρησιμότητα των SVM σε εφαρμογές βιομετρίας.

Η μηχανική μάθηση έχει διαδραματίσει σημαντικό ρόλο στην ανίχνευση ανωμαλιών, με τις εφαρμογές να επεκτείνονται σε διάφορους τομείς, όπως η ασφάλεια συστημάτων, η βιομηχανική παρακολούθηση και η ανάλυση βιομετρικών δεδομένων. Οι Lu et al. \cite{lu2010tcfom} παρουσίασαν ένα πλαίσιο ταξινόμησης κυκλοφορίας δεδομένων με τη χρήση OC-SVM, εισάγοντας τεχνικές που επιτρέπουν τη διάκριση κανονικών και μη κανονικών μοτίβων. Παράλληλα, οι Chen et al. \cite{chen2016software} χρησιμοποίησαν OC-SVM για την πρόβλεψη βλαβών λογισμικού, αποδεικνύοντας την αξία των μοντέλων αυτών για την αύξηση της αξιοπιστίας των συστημάτων. Επιπλέον, οι Fong και Narasimhan \cite{fong2021unsupervised} αξιοποίησαν εργαλεία μη επιβλεπόμενης μάθησης για την ανίχνευση ανωμαλιών, επιδεικνύοντας τη δύναμη αυτών των τεχνικών στην πρόβλεψη και την παρακολούθηση υποδομών. Οι Narukawa et al. \cite{narukawa2017realtime} ανέπτυξαν συστήματα ανίχνευσης κινδύνων σύγκρουσης σε ρομποτικά περιβάλλοντα, προσαρμόζοντας τα SVM για τη διαχείριση πραγματικού χρόνου. Οι Sun et al. \cite{sun2017abnormal} χρησιμοποίησαν deep OC-SVM για την ανίχνευση ανωμαλιών σε δεδομένα βίντεο, ενώ οι Seo \cite{seo2007application} επέκτειναν τη χρήση των SVM σε εφαρμογές περιεχομένου εικόνων, εστιάζοντας στην αναγνώριση μη κανονικών προτύπων. Ειδικότερα, οι Hayashi και Ruggiero \cite{hayashi2023handsfree} παρουσίασαν hands-free αυθεντικοποίηση, αξιοποιώντας SVM και δεδομένα IoT για την ανίχνευση ανωμαλιών σε πραγματικό χρόνο. Επιπλέον, οι Rabaoui et al. \cite{rabaoui2008using} ανέδειξαν τη χρησιμότητα των OC-SVM για την ανάλυση ήχου, αποδεικνύοντας τη δυνατότητα εφαρμογής τους σε περιβάλλοντα επιτήρησης.

Ο συνδυασμός της μηχανικής μάθησης και του NLP έχει προσφέρει νέες προοπτικές στην ανίχνευση ανωμαλιών και τη βελτίωση των συστημάτων ασφάλειας. Οι Chatzikyriakidis και Papageorgiou \cite{chatzikyriakidis2021transformer} αξιοποίησαν μοντέλα μετασχηματιστών για την ανάλυση ελληνικών κειμένων, προσφέροντας νέες μεθόδους ανίχνευσης αποκλίσεων μέσω γλωσσικών μοτίβων. Οι Karanikiotis et al. \cite{karanikiotis2020continuous} παρουσίασαν ένα σύστημα συνεχούς αυθεντικοποίησης μέσω ανάλυσης δεδομένων αφής, ενώ οι Schwartz et al. \cite{schwartz2013authorship} εστίασαν στη γλώσσα κοινωνικών δικτύων, δείχνοντας πώς η ανάλυση της γλώσσας μπορεί να συνδεθεί με την ταυτοποίηση χρηστών και τη βελτίωση της ασφάλειας. Οι Karpathy και Fei-Fei \cite{karpathy2018deep} συνδύασαν τη γλωσσική ανάλυση και την αναγνώριση εικόνων για τη δημιουργία περιγραφών, εισάγοντας συστήματα που μπορούν να αναγνωρίζουν ανωμαλίες σε πολυτροπικά δεδομένα. Επιπλέον, ο Stylios \cite{stylios2023behavioral} ανέλυσε τη χρήση βιομετρικών χαρακτηριστικών για την ανίχνευση μη κανονικών συμπεριφορών σε συστήματα συνεχούς αυθεντικοποίησης. Οι Hong et al. \cite{hong2008fingerprint} συνδύασαν SVM και Bayesian methods για την ταξινόμηση βιομετρικών δεδομένων, ενώ οι Ferrante και Marone \cite{ferrante2022stylometric} χρησιμοποίησαν βαθιά μάθηση και SVM για στυλιστική ανάλυση. Τέλος, οι Gopalakrishnan et al. \cite{gopalakrishnan2021explainable} εισήγαγαν μοντέλα XAI για την ανίχνευση ανωμαλιών σε περιβάλλοντα αυθεντικοποίησης, προσφέροντας μεγαλύτερη διαφάνεια στις αποφάσεις.

Τα εργαλεία που έχουν αναπτυχθεί για την υποστήριξη εφαρμογών μηχανικής μάθησης και ανάλυσης δεδομένων αποτελούν κρίσιμο παράγοντα για την πρόοδο στον τομέα. Ο Church \cite{church2017word2vec} εξέτασε το Word2Vec, τονίζοντας την αποτελεσματικότητά του στη δημιουργία σημασιολογικών αναπαραστάσεων λέξεων, που αποτελούν θεμέλιο για πολλές εφαρμογές. Το GloVe, όπως αναπτύχθηκε από τους Pennington et al. \cite{pennington2014glove}, εισήγαγε τη σύνδεση της στατιστικής συχνότητας λέξεων με τη σημασιολογία, παρέχοντας βελτιωμένες δυνατότητες ανάλυσης σε εφαρμογές NLP. Η πλατφόρμα TensorFlow, που αναπτύχθηκε από την ομάδα TensorFlow \cite{tensorflow2015-large-scale}, εισήγαγε υποδομές για την εκπαίδευση μεγάλων μοντέλων μηχανικής μάθησης σε διανεμημένα περιβάλλοντα, προσφέροντας ευελιξία και αποτελεσματικότητα στην επεξεργασία μεγάλων συνόλων δεδομένων. Παράλληλα, οι Devlin et al. \cite{devlin2018bert} εισήγαγαν το BERT, ένα εργαλείο βαθιάς μάθησης που άλλαξε το πεδίο της κατανόησης φυσικής γλώσσας μέσω της αμφίδρομης επεξεργασίας δεδομένων. Τέλος, οι Wolf et al. \cite{wolf2020transformers} παρουσίασαν τη βιβλιοθήκη Transformers, η οποία παρέχει ένα ευρύ φάσμα προ-εκπαιδευμένων μοντέλων για την επεξεργασία φυσικής γλώσσας, επιτρέποντας την εύκολη εφαρμογή σύγχρονων τεχνικών NLP.  

Συνολικά, η έρευνα στον τομέα της ανάλυσης κειμένου και της έμμεσης αυθεντικοποίησης έχει εξελιχθεί από απλές μεθόδους στατιστικής ανάλυσης σε σύνθετα μοντέλα βαθιάς μάθησης και μετασχηματιστές. Το παρόν έργο αξιοποιεί αυτές τις εξελίξεις για τη δημιουργία ενός σύγχρονου συστήματος συνεχούς αυθεντικοποίησης, συνεισφέροντας στην ασφάλεια και τη βελτίωση της εμπειρίας χρήστη.

